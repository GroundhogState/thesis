Energy level shifts in the dipole potential
\begin{equation}
\begin{split}
	\Delta E_g &= \frac{\hbar\Omega^2}{4\delta}\\
	\Delta E_e &= -\frac{\hbar\Omega^2}{4\delta}
\end{split}
\end{equation}
Dipole potential depth (1D)
\begin{equation}
	V_{Lat} = \frac{3\pi c^2}{2{\omega_0}^3}\frac{\Gamma}{\delta}\frac{2P}{\pi {w_0}^2} 
\end{equation}

Full potential in 3D
\begin{equation}
	V(\vec{x}) = -\sum_{i=1}^3 V_{x_i} e^{-2\frac{{\vec{r_i}}^2}{{w_i}^2}} \sin^2(k x_i)
\end{equation}

Where the $\vec{r_i}$ are the smallest perpendicular distance from the axis of the $i^{th}$ beam, and $w_i$ is its beam waist size. 
For traps very close to the beam axes, the following linearization is useful
\begin{equation}
	V(\vec{x}) = -\sum_{i=1}^3\Big( V_{x_i} \sin^2(k x_i) + \frac{m}{2}(\omega_i x_i)^2\Big)
\end{equation}
Where the second term corresponds to the global harmonic trapping by the Gaussian beam.

In an isotropic lattice,
\begin{equation}
	\omega_{lat} ^2 = V_{lat} \frac{2 k^2}{m} = \frac{V_{lat}}{E_r}\frac{\hbar^2 k^4}{m^2}
\end{equation}
 are the trapping frequencies of the lattice sites. At the centre of our lattice we should achieve trapping frequencies up to 1.4MHz at individual lattice sites. 

However the spatial variation in beam intensity leads to a position-dependent trapping frequency and spatially-varying ground states (c.f. $V_{GS} = \frac{\hbar}{2}(\hbar^2 \vec{k}(\vec{x})\cdot\vec{k}(\vec{x}) $))
More specifically, along the x-axis for example, the trapping frequencies vary as
\begin{equation}
\begin{split}
	\omega_{lat_x}(x) &= \omega_{lat_x}(0) e^{-\frac{y^2+z^2}{w_x ^2}}\\
	& \approx \omega_{lat_x}(0)\big(1-\frac{y^2+z^2}{w_x ^2}\big)
\end{split}
\end{equation}

The global confining potential from the Gaussian beam, and the corresponding overall trap frequency, are given by 
\begin{equation}
\begin{split}
	V_{ext} &= E_r \frac{2 r^2}{w_0^2}\big( 2\frac{V_l}{E_r} -\sqrt\frac{V_l}{E_r}\big)\\
	\omega_{ext}^2 &= \frac{4 E_r^2}{m w_0} \big(2\frac{V_l}{E_r} - \sqrt\frac{V_l}{E_r}\big)
\end{split}
\end{equation}

Where they also provide the following approximation for relatively large lattice potentials.
\begin{equation}
J \approx \frac{4}{\sqrt{\pi}} E_r\Big(\frac{V_0}{E_r}\Big)^{3/4}e^{-2\sqrt{V_0/E_r}}
\end{equation}
Which is valied to within 10\% for $V_0>15E_r$. \cite{MPSBand,MBPhys}

In the classic paper, Jaksch et al~\cite{ColdBosons} provide the following approximation
\begin{equation}
	U = \hbar \bar{\omega}\frac{a_s}{\bar{a_0}}/\sqrt{2\pi}
\end{equation}
Where $a_s$ is the scattering length, $a_0$ is the size of the ground state wavefunction, and the overbars indicate the geometric mean of the respective quantities. If we consider our lattice in such a configuration that each beam is operating at, say 30$E_R$, then this expression produces $ U = 4.27E^{-30}J = 0.311 E_R$

In the 2006 review on BEC dynamics~\cite{BECDyn}, we find a treatment of the limits of high and low optical potentials for the linear (non-interacting?) regime of a lattice-confined BEC. In particular, the tight-binding limit of the lowest energy band is
\begin{equation}
\begin{split}
	\frac{E(q)}{E_R} &= \sqrt{s} - 2J\cos(qd)\\
	J &= \frac{4}{\sqrt{\pi}} (s)^{3/4} e^{-2\sqrt{s}}\\
	s&=V_lat/E_R
\end{split}
\end{equation}
Note that this paper considers a potential of the form $V_{lat} \cos^2(kx)$, where $d=\lambda/2$ is the lattice periodicity. It seems that this J is the tight-binding approximation of the tunnelling energy, but this is unclear. In any case, the lattice amplitude is the same. In the case of our lattice this should be about $5.4 E_r$ with a very weak dependence on $q$. That is, the atom is strongly confined to a particular lattice site. 


In the weak-binding case, the result from Ashcroft and Mermin (1976) is quoted,
\begin{equation}
	\frac{E(\bar{q})}{E_R} = \bar{q}^2 \pm\sqrt{4\bar{q}^2+\frac{s^2}{16}}
\end{equation}
Where $\bar{q} = q/k-1, s=V_0/E_R$, and the plus (minus) sign provides the energy of the excited (ground) state, resp.