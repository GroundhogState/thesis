

% SO. This section has been 'good-ified' about 1/3 of the way. That leaves too much yet to be done. How fast can we get this?
% It will be helpful to focus each section. The headers:



\chapter*{Overview}
\markboth{OVERVIEW}{}


\begin{adjustwidth}{3cm}{0cm}
\begin{flushright}
{\emph{`In the weeks that had just passed, Commander Norton had often wondered what he would say at this moment.
	But now that it was upon him, history chose his words, and he spoke almost automatically, \\barely aware of the echo from the past: ~``Rama Base.\\ \emph{Endeavour} has landed."'}\\ 
- Arthur C Clarke\footnote{\emph{Rendezvous with Rama}, Publisher info}}
\end{flushright}
\end{adjustwidth}

\section*{Prologue}\label{sec:prologue}
\addcontentsline{toc}{section}{Prologue}  




	% %Image of the 2017 total solar eclipse, taken 149 years minus three days after the first observation of the Helium 587nm line

	\dropcap{Splitting} a ray of light from the solar chromosphere during the total eclipse of 1868, Pierre Jules C\'{e}sar Jansen resolved a bright yellow line through a spectroscope\footnote{The emission peak at 587.5618 nm Initially called the D$_3$ line after its proximity to the D1 and D2 Fraunhofer lines of sodium}. As no element known on earth emitted this colour, a new element was identified and named Helium after the greek sun titan, Helios. Helium is now understood to comprise some 24 per cent of the ordinary matter in the universe, outweighing the sum of all heavier elements, and to consist primarily of a primordial nuclear $\alpha$ particle neutralized by two electrons. To this day, Helium remains a nucleation point of cosmological knowledge. Spectrometry of the atmosphere of WASP-107b revealed absorption at of light from its parent star at 1083.331 nm, intrinsic to helium, and led to the ascertainment of the exoplanet's atmospheric erosion rate of some $10^{10}-3\times10^{11}$ grams per second \cite{Spake18}. On earth, the 1083.331 nm absorption line is employed in a handful of laboratories around the world to drive helium towards a new extremum in the cosmos. While helium fusion occurs at some 10$^8$ kelvin in the furnaces at the centre of giant stars, and the near-vacuum conditions in the Boomerang nebula reach a single degree kelvin, the helium studied in this thesis momentarily sustains temperatures as low as $10^{-8}$ kelvin. 

	Deep in the ultracold regime, dilute gases take on the unfamiliar character of quantum degeneracy, departing from the familiar ideal gas in the sense that the spin-symmetry of the constituent atoms now determines the statistical features of the ensemble. Atoms with integer spin $n$ are Bosons and are not bound by the Pauli exclusion principle as Fermions, with half-integer spin $\frac{1}{2}(2n+1)$, are. The quantum-degenerate behaviour of dilute bosonic gases has the character of a collection of atoms residing in a common quantum state, behaving as waves with a length scale larger than the space between the atoms, and with an emergent order distilled by evaporating away the chaos of thermal motion. The subtle interactions between such wave-like bosonic atoms give rise to exotic collective behaviours including the frictionless flow of superfluidity.

	Highly ordered, quiescent, and exquisitely isolated, ultracold dilute gases present scientists with almost perfectly idealized conditions to study the structure of matter, its interaction with light, and the emergence of collective phenomena from constituents. This thesis touches on each of these themes in turn: First, by extending the proud history of optical spectroscopy in helium; second, by measuring the frequency of the tune-out point near 413 nm with sufficienct accuracy to check the veracity of state-of-the-art calculations in quantum electrodynamics; and finally, by critically examining the limitations of ubiquitous imaging techniques for studying weakly-interacting gases.

\subsection*{Indivisible and unattainable}

	% Atomic theory; dark clouds resolved by new atomic-structural theory, synthesized with special relativity; the other cloud resolved by the birth of quantum stat mech, the spin-statistics theorem and leading to the flourishing of modern condensed matter theory
	% 	Theme of this resolution is the quantum physics; relativity only relevant insofar as QED

	Although his writings are lost to history, the greek philosopher Democritus is remembered for his hypothesis that there was a smallest thing: That one could break mountains into boulders, boulders into stones, stones to sand, and ... to something indivisible. He called these \emph{atomos}, for indivisible.	This was astoundingly prescient: The atomic theory, as it came to be known, would not find empirical validation for another millenia or so.	And, like all theories that prove to be correct, it too reached its point of failure a few hundred years thereafter. 

	The early validation of atomic theory came from the succes of the kinetic theory of gases in explaining the empirical laws of Avogadro, Boyle, and Gay-Lussac, and their synthesis in the ideal gas law. This led to new understandings of the ways to extract, harness, and apply energy in greater and greater quantities, propelling the process of industrialization and shaping the world of today. . Although Boyle himself raised the prospect of a minimum absolute temperature, the first estimation of it value in celsius was made by Guillaume Amontons by extrapolating the contraction of a cooling air column to the point where its volume would vanish: -240 $^\circ$C. This was improved by Johann Lambert to the value -270 $^\circ$C, close to the present value of $-273.15~^\circ$C. \footnote{Amontons was right about one thing, though: The absolute zero of temperature is an unattainable asymptote, https://www.nature.com/articles/ncomms14538... What are the consequences of information erasure c.f. landauer and entanglement 'generation' - how does that relate to erasure}
	The children of 2050 may grow up in a world shaped by an analogous explosion of comprehension and deployment of \emph{quantum} resources, of which the familiar `classical' states of systems in general are nearly devoid to us by virtue of being highly entangled with one another and a shared environment, and are sustaining quantum states/distilling and expending quantum resources entanglement/coherence/etc; (while quantum computing may be the poster child for the dawn of quantum technologies...)

	An enduring incarnation of atomic theory died when the \emph{inidivisibles} were found to divide:
	A distinct discovery, that each element possess distinct \emph{Fraunhofer lines}, or emission of light with specific wavelengths, would later find their resolution in a single insight.
	In truth, the insight was gradual, distributed among many participants, and is still unfolding today. The names we remember - Bethe, Bohr, Born, de Broglie, Dirac, Ehrenfest, Fermi, Feynman, Heisenberg, Lamb, Pauli, Planck, Schrodinger, Sommerfeld - stood among many others who were yet to be giants. The emblematic protagonist, Albert Einstein, is connected to the works herein by many threads: The photoelectric theory established the correspondence between photon energy and wavelength; the theory of special relativity fixed the speed of light as the universal constant connecting the energy and frequency of light through its linear dispersion in free space; the statistics of Bose gases with conserved particle number (\emph{i.e.} quantum-degenerate bosonic matter); the spontaneous and stimulated absorption and emission of radiation by bound charges; and many more indirect routes. \footnote{Quote re: humility from his book}


	-> GR and atoms yet to be connnected... experimental intervention on the horizon with atom interferometers...

	add in the postulate of de broglie - where was this first verified? -
	and the double slit experiment, then the whole world turns just about on
	its head - matter waves

	The `two clouds obscuring the sky of physics' William Thomson, 1st Baron Kelvin
	 Michaelson-Morley expt and inability of Maxwell-Boltzmann to explain specific heats in terms of continuous energy scales
	 -> then Planck's postulate which he made *to fit the data*...
	 -> plus The photoelectric effect => the quantum theory

	Theory of atomic structure; Bohr model and old quantum theory
	Evolution into QFT and standard model
	Following Rydberg's observation of that the wavenumber was proportional to the difference of inverse-squares of whole numbers in 1888, by 1913 Bohr proposed a new model, informed by Rutherford's scattering experiments, that electrons orbited the nucleus like planets.
			Bohr assumed the orbital angular momentum had to be quantized.

	Arguably, spectroscopy is the mother of all our understanding of matter. From spectroscopy was born quantum theory, spin, and the prediction of antimatter in relativistic quantum electrodynamics.


	RSP [145,165] single-body: quantization of [145] black body ratioan [165] energy levels of hydrogen, and attendant revolutions from the wave picture  
	M Planck, The theory of heat radiation Blakiston's son \& co 1914
	E Schrodinger, Quantisierung als eigenwertproblem, annalen der physic 384, 1926
	[11,12] Bell state measurements by Aspect et al ruling out HV theories, loophole-free tests [76,64,172] RSP
	
\subsection*{Quantum cathedrals}
% other titles: QED; a/the (quantum) revolution; The works of giants; Cathedrals of the giants; The 

% QED's experimental success, its foundational role in the SM, and current issues
	Derivation of Rydberg constants from fundamental constants early success [N Bohr, on the constitution of atoms and molecules] 
	'Doublet' features finally explained by dirac's relativistic description, uncovering spin and antimatter [P A M Dirac, the quantum theory of the electron]
	Subsequent observation of the Lamb shift [10,11] required QED [12] by Bethe  NTM
	Astoundingly fast progress within 2yrs [Dyson 13], after which any observable could be expressed as a sum over constituent processes of increasing complexity weighted by alpha; and which posited that the vacuum was not empty but alive with virtual particles which manifested as measurable changes in the energy levels of atoms, including the Lamb shift
	And also, QED predicts anomalous electron magnetic moment to some 9 [20] - accurately [21] NTM
	We are in a regime where QED and experiments seem to be good - can we thus turn the lens back on the constants?	
	Are constants really constant? [62-73] for tests of drift or spatial variation  NTM

	Quantum Electrodynamics, or QED, describes the interaction of charged particles with the electromagnetic field, whose fundamental excitations are identified with the more familiar photons, or particles of light.
	QED therefore describes the physics that governs all we see with our eyes, the interatomic forces from which arise the various familiar phases of matter, prevent solids from passing through one another, almost all technology (even nuclear physicists use electronic control and diagonistic technology), and indeed the dynamics of the action potentials in neurons.
	Hence, the purview of QED may well include the physics underlying the most intriguing of phenomena, perception.
	The detailed connection between quantum field theory and subjective self-awareness are beyond the scope of this thesis.
	For decades, quantum electrodynamics has stood unchallenged as the most accurate quantitative description of the world to date.
	Among its triumps include the prediction of the Rydberg constant to absurd precision and the correct prediction of the existence of antimatter.
	As the first synthesis of special relativity and quantum mechanics, QED laid foundations for more general quantum field theories, ultimately leading us to the standard model of particle physics.
	Undoubtedly, QED is a foundation stone in one of the great pillars of our understanding of the cosmos.
	QED, and QFT in general, presently has no formulation that is consistent with general relativity (other than in string theory, which despite its ambition and elegance has yet to satisfy experimental physicists).
	However, until we have the technology to synthesize black holes or other extreme gravitational conditions, we may not have experimental access to the high energy densities required to probe the Planck scale where quantum and gravitational effects are expected to be of comparable magnitude.
	Fortunately, we may not have to wait so long: The infamous proton radius puzzle, regarding the disagreement between experimental determinations of the proton charge radius, remains unresolved.
	Further, there remain statistically significant disagreements between predicted and measured energy levels in Helium.
	If there is an identifiable bias in theoretical predictions then, optimistically, one may find a legitimate need for physics beyond the standard model to explain these results.
	Therefore, experimental atomic physicists may find themselves prospectors for the fundamental discovery of the century.
	The experiments described in the following two chapters constitute searches for evidence to constrain the search space of theories that purport to resolve the ongoing disagreements.
	Before describing the aims, findings, and methods of the experiments, I will provide a short refresher on atomic theory, terminology, and notation that is relevant to the following results.

	But for all its triumphs, our best physical theory, quantum electrodynamics, falls short in some high-precision instances.
	For example, if you switch the electron in Hydrogen for a Muon and measure the respective Lamb shifts, you can determine the radius of the proton and find that it's different in each case.
	We need more measurements to constrain or discard competing theories.
	Fortunately, the simplicity of Helium allows predictions of its transition lines to some parts per trillion, accurate enough to compete with modern spectroscopy.
	This is parallel to proton radius puzzle... Other gaps in the standard model...
	Despite his role in the eponymous experiment, it's worth bearing in mind Albert Michelson's comment that `the great principles had already been discovered, and that physics would henceforth be limited to finding truths in the sixth decimal place.'\footnote{ at the inauguration of the Ryerson Physics Laboratory at the University of Chicago https://www.bbvaopenmind.com/en/science/physics/lord-kelvin-and-the-end-of-physics-which-he-never-predicted/}.
	Seen another way, the years that followed could be read as evidence that the revolutions are not in the margins, but in the paradoxes.

\subsection*{Collective action}
	The QSM revolution, prediction and early study of BEC... 
		'large' quantum states and quantum resources
		somehow connect to cold gases: The tools of tomorrow being built in the cold-atom labs of today? By analogy, we've gone from hand-cut axes to laser cutters and the attendant precision has exploded the variety of uses for material resources; and so one is led to wonder about the ways these quantum resources will be put to use by means of the new era of ultra-precise, ultra-cold, and ultra-... era.... The state-of-the-art being advanced by groups in XYZ technologies, and their strident peer: cold atoms

	In 1925, extending Bose's work [7], Einstein predicted that so-called bosons collapse into a new state of matter [14] - noting critical temperature is a millionth of the coldest part of the universe TKV
	Since these: Lasing, superfluidity, and superconductivity 
	Lasers first demonstrated in the 50s but not until the 80s was laser cooling done and, 70 years after BE work and a century of QM, quantum degenerate matter was realized in the laboratory
	Superfluidity and such... bogoliubov and other giants of condensed matter

	Kamerlingh Onnes first liquified and started low-temperature physics (Nobel prize, no?)
	Ultracold atoms A brief history Cooling and trapping - who, when, why?
	BEC - a complete surprise and experimental triumph

	The discovery (Kapitza 1938) of superfluidity in helium was immediately thought by London to be connected to this effect now known as BEC.
	Three years later Landau formalized the two-fluid model suggested by Tisza in 1940, and then in 1947 Bogoliubov provided the microscopic theory underlying the Landau model.

	Later, Landau \& Lifshitz (1951), penrose (1951) and Penrose and onsager (1956) introduces ODLRO, of which superfluidity in both bosons and fermions is a consequence (in the one-body and two-body density matrices?)

	Usual story goes that Bose noticed the effect of counting statistics on photons, and Einstein predicted that non-interacting atoms would undergo a phase transition at low temperatures.

	The concept of coherence has enjoyed a long history and several reconceptions.
	The classical definition of coherence was concerned with the phase of an oscillating signal, such as an electromagnetic wave.
	A wave was said to be coherent if one could predict with certainty its phase at a later time, or at a different point on its path of propagation.
	That is, the phase at two points in spacetime were perfectly correlated: Information about the phase at one point yielded complete information about the phase at another.
	After the development of the quantum theory of light, Hanbury-Brown and Twiss used correlations in intensity, the square of the electric field amplitude, to measure the diameter of stars.
	A classical explanation was accepted until Roy Glauber unified the concepts of coherence and correlation in his seminal paper.
	He extended the concept of coherence to arbitrarily high orders by invoking the correlations between additional points in spacetime.
	Light was said to be perfectly coherent to nth order if knowledge of the phase of light at one point yielded knowledge of the phase at n other points? This definition was extended thereafter to include massive bosons, such as atoms.
	Decades later, the modernized definition of coherence was explored by experiments in quantum optics and with ultracold atoms.
	Atoms have been used to reproduce many classic phenomena in optics, including the construction of atomic interferometers, atomic diffraction gratings, and more recently the reproduction of the Hanbury-Brown and Twiss effect.
	

	Resource theories have gained much ground lately, most prominently the resource theories of entanglement and thermodynamics.
	However, contextuality and reference frames have been explored as resources as well.
	Any resource theory must characterize the resource, give a framework for quantifying the resource content of a state, and describe the operations that preserve and diminish the resource content of a given quantum state.
	Resource theories pertain to coherence, entanglement; where do they come from? An incoherent state, with no resource content, is therefore any state with a fully diagonal density matrix in a given basis.
	A coherent state is any state with  nonzero off-diagonal elements, Where does coherence come from? How much do you get in a BEC (distillable) and at what cost?	In which we connect with past measurements of coherent BECs, perhaps across the condensation transition, and test the performance of various coherence monotones across said transition.


	and the still nascent technolgies of single-system state determination.
	These are still developing but have laid the foundations for the third
	quantum revolution, the large-scale engineering of quantum states by
	controllable interactions between multiple subsystems.
	The posterchild
	for such technologies is, of course, quantum computing.
	Notwithstanding
	ongoing controversy over the viability and usefulness of quantum
	computing, the challenge of large-scale quantum engineering has spurred
	an explosion of technical developments.
	Moreover, the growing prominence
	of quantum technology has drawn the curious eyes of computer scientists,
	who now join forces with physicists in attempts to unravel the basic
	structure of the cosmos from process-theoretic perspectives.


	RSP [51] Many-body: Global wavelike interference of interacting objects, as in EPR [51] - entanglement 
	EPR, can quantum-mechanical description of physical reality be considered complete? Phys Rev 47, 1935
	J Eisert et al, quantum many-body systems out of equilibrium, nature physics 11, 2015

\subsection*{Approaching the unapproachable} % cold atom stuff
	\subsubsection{Laying foundations}
	The themes of condensed matter, quantum information, and ATMOP find their synthesis in UCA labs...

	% \subsection*{Creation of Bose-Einstein condensates}
	Doppler and molasses temperatures are called Cold.
	Collisions in the sub-microkelvin regime are below the recoil energy of light and so must occur in the dark - this is the 'ultracold' regime.
	Absolute limits of cooling Thermodynamic limits Third law \& quantum	proof Trap losses Modern methods Cooling fermions Prospects for feedback	cooling? Quantized refrigerators Algorithmic cooling Other techniques:	dilution fridges etc
	% Making, Probing, and understanding BEC
	% Metcalf & Van der Straten
	
	TKV Laser cooling limits are low enough to load mag traps which can be evap cooled
	% H F Hess, Evaporative cooling of magnetically trapped and compressed spin-oolarized hydrogen, Physical Review B 34, 3476-3479, Sept 1986
	% W Ketterle and N J van Druten, Evaporative cooling of traped atoms, advanes in atomic molecular and optical physics 37, 181-236, 1996
	% O J Luiten, M W reynolds, J T M Walraven, Kinetic theory of the evaporative cooling of a trappe dgas, Physical Reveiw A 53, 381-389 Jan 1996


	BEC history
	While the industry of European statistical physics was in its infancy, a	young admirer of Einstein began asking questions would plant the seed of	a flurry of work culminating in a technical triumph over the next seven	decades.
	Satyendra Nath Bose made postulates about distinguishability -	and noticed that such particles had dramatically fewer distinguishable	microstates than they would if they were distinguishable.
	This, of	course, has dramatic consequences for the statistical physics of systems	constituted of these particles, which are now known as Bosons in his	honour.
	Among the predictions that follow from postulating the	indistinguishability of particles is that in the low-temperature limit,	the partition of energy among constituents diverges from the more familiar maxwell-boltzmann distribution.
	The threshold here can be thought of as	the point where the intrinsic wavelike nature of particles becomes	pronounced enough that the wavepackets of neighbouring particles begin	to overlap, heralding a regime where the particle picture breaks down.	Explicitly, Louis de Broglie's postulated that the relationship of	momentum and wavelength of photons, $\lambda_{dB} = h/p$, was exactly	true for all particles.
	The reason we do not see particles interfere at	everyday scales is that the so-called de Broglie wavelegth is smaller	than the particles themselves.
	and in three dimensions one can therefore ascribe a the particles	a'quantum volume' of $\lambda_T^3$.
	For a gas of density $n$, the	volume per particle is $1/n$.
	From this argument, the quantum nature	of gas particles cannot be ignored when $\lambda_T^3 \approx 1/n$,	or when $n\lambda_T^3\approx 1$.
	The latter quantity is called the	phase space density, referring to the concentration of the likely atomic	states into a small region of phase space, consisting of the	position-momentum conjugate variables (footnote: Phase space is not just	x/p but could refer to any set of conjugate variables, mathematically	speaking).

	Macroscopic coherence is also manifest as off-diagonal long range order.. WHAT DOES THIS MEAN -> High-order corrfuns integrate in?
	Comes from coherences in density matrix...? as described by Penrose and Onsager (and leading	to their definition of condensation in terms of the eigenvalues of the	density matrix), and also has close analogies with Glauber's theory of	optical coherence.
	Glauber's theory was extended by Sudarshan (?) to	matter waves, which are distinct from the photonic case by ???.
	The	theory of coherence makes predictions about the arrival-time	correlations and distinguishes the g(2) function of the condensate	(FUNCTION) from the thermal state (FUNCTION).
	These predictions were	borne out by early experiments with metastable Helium, conducted in this	laboratory using the same machine described in this thesis.
	For these	reasons, the BEC is often referred to as a coherent state of matter, and	the resulting pulses of atomic matter waves are called atom lasers in	analogy with the coherent light sources, or lasers.

	This, along with the various analogies between the optical propagator	(the Huygens' equation) and the quantum mechanical one (the Schrodinger	equation?), especially in the advent of techniques for reflection and	dispersion of the momentum of coherent matter waves, led to the	emergence of the term \emph{atom optics}, and heralded a slew of	experiments with matter waves that demonstrated the equivalence of	optical and atomic systems, including matter wave interferometers and	foundational experiments like Wheeler's delayed choice experiment.
	A	distinguishing feature of atoms from light is that the atoms have	intrinsic rest mass, and hence interact with each other gravitationally.	this is the root of ongoing experimental campaigns to harness this	distinguishing feature for applications such as gravimetry, and also to	probe the interface of quantum mechanics and gravity, a central	outstanding problem in modern physics.

	there are some folks who claim steady-state BEC by optical cooling which doens't need the atoms to talk to each other.

	DFG papers; Greiner et al 2003 (40K), Zwirlein et al (2003) bourdel et al (2003) bartenstein et al(2004) and partridge et al (2005) with 6Li.
	Others have followed

	Naturally, the coldest object in the universe will tend to heat up by virtue of being in the universe.
	Unfortunately this means the lifetime of a BEC is always limited.
	Crafty experimentalists have found workarounds...
	Moreover you can never get to 0K

		% NTM [169-172] HOM, Wheeler as classic recapitulation of QM experiments 
		% R lopes et al, aomic HOM experiment, nature 520 2015
		% C K Hong, Z Y Ou, L Mandel, measruement of subpicosecond time intervals between two photons by interference, phys rev lett 59, 1987
		% A G manning et al, single-atom source in the pico-kelvin regime, phys rev lett 113, 2014
		% A G Manning et al, wheelers delayed-choice expt with a single atom, phys rev less 59 1987

		% https://journals.aps.org/rmp/abstract/10.1103/RevModPhys.78.483 
		% photoassoc unlocked by cold atoms

	% RGL [5,6] Cornell, Wiemann, Ketterle nobel 
	% E A Cornell & C E Wiemann, nobel lecture: Bose-Einstein condensation in a dilute gas, the first 70 years and some recent experiments, rev mod phys, 2002
	% SSH  Realization of BEC [1,2,3], previous work on cooling and trapping [4-8], 
		% M H ANderson, J R Ensher, M R Mathews, C E Wieman, and E A cornell, observation of bose-einstein condensation in a dilute atomic vapor, Science 269, 198-201, July 1995
		% K B Davis, M O Mewes, M R Andrews, N J van Druten, D S Durfee, D M Kurn, W Ketterle, Bose-Einstein condensation in a gas of sodium atoms - Physical Review Letters 75, 3969-3973, Nov 1995
		% C C Bradley, C A Sackett, J J Tollett,  G Hulet, evidence of Bose-Einstein condesnation in an atomic gas with attractive interactions, phyiscal review letters 75, 1687-1690, aug 1995
		% W D Phillips, H Metcalf, Laser deceleration of an atomic beam, Phys Rev Lett 48, 1982
		% Chu et al, three-dimensional viscouc confinement and cooling of atoms by resonance radiation pressure, phys rev lett 55, 1985
		% Ch et al, experimental observation of optically trapped atoms, phys rev lett 57, 1986
		% Raab et al, trapping of neutral sodium atoms with radiation pressure, phys rev lett 59, 1987
		% P D Lett et al, observation of atoms laser cooled below the Doppler limit, phys rev lett 61, 1988
	SSH [9] inertial sensors [10] atomic clocks 
	% A D Croning et al, optics and interferometry with atoms and molecules

		NTM [111-120] key papers in cold atoms up to BEC/DFG 
		% B DeMarco, D S Jin, onset of fermi degeneracy in a trapped atmoic gas, science 285 1999
		% A G truscott et al, observation of fermi pressure in a gas of trapped atoms, science 291, 2001
		% F Schreck et al, quasipure Bose-Einstein condensate immersed in a fermi sea, phys rev lett 87 2001

	RGL [1-3] nobel works for Chu, Cohen-Tannoudji and Phillips (remember that time you met him - BEC was a total surprise) 
	Chu, Cohen-Tannoudji, Phillips - Nobel lectures, rev mod phys 70, 1998

	

	BEC in space 2020

	2020 Mordini - measurement of the equation of state - BEC still yielding to fundamental studies

	Still active exploration of realms of applicability of conventional theories and extending the understanding of condensed matter beyond areas amenable to mean-field theories etc; soon enough quantum simulators might do something useful... 

	kastberg95
	lattices briefly held the record for the lowest kinetic temperature record

	Get specific about some of the tools used in cold atoms c.f. collective phenomena just to give some examples/knowledge of other techniques out there... 'methods for the very general study of condensed matter...' tie back to depletion chapter somehow

	\subsubsection{New frontiers}

	Some have said that we live in the `silicon age', in light of the
	pervasiveness of computing technology based on silicon substrates.
	What	Understanding	gained since the conception of QM has led to myriad other technologies	that foundationally depend on the quantum world.
	A second quantum	revolution began with the creation and manipulation of single quantum	states, for which Haroche and Wineland were awarede a Nobel, but	includes technologies such as ion traps and coherent control mechanisms,
	Nuclear magnetic	resonance, for example, underpins the life-altering technology of	magnetic resonance imaging and finds use in the study of biomolecular	structure and industrial applications, forming multibillion dollar	industries.
	A second prominent example is the laser, whose functioning
	depends on the quantum theory of light and matter.
	Of course, lasers outside of	the atomic physics laboratory are now used in fields as diverse	as electromagnetic and gravitational astronomy, medicine, self-driving	vehicles and robotics, cosmetics, manufacturing, remote sensing, and	miltary use.

	yan20 ultracold dipolar molecules with microwave-dressed resonant interaction; enhances interaction above s-wave scattering limit for controlled interactions and immediately applicable in lattices....

	In the early days, cold gases were
	fantastic resources for studying atomic structure and basic interactions
	such as dimerization, because their low kinetic energies and densities
	dramaticall reduced the homogenous broadening effects that spectroscopy
	would otherwise be susceptible to.
	The development of advanced atomic
	clocks was a logical extension.
	Mre about spectroscopy?
	In the later 90s though, Jaksch and
	Zoller proposed quantum simulation in optical lattices.
	Lattices
	themselves had been used before for some studies - check out Orzel for
	the work that \emph{almost} got to the quantum phase transition - but
	yeah.
	That field is exploding now, with advances into microscopy and
	stuff.

	Quantum technology is a young term, but has been growing exponentially	since used in print for the first time in about 1970 according to google	Ngram. 
	Despite their foundational reliance on the quantum picture
	of the world, these technolgies may one day be seen as `primitive' in
	the same way that a typewriter or vacuum tube is now.



		Amico 2020 - roadmap on atomtronics

		% RGL [19,20] record breaking ultracold atom clocks surpassing Cs fountains? 
	% S L Campbell et al, a fermi-degenerate three-dimensional optical lattice clock
	% 	G E Marti et al, imaging optical frequencies with 100 mu Hz precision and 1.1 mu m resolution, phys rev lett 120, 2018
		
\subsection*{Coming into focus} % ultracold helium and motivation for this thesis
	The historical study of Helium is a microcosm of the maturing enterprise of physics....

	remember the line 'exquisitely isolated' - find that Ketterle lecture if you have time

	RGL He is unique in the intersection of tractable structure calculations and access to ultralow temperature - H has reached degeneracy [15] but the experimental sophistication favours traditional atomic beams [16-18] 
	He is unique in the intersection of tractable structure calculations and access to ultralow temperature - H has reached degeneracy [15] but the experimental sophistication favours traditional atomic beams [16-18] RGL

	Early uses of spectra Discovery of Helium? Types of spectroscopy
	Emission \& absorption spectroscopy State of the art methods Landmark
	results Lamb shift proton radius



		Work of other groups:
		NTM [123-130] He* BECs 
		% A robert et al, A bose-einstein condensate of metastable atoms, science 292, 2001
		% F Pereira dos Santos et al, bose einstein condenstation of metastable helium, phys rev lett 86, 2001
		% A S Tychkov et al, metastable helium bose-einstein condensate with a large number of atoms
		% R G Dall & A G truscott 2007
		% S C doret et al, buffer-gas cooled bose-einstein condensate phys rev lett 103, 2009
		% M keller et al, bose-einstein condensate of metastable helium for quantum correlation experiments, phys rev a 2009
		% Q bouton et al, fast production of Bose-Einstein conensates of metastable helium, phys rev a 91, 2015
		% A S flores, simple method for producing Bose-Einstein condensates of metastable helium using a single-beam optical dipole trap, Appl Phys B 121, 2015

				SSH IOP group works [69] -> Spend the time to collect all the mHe references you can find...
		% O. 	Sirjean, S. 	Seidelin, J. 	V. 	Gomes, D. 	Boiron, C. 	I. 	Westbrook, A. 	Aspect,and G. 	V. 	Shlyapnikov, ``Ionization Rates in a Bose-Einstein Condensate of Metastable Helium,'' Phys. 	Rev. 	Lett. 	89, 220 406 (2002). 		% S. 	Seidelin, J. 	V. 	Gomes, R. 	Hoppeler, O. 	Sirjean, D. 	Boiron, A. 	Aspect, and C. 	I. 	Westbrook, ``Getting the Elastic Scattering Length by Observing Inelastic Collisions in Ultracold Metastable Helium Atoms,'' Phys. 	Rev. 	Lett.		93, 090 409 (2004). 		% A. 	Perrin, H. 	Chang, V. 	Krachmalnicoff, M. 	Schellekens, D. 	Boiron, A. 	Aspect,		and C. 	I. 	Westbrook, ``Observation of Atom Pairs in Spontaneous Four-Wave		Mixing of Two Colliding Bose-Einstein Condensates,'' Phys. 	Rev. 	Lett. 	99,		150 405 (2007). 		% V. 	Krachmalnicoff, J.-C. 	Jaskula, M. 	Bonneau, V. 	Leung, G. 	B. 	Partridge,		D. 	Boiron, C. 	I. 	Westbrook, P. 	Deuar, P. 	Zi´n, M. 	Trippenbach, and K. 	V.		Kheruntsyan, ``Spontaneous Four-Wave Mixing of de Broglie Waves: Beyond		Optics,'' Phys. 	Rev. 	Lett. 	104, 150 402 (2010. 		% J.-C. 	Jaskula, M. 	Bonneau, G. 	B. 	Partridge, V. 	Krachmalnicoff, P. 	Deuar,		K. 	V. 	Kheruntsyan, A. 	Aspect, D. 	Boiron, and C. 	I. 	Westbrook, ``Sub-		Poissonian Number Differences in Four-Wave Mixing of Matter Waves,'' Phys.		Rev. 	Lett. 	105, 190 402 (2010). 		% SSH ENS goup dimer photoassociation [74] and scat len [75]. 		% J. 	L´eonard, M. 	Walhout, A. 	P. 	Mosk, T. 	M¨uller, M. 	Leduc, and C. 	Cohen-		Tannoudji, ``Giant Helium Dimers Produced by Photoassociation of Ultracold		Metastable Atoms,'' Phys. 	Rev. 	Lett. 	91, 073 203 (2003). 		% S. 	Moal, M. 	Portier, J. 	Kim, J. 	Dugu´e, U. 	D. 	Rapol, M. 	Leduc, and C. 	Cohen-		Tannoudji, ``Accurate Determination of the Scattering Length of Metastable		Helium Atoms Using Dark Resonances between Atoms and Exotic Molecules,''		Phys. 	Rev. 	Lett. 	96, 023 203 (2006).
		
		NTM [163-168] atom correlations with He* 
		% T Jeltes et al, comparison of the HBT effect for bosons and fermions, nature 445, 2007
		% M SChellekens et al, HBT effect for ultracold quantum gases, science 310, 2005
		% S S Hodgman et al, direct measrurement of long-range third-order coherence in BEC, science 331, 2011
		% R G Dall et al, ideal n-body correlations with massive particles, nat phys 9, 2013
		% W RuGway et al, correlations in amplified four-wave mixing of matter waves, phys rev lett 107, 2011
		% R Lopes et al, second-order coherence of superradiance from a bose-einstein condensate, phys rev A 90, 2014

		NTM [173-176] towards entangled pairs 
		% 		A perron et al, observation of atom pairs in spontaneous four-wave mixing of two colliding bose einstein condensates, phys rev lett 99, 2007
		% V Krachmalnicoff et al, spontaneous four-wave mixing of de broglie waves: beyond optics, pys rev lett 104, 2010
		% J C Jaskula et al, sub-Poissonian number differences in four-wave mixing of matter waves, phys rev lett 105, 2010
		% K V kherunstyan et al, violation of the Cauchy-Schwartz in equality with matter waves
		% Shin PhD work
			
		Spectroscopy with He - point to RGL/NTM? for historical review until 2017ish
			RGL [21,22] forbidden measurements  
				% R an Rooij et al, frequency metrology in quantum degenerate helium: direct measurement of the 2 3S1 - 2 1S0 transition, Science 333, 2011
				% R P M J W Notermans et al, high precision spectroscopy of the forbidden 2 3S1 2 1P1 transition in quantum degenerate metastable helium, phys rev lett 112, 2014
		NTM [131] 3He* DFG, as direction for french group 
		% J M McNamrara, degenerate bose-fermi mixture of metastable atoms, phys rev lett 97, 2012
		Recent optical lattice work


\section*{Pr\'{e}cis}\label{sec:abstract}
\addcontentsline{toc}{section}{Pr\'{e}cis}  



	% Three approaches towards measurnig nothing... 
	This thesis documents three projects that were undertaken in the ANU
	Helium BEC laboratory over the period of 2018-2019 and one project,
	which remains unfinished, that was undertaken in 2016-2018.
		A number of
	other graduate students were in residence at the time, and each of them
	contributed variously to the experiments.
		These experiments constitute a
	chapter each, and conclude with acknowlegements of the contributions of
	each person involved.
		The chapters are arranged in such a way as to
	provide a progression through subjects of increasing complexity, and are
	separated into three parts.

	The first part includes a general introduction to ultracold atomic
	physics and an overview of the apparatus used to complete the
	experiments described in this thesis.

	The second part includes chapters 3 and 4, and concentrates on a pair of
	experiments regarding the atomic structure of Helium.
		The experiments in
	this section are motivated by open questions in atomic structure theory,
	and includes a review of essential concepts.
		Chapter 3 concerns a set of
	measurements of electronic absorption lines, which in some sense is the
	most elementary concept encountered in this thesis.
		Chapter 4 describes
	the measurement of a tune-out frequency in Helium, at which there is a
	null response of the atomic dipole induced by an oscillating electric
	field.
		Such tune-out frequencies are determined by the interplay of an
	array of atomic transitions, and so this chapter represents a marginal
	increase in complexity.
		Conversely, the signals sought in chapters 3 and
	4 are presented in sequence of decreasing signal power, ultimately
	converging on a measurement of a null response, attempting to determine
	when \emph{nothing} happens.

	Following these studies of the internal structure of atoms, the third
	part, consisting of chapters 5 and 6, is concerned with interacting
	systems.
		Chapter 5 describes a third completed project concerning the
	effects of weak interactions on the density and correlations of the
	momentum spectrum of ultracold Bose gases.
		This section includes an
	overview of the relevant physics of ultracold gases.
		Chapter 6 discusses
	the motivation for, and progress towards, an optical lattice trap for
	ultracold Helium.
		This was the initial project of my PhD, but after two
	years of work the decision was made to discontinue working to construct
	this apparatus.
		Chapter 7 presents a summary of the findings of all the
	projects in this thesis, presents future directions of research using
	ultracold helium, and concludes by unraveling the narrative thread of
	this thesis.
	the th





