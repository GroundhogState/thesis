% \hypertarget{conclusion}{%
% \section{Conclusion}\label{conclusion}}

\chapter{Conclusion}
\markboth{CONCLUSION}{}

\begin{flushright}
\emph{``Good judgment depends mostly on experience, and\\ experience usually comes from poor judgment"} \\- Anonymous\footnote{The origin of this frequently misattributed aphorism is unknown \url{https://quoteinvestigator.com/2017/02/23/judgment/}}
\end{flushright}

% \todo{Actually write a conclusion}

% \section{Synthesis}\label{sec:synthesis}
% \section{Outlook}\label{sec:outlook}
% \section{Afterword}\label{sec:afterword}
% \section*{Gratitude}\label{sec:gratitude}
% \addcontentsline{toc}{section}{Gratitude}

% % Pick a good-looking drop cap https://www.fontsplace.com/free/images/t/typographer-caps_font_preview_characters.gif
	 
	 % https://jila.colorado.edu/~junye/yelabsOLD/pubs/scienceArticles/2007/sArticle_2007_10_Helium_DFCS_EPJD.pdf
	
	The versatile trapping capabilities of optical lattices [Bloch review and more recent] have opened doors into a new era of precision spectroscopy and quantum simulation of condensed matter systems (see chapter Y for further discussion). [https://journals.aps.org/rmp/abstract/10.1103/RevModPhys.83.1523]
	Put in a box with single atom detection
		get the Ott review
		
	In parallel, the miniaturization of magnetic and RF traps down to chip-scale circuits is an important step towards developing and deploying `atomtronic' technologies employing the techniques under development in cold atom labs today [roadmap on atomtronics].
	More exotically, condensates of bound matter-antimatter atoms, positronium, are subjects of study for their potential as gamma-ray laser gain medium [https://journals.aps.org/pra/abstract/10.1103/PhysRevA.89.043624]
	and BEC has recenlty been achieved in constant freefall on board the international space station [] towards long-baseline atom-interferometric gravimetry


This thesis describes work that consists of the following scientific
contributions: * Martin was wrong * New lines * Weakest transition *
Test of QED * Partial resolution of QD which didn't answer much and
isn't decisive re: theory * work towards a new regime of condensed
matter simulation

It is instructive to reflect on the efficiency of these machines as generators of quantum matter.
	The mass input is X to produce Y condensed atoms and detect only 10\% of them.
	The COP is complicated by the atom loss along the way but if we consider the change in energy of only the charge of atoms distilled to the BEC, then we find a COP of X.
	The duty cycle of about 30mHz means that even with large BECs the average flux rate is only X.
	These are low, but one would do well to recall that the first steam engines had efficiencies below 1\%, and several engineering breakthroughs later brought them towards several tens of percent, and gasoline engines followed a similar trend.
	In the production of BEC in the new machine, the duty cycle is reduced by a factor of ten and the atom number by a factor of two, an increase in the time-coherence product of a factor of twenty.
	The downscaling of helium condensate machines is a nontrivial challenge, but one is brought to wonder about the prospects of future fast-cycling, compact devices routinely creating and manipulating large quantum states.
	As James Watt could not conceive of the full implications of a developed rail network, we too are surely blind to the world we may be building


Lord Kelvin's famous remark - 'aside from two clouds, physics is done' - his arrogance aside, we are in a different position: Rather than a 'nearly done' job with a few details to work out, we are vividly aware of the sites of our ignorance.
	As Einstein (?) said, a well-formulated question contains its own answer; but we are still some ways from determining whether we are even asking the right questions!

\begin{itemize}
% \tightlist
\item
  future specroscopy - prospects for helium 3
\item
  Lattices - possible directions
\item
  Quantum correlations \& foundations experiments
\end{itemize}

From Howl (2021) refs 60,61 on Feshbach resonances for fundamental physics

A personal reflection.
	Was lured in to the academy by the promise of
questions like `what is the difference between simplicity and
complexity' - when are things simple? When is something intrinsically
hard to represent, even if it exists effortlessly in its own right? This
drew me to lattice - to build something with the hands that would be so
elusive for theory and computing.
	But yeah - that might have been
possible, but I hadn't the background to make reasonable progress in the
three year timeframe of an Australian PhD.
	Also, frankly, I didn't have
the tuition or support and so was left alone for most of the time,
crushing motivation and without praise and squirreled my time away in a
simulation that was much more to my theoretical tastes.
	But hey.
	I have
to find a nice way to dress this up.
	Most of these words won't make it.
So having got these ones out of the way, the hope is that I can move on
from them and think more deliberately about what it is that I would say
here.

The chronological order of work is reversed.
	I should perhaps detail the
actual sequence of events - and frame them as pragmatic decisions from
which I learned some useful things about scoping projects properly.
There isn't really much that window dressing can add here.
	But that's
okay - rather than telling the entire truth, just tell enough of it.
	But
don't let the complexity frighten you out of doing your best to be
honest, at least in this tiny sliver, the final vestiage of the
self-expression that you'd hoped you'd find the opportunity to explore.
This has been an exercise in shedding the ego, however incomplete that
process may yet remain.

\#\#\#A critical reading of science Technology evidently brings us great
things.
	But let us not lose sight of the implicit `right to know' - that
we still embody the notion of a conquest over nature, of mastery of the
other, the greater, the ultimately incomprehensible.
	Let us not fall
victim to our own arrogances, and recall that in our turbulent times,
our investigations come at cost.
	The nature of reality will, according
to our ultimate foundations, remain.
	But the conditions of society in
which we have the ability to pursue them are not guaranteed.
	One can
never predict the outcomes of discoveries, or what miraculous things may
be born of new technologies, but in the problem of allocating compute
power from human wetware - which still retains a certain je ne sais
quois that has not been replicated by industrial-scale algorithms
running in silico - we should be mindful of the hubris of endless
pursuit of the mastery of nature.
	This remains ingrained in our
mythology - that more advanced technology is always worth the price that
is paid.
	In the limit of free, clean energy and post-scarcity
manufacture whereby human labour is eliminated, this may be true.
	But,
unlike the atoms in this thesis, we do not live in vacuum.

\subsubsection{New frontiers}

	BEC for analagoe gravity/early cosmological simulation
		https://journals.aps.org/prd/abstract/10.1103/PhysRevD.78.084013

	there are some folks who claim steady-state BEC by optical cooling which doens't need the atoms to talk to each other.
	The themes of condensed matter, quantum information, and ATMOP find their synthesis in UCA labs...
	Still active exploration of realms of applicability of conventional theories and extending the understanding of condensed matter beyond areas amenable to mean-field theories etc; soon enough quantum simulators might do something useful... 
	

	In the early days, cold gases were fantastic resources for studying atomic structure and basic interactions such as dimerization, because their low kinetic energies and densities dramaticall reduced the homogenous broadening effects that spectroscopy would otherwise be susceptible to. The development of advanced atomic clocks was a logical extension. Mre about spectroscopy? In the later 90s though, Jaksch and Zoller proposed quantum simulation in optical lattices. Lattices themselves had been used before for some studies - check out Orzel for the work that \emph{almost} got to the quantum phase transition - but yeah. That field is exploding now, with advances into microscopy and stuff. 	yan20 ultracold dipolar molecules with microwave-dressed resonant interaction; enhances interaction above s-wave scattering limit for controlled interactions and immediately applicable in lattices....

	BEC in space 2020; comparisons with earth-bound experiments to look for GR stuff...

	
	Quantum technology is a young term, but has been growing exponentially	since used in print for the first time in about 1970 according to google	Ngram. 
	Despite their foundational reliance on the quantum picture
	of the world, these technolgies may one day be seen as `primitive' in
	the same way that a typewriter or vacuum tube is now.

	Amico 2020 - roadmap on atomtronics

\footnote{mathematical models - could be thought of as the quantitative storytelling part of humanity, the things that make it unique - and so there's something really sublime about the exactness of the correspondence with the world}, written extensively about in things like `the unreasonable effectiveness of mathematics' and this correspondence is still explored in the problems of foundations of QM: what exactly does this mathematics say abotu the nature of the world? Why does it run this way? This deep inquiry will always be part of the human spirit, I think, but yeah.
The frontiers of physics now marry the fundamental concepts of matter, energy, and information, and we now find much value in the comptuational perspective even at the quantum level...

Aside; gravitation still holds many mysteries and 
	However, until we have the technology to synthesize black holes or other extreme gravitational conditions, we may not have experimental access to the high energy densities required to probe the Planck scale where quantum and gravitational effects are expected to be of comparable magnitude.
	QED, and QFT in general, presently has no formulation that is consistent with general relativity (other than in string theory, which despite its ambition and elegance has yet to satisfy experimental physicists).
	We can never be sure that a given line of inquiry will crack open the mystery; c.f. Michelson, the discovery was not in the sixth decimal place but by asking the right question
	Despite his role in the eponymous experiment, it's worth bearing in mind Albert Michelson's comment that `the great principles had already been discovered, and that physics would henceforth be limited to finding truths in the sixth decimal place.'\footnote{ at the inauguration of the Ryerson Physics Laboratory at the University of Chicago https://www.bbvaopenmind.com/en/science/physics/lord-kelvin-and-the-end-of-physics-which-he-never-predicted/}.
	Seen another way, the years that followed could be read as evidence that the revolutions are not in the margins, but in the paradoxes.

Some have said that we live in the `silicon age', in light of the
	pervasiveness of computing technology based on silicon substrates.
	What	Understanding	gained since the conception of QM has led to myriad other technologies	that foundationally depend on the quantum world.
	A second quantum	revolution began with the creation and manipulation of single quantum	states, for which Haroche and Wineland were awarede a Nobel, but	includes technologies such as ion traps and coherent control mechanisms,
	Nuclear magnetic	resonance, for example, underpins the life-altering technology of	magnetic resonance imaging and finds use in the study of biomolecular	structure and industrial applications, forming multibillion dollar	industries.
	A second prominent example is the laser, whose functioning
	depends on the quantum theory of light and matter.
	Of course, lasers outside of	the atomic physics laboratory are now used in fields as diverse	as electromagnetic and gravitational astronomy, medicine, self-driving	vehicles and robotics, cosmetics, manufacturing, remote sensing, and	miltary use.


	
	'large' quantum states and quantum resources
	Resource theories have gained much ground lately, most prominently the resource theories of entanglement and thermodynamics.
	Any resource theory must characterize the resource, give a framework for quantifying the resource content of a state, and describe the operations that preserve and diminish the resource content of a given quantum state.
	Resource theories pertain to coherence, entanglement; where do they come from? An incoherent state, with no resource content, is therefore any state with a fully diagonal density matrix in a given basis.
	A coherent state is any state with  nonzero off-diagonal elements, Where does coherence come from? How much do you get in a BEC (distillable) and at what cost?	In which we connect with past measurements of coherent BECs, perhaps across the condensation transition, and test the performance of various coherence monotones across said transition.
	[11,12] Bell state measurements by Aspect et al ruling out HV theories, loophole-free tests [76,64,172] RSP
	This led to new understandings of the ways to extract, harness, and apply energy in greater and greater quantities, propelling the process of industrialization and shaping the world of today.
	and the still nascent technolgies of single-system state determination.
	These are still developing but have laid the foundations for the third
	quantum revolution, the large-scale engineering of quantum states by
	controllable interactions between multiple subsystems.
	The posterchild
	for such technologies is, of course, quantum computing.
	Notwithstanding
	ongoing controversy over the viability and usefulness of quantum
	computing, the challenge of large-scale quantum engineering has spurred
	an explosion of technical developments.
	Moreover, the growing prominence
	of quantum technology has drawn the curious eyes of computer scientists,
	who now join forces with physicists in attempts to unravel the basic
	structure of the cosmos from process-theoretic perspectives.
	somehow connect to cold gases: 

The tools of tomorrow being built in the cold-atom labs of today? By analogy, we've gone from hand-cut axes to laser cutters and the attendant precision has exploded the variety of uses for material resources; and so one is led to wonder about the ways these quantum resources will be put to use by means of the new era of ultra-precise, ultra-cold, and ultra-... era.... The state-of-the-art being advanced by groups in XYZ technologies, and their strident peer: cold atoms
	
	The children of 2050 may grow up in a world shaped by an analogous explosion of comprehension and deployment of \emph{quantum} resources, of which the familiar `classical' states of systems in general are nearly devoid to us by virtue of being highly entangled with one another and a shared environment, and are sustaining quantum states/distilling and expending quantum resources entanglement/coherence/etc; (while quantum computing may be the poster child for the dawn of quantum technologies...)
	BEC - a complete surprise and experimental triumph

	In truth, the insight was gradual, distributed among many participants, and is still unfolding today. The names we remember - Bethe, Bohr, Born, de Broglie, Dirac, Ehrenfest, Fermi, Feynman, Heisenberg, Lamb, Michelson, Morley, Pauli, Planck, Schrodinger, Sommerfeld - stood among many others who were yet to be giants. The emblematic protagonist, Albert Einstein, is connected to the works herein by many threads: The photoelectric theory established the correspondence between photon energy and wavelength; the theory of special relativity fixed the speed of light as the universal constant connecting the energy and frequency of light through its linear dispersion in free space; the statistics of Bose gases with conserved particle number (\emph{i.e.} quantum-degenerate bosonic matter); the spontaneous and stimulated absorption and emission of radiation by bound charges; and many more indirect routes. \footnote{Quote re: humility from his book}.

	Einstein's special theory of relativity deals with the differences in observations made from inertial reference frames, falling freely along geodesic trajectories through spacetime. The full theory of general relativity covers a much larger territory, and famously resists unification with quantum theory. 

	-> GR and atoms yet to be connnected... experimental intervention on the horizon with atom interferometers...

	The development of atom interferometers for quantum experiments with gravitational connection is one way forward. The cooling of larger masses may prove more fruitful for foundational experiments

Why is it,
	for example, so difficult to efficiently simulate quantum processes?
	Where is the border between efficient and intractable? The proof of
	genuine quantum advantages in certain processes may be one of the most
	profound statements about the nature of reality of this generation.
	Wherefore the nature of this advantage? Perhaps we will know before the
	century is out - perhaps, if ongoing crises arenot addressed - we will
	never know, and the cosmos may miss its chance to delve most deeply into
	its own self-awareness.... Particularly consider the confluence of embedded AI etc etc...


			On new physics: We are searching for a revolution in the margins.
	A hair's width between complex theory and calculations.
	Let's assume the calculations are perfect and the experimental errors are overly cautious.
	What do we find - another term in the standard model? Very well.
	It's true that the Higgs detection was a spectacular vindication.
	Was it a revolution? Did it change how we think of our place in the world, alter what we thought was possible? Maybe if you are a theoretical particle physicist.
	And so with these results; they may contribute to an advance in physics, defined as a change in our model of the world in a more profound way than better precision in the fundamental constants.
	Forgive my skepticism; the 'new physics' we find in these experiments may extend the standard model, but for the 'pure' scientist, the goal is not to extend SM, but to supercede it.
	Of course, if we knew where the revelations were, we would spend our money very differently, but in some sense *search* is inseparable from re*search*, with apologies to etymologists.


Whereas millenia have passed without significant evolution, each generation living the same lives as the last, we are surrounded/circumscribed/immersed in the pervasive myth of progress.
	The undescoring narrative always invites the conscientious scientist to ask the impossible question; \emph{what} is the future we are building? \emph{For whom} are we building it, and \emph{why}? <- perhaps better placed in the afterword, gels with notion we are not in vacuum
	

Let us put ourselves in Pythagoras' shoes. What was it like to have such a rudimentary understanding of mathematics? To be so far away from the interplay of geometry and algebra that we have now? These proofs must have seemed magic - what are we scratching at in these labs of the modern quantum endeavour? Pythagoras was practically deified for discovering what is now taught to ten year olds. Where are we going? What is going to happen next? What are we becoming?

% % Schrodinger quote goes here?



\vfill
\begin{flushright}
\emph{``We shall not cease from exploration\\
And the end of all our exploring \\
Will be to arrive where we started \\
And know the place for the first time."\\} 
- T. S.	Eliot
\end{flushright}
\vfill

% Notes on some other dissertation conclusions
	


% % 	Davis, Dynamics of BEC

% % 	Greiner, ultracold gases in 3D lattices 375 wds
% 	2 page outlook
% 	Open questions answerable with the present system
% 	POssible extensions (eg fermionic lattice, photoassociation, non-BH hamiltonians, vortices, spin-dependent lattices)

% % 	Schneider, interacting fermions 2 pages
% 	One para main topic
% 	One para per three experiments
% 	One para possible issues and improvements
% 	Outlook
% 		A few directions for future research and what the lab is up to

	

% % 	Bakr, Microscopic QPTs 5 pages double spaced
% 	No conclusion, each chapter self contained
% 	Outlook chapter instead
% 	Roughly one para per suggested direction or improvement


% % 	Hodgman, lifetime & correlation measurements 506 wds
% 	Summary 2 pages
% 		One para summarizing transition works
% 		One para detailing new PAL method
% 		One para on distinguishing thermal and coherent clouds
% 		Couple paragraphs extended summary of a particular experiment
% 	Future work
% 		2 pages inc big figure
% 		Identify a couple transitions whose rates could be measured
% 		Other applications of correlation funs
% 		A couple big-ticket items (EPR for massive particles)

% % 	Struck, Artificial gauge fields 
% 	Each chapter has own conclusion and outlook


% % 	Aidelsburger, artificial gauge fields 
% 	1.5 pages conc
% 		The main topic was...
% 		Summary of the subject and findings of each major chapter

% 	1.5 outlook
% 		Issues and improvements on experiments
% 		A couple incremental steps forward

% % 	Izaac, continuous-time quantum walks 
% 	Most chapters have their own conclusion
% 	3.5 page conclusion
% 	One (big) paragraph of detailed summary of each chapter
% 	Unanswered questions, possible applications of the theory, and a speculative forward-looking paragraph
% 		'we are yet to scratch the surface of what quantum walks are capable of'


% % 	Lukin, entanglement in 1D 341 wds
% 	Super short thesis. 2 page conclusion
% 	In this thesis (summary of major results)
% 	one para summary of major works: Technique, finding, implication
% 	A couple of possible extensions of the apparatus

% % 	Phelps, dipolar QGM 
% 	6.5 pages double space
% 	OUtstanding challenges 
% 	Three sections, 3-5 para each, with potential future directions. 


% % 	Shin, Nonlocal correlations
% 	3 page conclusion
% 	One para summary of thesis contents
% 	3 sections, 3-4 paragraphs on possible research topics
