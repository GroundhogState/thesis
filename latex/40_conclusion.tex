% \hypertarget{conclusion}{%
% \section{Conclusion}\label{conclusion}}

\chapter{Conclusion}
\markboth{CONCLUSION}{}

\begin{flushright}
\emph{``Good judgment depends mostly on experience, and\\ experience usually comes from poor judgment"} \\- Anonymous\footnote{The origin of this frequently misattributed aphorism is unknown \url{https://quoteinvestigator.com/2017/02/23/judgment/}}
\end{flushright}

\todo{Actually write a conclusion i.e. summary and outlook for the various themes and the field in general.}

\section{Synthesis}\label{sec:synthesis}
\section{Outlook}\label{sec:outlook}
\section{Afterword}\label{sec:afterword}
\section*{Gratitude}\label{sec:gratitude}
\addcontentsline{toc}{section}{Gratitude}

This thesis describes work that consists of the following scientific
contributions: * Martin was wrong * New lines * Weakest transition *
Test of QED * Partial resolution of QD which didn't answer much and
isn't decisive re: theory * work towards a new regime of condensed
matter simulation

It is instructive to reflect on the efficiency of these machines as generators of quantum matter.
	The mass input is X to produce Y condensed atoms and detect only 10\% of them.
	The COP is complicated by the atom loss along the way but if we consider the change in energy of only the charge of atoms distilled to the BEC, then we find a COP of X.
	The duty cycle of about 30mHz means that even with large BECs the average flux rate is only X.
	These are low, but one would do well to recall that the first steam engines had efficiencies below 1\%, and several engineering breakthroughs later brought them towards several tens of percent, and gasoline engines followed a similar trend.
	In the production of BEC in the new machine, the duty cycle is reduced by a factor of ten and the atom number by a factor of two, an increase in the time-coherence product of a factor of twenty.
	The downscaling of helium condensate machines is a nontrivial challenge, but one is brought to wonder about the prospects of future fast-cycling, compact devices routinely creating and manipulating large quantum states.
	As James Watt could not conceive of the full implications of a developed rail network, we too are surely blind to the world we may be building


Lord Kelvin's famous remark - 'aside from two clouds, physics is done' - his arrogance aside, we are in a different position: Rather than a 'nearly done' job with a few details to work out, we are vividly aware of the sites of our ignorance.
	As Einstein (?) said, a well-formulated question contains its own answer; but we are still some ways from determining whether we are even asking the right questions!

\begin{itemize}
% \tightlist
\item
  future specroscopy - prospects for helium 3
\item
  Lattices - possible directions
\item
  Quantum correlations \& foundations experiments
\end{itemize}


A personal reflection.
	Was lured in to the academy by the promise of
questions like `what is the difference between simplicity and
complexity' - when are things simple? When is something intrinsically
hard to represent, even if it exists effortlessly in its own right? This
drew me to lattice - to build something with the hands that would be so
elusive for theory and computing.
	But yeah - that might have been
possible, but I hadn't the background to make reasonable progress in the
three year timeframe of an Australian PhD.
	Also, frankly, I didn't have
the tuition or support and so was left alone for most of the time,
crushing motivation and without praise and squirreled my time away in a
simulation that was much more to my theoretical tastes.
	But hey.
	I have
to find a nice way to dress this up.
	Most of these words won't make it.
So having got these ones out of the way, the hope is that I can move on
from them and think more deliberately about what it is that I would say
here.

The chronological order of work is reversed.
	I should perhaps detail the
actual sequence of events - and frame them as pragmatic decisions from
which I learned some useful things about scoping projects properly.
There isn't really much that window dressing can add here.
	But that's
okay - rather than telling the entire truth, just tell enough of it.
	But
don't let the complexity frighten you out of doing your best to be
honest, at least in this tiny sliver, the final vestiage of the
self-expression that you'd hoped you'd find the opportunity to explore.
This has been an exercise in shedding the ego, however incomplete that
process may yet remain.

\#\#\#A critical reading of science Technology evidently brings us great
things.
	But let us not lose sight of the implicit `right to know' - that
we still embody the notion of a conquest over nature, of mastery of the
other, the greater, the ultimately incomprehensible.
	Let us not fall
victim to our own arrogances, and recall that in our turbulent times,
our investigations come at cost.
	The nature of reality will, according
to our ultimate foundations, remain.
	But the conditions of society in
which we have the ability to pursue them are not guaranteed.
	One can
never predict the outcomes of discoveries, or what miraculous things may
be born of new technologies, but in the problem of allocating compute
power from human wetware - which still retains a certain je ne sais
quois that has not been replicated by industrial-scale algorithms
running in silico - we should be mindful of the hubris of endless
pursuit of the mastery of nature.
	This remains ingrained in our
mythology - that more advanced technology is always worth the price that
is paid.
	In the limit of free, clean energy and post-scarcity
manufacture whereby human labour is eliminated, this may be true.
	But,
unlike the atoms in this thesis, we do not live in vacuum.


\footnote{mathematical models - could be thought of as the quantitative storytelling part of humanity, the things that make it unique - and so there's something really sublime about the exactness of the correspondence with the world}, written extensively about in things like `the unreasonable effectiveness of mathematics' and this correspondence is still explored in the problems of foundations of QM: what exactly does this mathematics say abotu the nature of the world? Why does it run this way? This deep inquiry will always be part of the human spirit, I think, but yeah.
The frontiers of physics now marry the fundamental concepts of matter, energy, and information, and we now find much value in the comptuational perspective even at the quantum level...

Why is it,
	for example, so difficult to efficiently simulate quantum processes?
	Where is the border between efficient and intractable? The proof of
	genuine quantum advantages in certain processes may be one of the most
	profound statements about the nature of reality of this generation.
	Wherefore the nature of this advantage? Perhaps we will know before the
	century is out - perhaps, if ongoing crises arenot addressed - we will
	never know, and the cosmos may miss its chance to delve most deeply into
	its own self-awareness.... Particularly consider the confluence of embedded AI etc etc...


			On new physics: We are searching for a revolution in the margins.
	A hair's width between complex theory and calculations.
	Let's assume the calculations are perfect and the experimental errors are overly cautious.
	What do we find - another term in the standard model? Very well.
	It's true that the Higgs detection was a spectacular vindication.
	Was it a revolution? Did it change how we think of our place in the world, alter what we thought was possible? Maybe if you are a theoretical particle physicist.
	And so with these results; they may contribute to an advance in physics, defined as a change in our model of the world in a more profound way than better precision in the fundamental constants.
	Forgive my skepticism; the 'new physics' we find in these experiments may extend the standard model, but for the 'pure' scientist, the goal is not to extend SM, but to supercede it.
	Of course, if we knew where the revelations were, we would spend our money very differently, but in some sense *search* is inseparable from re*search*, with apologies to etymologists.


Whereas millenia have passed without significant evolution, each generation living the same lives as the last, we are surrounded/circumscribed/immersed in the pervasive myth of progress.
	The undescoring narrative always invites the conscientious scientist to ask the impossible question; \emph{what} is the future we are building? \emph{For whom} are we building it, and \emph{why}? <- perhaps better placed in the afterword, gels with notion we are not in vacuum
	

Let us put ourselves in Pythagoras' shoes. What was it like to have such a rudimentary understanding of mathematics? To be so far away from the interplay of geometry and algebra that we have now? These proofs must have seemed magic - what are we scratching at in these labs of the modern quantum endeavour? Pythagoras was practically deified for discovering what is now taught to ten year olds. Where are we going? What is going to happen next? What are we becoming?

% Schrodinger quote goes here?

\begin{center}
\emph{``The key to discoveries is to look at those\\
 places where there is still a paradox.\\
It’s like the tip of an iceberg.\\
If there is a point of \\
dissatisfaction, \\
take a closer \\
look at it.'' \\}
 \end{center}
\begin{center}
~E. Arikan
 \end{center}

\newpage
\vfill
\begin{centering}
\emph{``We shall not cease from exploration\\
And the end of all our exploring \\
Will be to arrive where we started \\
And know the place for the first time."\\} 
- T. S.	Eliot
\end{centering}
\vfill