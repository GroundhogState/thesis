To-do list:\newline
In \verb|{10_overview.tex}|:
\begin{itemize}
\item {What would you say the contribution of the QD work was? I've commented out my suggestion here.}
\item {Get most recent citations}
\end{itemize}
In \verb|{12_apparatus.tex}|:
\begin{itemize}
\item {Chapter with Andrew}
\item {These are wrong, will fix}
\item {Is this necessary?}
\end{itemize}
In \verb|{13_lattice_build.tex}|:
\begin{itemize}
\item {Include \href{https://www.nature.com/articles/s41586-021-03582-4}
\item {Get coil specs}
\end{itemize}
In \verb|{20_metrology.tex}|:
\begin{itemize}
\item {How state-of-the-art theory works, the contributions from experiments. Add citations. Look over the ol Pachucki papers etc for primary refs. }
\item {Some context about the history of precision spectroscopy - lift from later chapter? Put tuneout section in context, at least with a couple sentences, but approach distinctly later}
\item {orientation dependence goes somewhere in here (eg near the $5\triplet D$ section?)}
\end{itemize}
In \verb|{22_tuneout.tex}|:
\begin{itemize}
\item {QED and the proton radius}
\item {Rewrite}
\item {Classical oscillator intuition. Dipole force recap and extension. Why the real part? Go all the way to defining $\fto(-1,0)$ but save the ellipsometry for later. Perhaps: Materials analogy, susceptibility/polz/etc}
\item {Comment on difference between magic and tune-out lattices for qsim/control}
\item {Read these past measurements}
\item {Make first paras a summary of method, incl. identifying f(-1,0), and then outline the rest of the section as detailed discussions of the key points}
\item {Relationship to magic wavelengths, etc etc}
\item {Compress, be clearer it's a summary, acknowledge theorists. Prioritize reformulation as zero in rayleigh scattering cross section.}
\item {Ensure one has the $2\pi$s in the right places; I think I use rad hz here}
\item {Longer caption text. Integrate with this section.}
\item {is there a factor of 2 missing here?}
\item {VALUE}
\item {Fix table}
\item {graphic showing key concepts in polarization part}
\item {start here}
\item {point out densities and coherences}
\item {where we are going:}
\item {Trap freq allan deviation?}
\item {provide a couple}
\item {update Caption, }
\item {Put polarization theory here and then explain the various stages of analysis.}
\item {What was the intensity used in the experiment?}
\item {Integral expression}
\item {Not quite sure what this para is saying specifically. Don't rephrase, rewrite.}
\item {Rework after analysis}
\item {Rework or omit with reference}
\item {Substantially rewrite this section in own voice in own framework. Finally put the math togther as you worked on before. Separate the polarization theory from the polarizability theory. }
\end{itemize}
In \verb|{31_depletion.tex}|:
\begin{itemize}
\item {How would you observe this?  What would it look like?}
\item {cite more experiments}
\item {Density plots out to larger $|k|$ to show descent into noise floor? How far could we push it until we hit dark counts?}
\item {txy-to-vel vs linear error?}
\item {CHECK THIS 3.}
\item {Clarify \& graph of dark count statistics}
\item {Worst-case bias of -1 counts}
\end{itemize}
In \verb|{40_conclusion.tex}|:
\begin{itemize}
\item {Actually write a conclusion}
\end{itemize}
