



% %  If using typographer caps, consider starting with an O or an S as these are visually quit striking - the H and R are also quite good


% % Preamble: A short list of key references/reviews for the student

\todo{As a general comment, focus a bit more on what you actually did and include details that are important to your thesis rather than general information or speculation.  Quite often you will talk about some concept generally, which is fine to start with as long as it is concise, but then you won't include any of the specifics for our experiment, so the reader doesn't know what the actual parameters we use actually are!}

\chapter{Theoretical background}

"No one's mouth is big enough to utter the whole thing" - Alan Watts

	{A} cold atom experimentalist draws on a panoply of tools (both conceptual and instrumental) which are all intricate and absorbing in their own right. This chaper presents the essential ideas needed to give form and context to the content of the major works reported in this thesis\footnote{A neophyte physicist may feel a compulsion to delve even deeper, under the weight of one's curiosity. It bears reflecting on the caveat that knowledge is not power but potential energy: One has to do real work to realize that potential.}. 
	We will glance at the Hydrogen atom to establish a language with which to elucidate the deceptively simple structure of Helium.  We must be equipped with some study of the coupling between electromagnetic fields and light, given the ubiquitous use of laser sources and magnetic trapping in this dissertation, and the focus\footnote{The \emph{focus} of a lens or mirror is the point of maximum concentration of light that is refracted or reflected from a distant or uniform source. Originally, the term referred to the fireplace at the centre of traditional single-room dwellings; still the brightest point of light, but also the source itself.} on laser spectroscopy. Of course, the fun doesn't end when the lights turn off: Even dark, cold, and dilute helium exhibits important and occasionally explosive two-body interactions, which both pave the road towards absolute zero while imposing limits on the size and lifetime of atomic condensates. Finally, we will review the basic features of the emergence of macroscopic coherence in the form of a Bose-Einstein condensate. 
	% on colons: I thought that complete sentences get capitalized, elaborations don't

\section{Atomic structure}

	The discussion in this section is a short tour of atomic physics. Many high-quality textbooks in atomic physics go into much greater detail than is required for this thesis. In this section, only the essential points are presented. Some lengthy steps in the calculations are omitted, and references provided for the complete working. What follows largely comes from \cite{FootAtomic,BinneyBook} unless otherwise specified. Let us first briefly consider the Hydrogen atom, so that we can discuss its more complex cousin Helium in sensible terms. 
	\todo{Condense repetitive sentences}

	Quantum mechanics is the study of systems whose state at a any time $t$ is completely specified by wavefunction $\ket{\psi(t)}$ and whose dynamics are determined by the time-dependent Schr\"{o}dinger equation,
	\begin{equation}
		i\frac{\partial}{\partial t}\ket{\psi(t)} = \hat{H}\ket{\psi(t)}.
		\label{eqn:TDSE}
	\end{equation}
	\todo{General definition is a jump given the lead in to H atom}

	The wavefunction $\ket{\psi}$ is represented by a \emph{state vector} that is an element of the complex Hilbert space $\mathcal{H}$. The \emph{Born rule} postulates that the probability of a quantum system being observed in the state $\ket{\psi}$ given that it is known to be in the state $\ket{\phi}$ is given by squared inner product $|\braket{\psi}{\phi}|^2$. States are orthogonal if the inner product is zero, but the system may evolve naturally from $\ket{\phi}$ to have a nonzero projection onto $\ket{\psi}$ under the action of $\hat{H}$. The Hamiltonian $\hat{H}$ is a linear operator with the Hermitian property $\hat{H}^\dagger=H$, where the dagger denotes the conjugate transpose operation. \footnote{Formally, $\mathcal{H}$ is a vector space $\mathbb{C}^D$ of dimension $D$ which is complete with respect to the $L^2$ norm $||x||=\sqrt{\braket{x}{x}}$ induced by the inner product $\braket{x}{y}\rightarrow\mathbb{C}$, and $\hat{H}\in\mathcal{B}$, the Banach space of bounded linear operators $\hat{O}:\mathcal{H}\rightarrow\mathcal{H}$. $\mathcal{B}$ is also a vector space with a norm (the trace norm) but not an inner product. The states themselves are defined up to scalar multiplication, and hence are actually rays in $\mathcal{H}$ better thought of as points in projective space; we will simply assume they are normalized as $\bra{\psi}\psi\rangle=1$ for brevity. We will say no more of the tremendous consternations which stem from the Born rule.}	The Hermitian property guarantees $\hat{H}$ is \emph{normal}, $[\hat{H}^\dagger, \hat{H}]=0$ and therefore the time-independent Schr\"{o}dinger equation
	\begin{equation}
		\hat{H}\ket{\psi} = E\ket{\psi}
		\label{eqn:TISE}
	\end{equation}
	specifies the eigenvectors $\ket{e_n}$ of $\hat{H}$ which provide a complete orthonormal basis for $\mathcal{H}$. This allows any (pure) quantum state to be written in the form $\ket{\psi} = \sum_n a_n\ket{e_n}$. An interpretation of this fact in the light of the Born rule is that the energy eigenstates $\ket{e_i}$ correspond to the distinguishable, mutually exclusive states of a system, which can be discriminated from other states by their energy eigenvalue $E_i$. In the cases where eigenvalues coincide, there exists at least one other observable that distinguishes the energy eigenstates. 
	\todo{Clarify/check logic}
	
	Let us consider an atom immersed in an electric field oscillating with frequency $\omega = 2 \pi f$ rad Hz, and write the Hamiltonian in the form
	\todo{Consider better separating atomic structure and atomic interaction if possible, maybe just some extra headings}
	\begin{equation}
		\hat{H}(t) = \hat{H}_0 + \hat{H}_{I}(t)
	\end{equation}
	where the bare atomic Hamiltonian $H_0$ sets the energy scale of the system, and the monochromatic time-dependent perturbation takes the form $\hat{H}_I(t) = \Lambda \cos(\omega t)$. We will give physical meaning to $\Lambda$ in a moment. The energy eigenbasis for a single charged particle bound in a central potential\footnote{Better known as the hydrogen atom} have the form,
	\begin{equation}
	\psi_{nlm}(r,\theta,\phi) = 
	\sqrt{\left(\frac{2}{na_0 ^*}\right)^3\frac{(n-l-1)!}{2n(n+l)!}}e^{-\rho/2}\rho^l L_{n-l-1}^{2l+1}(\rho) Y^{m}_{l}(\theta,\phi)
	\end{equation}
	when written in spherical coordinates $(r,\theta,\phi)$, where $\rho = 2r/na_0 ^*$ and $a_0 ^* = \frac{4\pi\epsilon_0 \hbar^2}{\mu e^2}$ is the reduced Bohr radius.
	The radial Laguerre polynomials $L_{n-l-1}^{2l+1}(r)$ and the spherical harmonics $Y_{l,m}(\theta,\phi)$ are labeled by the angular momentum $l$, the magnetic quantum number $m$, and the energy is fixed by the principal quantum number $n$ as $E_n = -hcR_\infty/n^2$. The bound-state energy $E_n$ is negative - one must do work to ionize the atom and produce the free ion-electron pair whose energy is defined to be zero. The quantum numbers $l$ and $m$ distinguish states that are otherwise degnerate in $n$, and serve to lift the degeneracy via the Zeeman shift as I discuss in a later section. Here,  $h=2\pi\hbar$ is the Planck constant, $\varepsilon_0$ is the electric permittivity of free space, and $\mu$ is the Bohr magneton.
	
	

	Let us recall\todo{Usually used in reference to prior discussion. Restate.} that the time-dependent state can be written in terms of the eigenbasis $\ket{\psi_n}$ of $\hat{H}_0$,
	\begin{equation}
		\ket{\Psi(t)} = \sum_n c_n(t)e^{-i\omega_n t}\ket{\psi_n},
	\end{equation}
	where $\omega_n= E_n/\hbar$. Substitution into Eqn \ref{eqn:TDSE} reduces to the coupled set of differential equations 
	\begin{equation}
		i\hbar\dot{c}_n(t) = \sum_m e^{-i(E_m-E_n)t/\hbar}\bra{\psi_m}\hat{H}_I\ket{\psi_n},
	\end{equation} 
	which can be solved succinctly after making two simplifications. First, let us assume (without loss of generality) that the atom is initially in the state $\psi_1$, where $c_1(0)=1$ and $c_i(0)=0~\forall i\neq1$. Then at a time $t$ the probability that the atom is in the state $\psi_k$ is \cite{FootAtomic,BinneyBook}
	\begin{equation}
		P_k(t) = 4|\bra{\psi_k}H_i\ket{\psi_1}|^2 \frac{\sin^2(\frac{1}{2}(\omega_k-\omega)t)}{(\omega_k-\omega)^2},
	\end{equation}
	where $\omega_k = (E_1-E_k)/\hbar$. An immediate consequence is that the denominator in Eqn, (1.7) suppresses excitation into $\ket{\psi_k}$ unless $\omega$ is close to the resonant frequency $\omega_k$. This weak driving permits us to make a further simplification and consider an atom with only \emph{two} states, labeled 1 and 2, with a resonant frequency $\omega_0 =(E_2-E_1)/\hbar$. Further, the numerator captures the essential feature that the state oscillates between states in response to the driving field. We are now ready to impart a physical meaning to the coupling term: An atom exposed to an oscillating field will respond by oscillating between energy eigenstates, each of which having their own charge distribution through space. This produces an oscillating electric dipole whose interaction with the electric field can be simplified by working in the \emph{dipole approximation}: Assume the electric field has a constant value throughout space, but oscillates in time as $\textbf{E}(t) = \textbf{E}_0 \textrm{Re}(e^{-i\omega t}\hat{\varepsilon})$, where $\hat{\varepsilon}$ is the unit polarization vector, and $\textbf{r} = r\hat{\textbf{r}}$. The interaction energy is then given by retaining only the dipole operator $-e\textbf{r}$ from the multipole expansion of the electronic charge distribution\footnote{This is a good approximation when the wavelength $\lambda = c/f$, fixed by the vacuum speed of light $c$, is much larger than the atom This is universally applicable for our purposes as the size of the helium atom is $\approx 30$pm, some 0.1\% of the shortest wavelength of light used in this thesis}. The interaction Hamiltonian can then be written as
	\begin{equation}
		\hat{H}_I = e\textbf{r}\cdot\textbf{E}(t),
	\end{equation}
	and the excitation probability can be written in the more familiar form \cite{FootAtomic,BinneyBook}
	\begin{equation}
		P_2(t) = \Omega^2 \frac{\sin^2(\frac{1}{2}(\omega_0-\omega)t)}{(\omega_0-\omega)^2},
		\label{eqn:transition_prob}
	\end{equation}
	in terms of the Rabi frequency
	\begin{equation}
		\Omega = \frac{\bra{\psi_1}e\textbf{r}\cdot\textbf{E}\ket{\psi_2}}{\hbar}.
	\end{equation}
	\todo{Physical interpretation of rabi freq here, perhaps?}
	
	The dipole operator $-e\textbf{r}$ inherits the structure of the orbitals, which allows a separation of the expectation value in the preceding equation into radial and angular parts, $\bra{2}\textbf{r}\cdot\hat{\varepsilon}\ket{1} = \bra{2}R\ket{1}\mathcal{I}$. Setting aside the radial part $R$, the angular integral $\mathcal{I}$ can be written as a contraction over the spherical harmonic basis functions, 
	\begin{equation}
		\mathcal{I}=\int_{0}^{2\pi}\int_{0}^{\pi} Y^{*}_{l_2,m_2}(\theta,\phi)\hat{r}\cdot\hat{\varepsilon}Y_{l_1,m_1}(\theta,\phi)\sin\theta d\theta d\phi
	\end{equation} 
	which is zero unless some constraints, known as \emph{selection rules}, are satisfied. To proceed, we assume that the atom is immersed in a magnetic field and define the $z$ axis to be the direction of the magnetic field vector $\textbf{B}$\footnote{This is true for almost all the contexts we encounter in this thesis, but where a magnetic axis is not present, one should perform an average over all angles as required.}. The dipole operator can then be written as the superposition of the linear and circular oscillating field components,  
	\begin{equation}
		\hat{r}\cdot\hat{\varepsilon}\propto A_{\sigma^-}Y_{1,-1} + A_z Y_{1,0} + A_{\sigma^+}Y_{1,+1},
		\label{polz_decomp}
	\end{equation}
	where the $A_{\sigma^\pm}$ are the amplitudes of the clockwise- and anti-clockwise circular polarization, and $A_z$ the amplitude of linear polarization in the atomic reference frame. When this expression is inserted into the integral $\mathcal{I}$, the orthogonality of spherical harmonics and the relationship
	\begin{equation}
		Y_{1,m}Y_{\lambda,\mu}= A Y_{1+\lambda,m+\mu}+BY_{\lambda-1,m+\mu}
	\end{equation}
	 ensures
	\begin{equation}
		\Omega\propto a\delta_{l_2,l_1+1}\delta_{m_2,m_1+m} + b\delta_{l_2,l_1-1}\delta_{m_2,m_1+m}
	\end{equation}
	where $m=\pm1,0$ as in the expansion of the dipole operator, and $a$, $b$ are constants whose exact values are not required here. Thus $\mathcal{I}=0$ unless $\Delta l=\pm1$ and either $\Delta m_l=0$ or $\Delta m_l=\pm1$. Transitions not satisfying these conditions are said to be \emph{forbidden}, but in reality they still occur, and can be accounted for by including higher terms in the series expansion of the interaction Hamiltonian. Emission or absorption events with $\Delta m_l=0$ are called $\pi$-transitions, and correspond to the dipole moment induced by light with linear polarization along the quantization axis, whose amplitude is captured by the $A_z$ term in Eqn. \ref{polz_decomp}. The $\sigma$ transitions couple to oscillations in the plane normal to the quantization axis, and correspond to transitions where $\Delta m_l=\pm 1$, driven by the $A_{\sigma^\pm}$ terms. 

	\todo{orientation dependence goes here}

	This description of the coupling between atomic states and the electric field is sufficient, with some further work, to derive the Einstein rate equations for absorption and stimulated emission \cite{FootAtomic}. However, the more familiar phenomenon of spontaneous emission remains out of reach and requires the full-blown theory of quantum electrodynamics for an explanation, so we shall not proceed to derive it here. Instead we will have to be satisfied with a classical model which captures some of the essential features so far left unmentioned.


% Can potentially shoehorn the OBEs in here 

	\subsection*{Classical oscillator model}

	In the \emph{Lorentz oscillator} picture, one approximates an atom by a positive charge fixed at the origin with a harmonically-coupled negative charge with a single degree of freedom $x(t)$ whose potential is zero at $x(t)=0$. The electron's equation of motion is $\ddot{x} + \Gamma\dot{x} + \omega_0^2x = -e E(e)/m_e$, where
	\begin{equation}
		\Gamma = \frac{e^2\omega^2}{6\pi \varepsilon_0 m_e c^3}
	\end{equation}
	is the damping rate corresponding to radiative decay. The spectrum of an exponentially decaying oscillator has the familiar Lorentzian lineshape,

	\begin{equation}
		\alpha(\omega) = \frac{6\pi\varepsilon_0c^3}{\omega_0^2}\frac{\Gamma}{\omega_0^2-\omega^2-\textrm{i}(\omega^3/\omega_0^2)\Gamma},
	\end{equation}

	Where the polarizability $\alpha(\omega)$ determines the amplitude of the dipole response via $\pvec(t) = \alpha(\omega)E(t)$. This reponse function features the characteristic full-width at half-maximum (FWHM) scale of each transition, which is the inverse of the lifetime $\tau=1/\Gamma$. The interaction energy of the dipole and the electric field is $U_{dip} = -\frac{1}{2}\langle\textbf{p}\cdot\textbf{E}\rangle = -\frac{1}{2\varepsilon_0 c}Re(\alpha)I(\rvec)$, which inherits a spatial structure from the intensity $I(\rvec)$. When the polarizability is positive (that is, when the light is red-detuned from $\omega_0$ and the denominator has a positive real part) then the dipole oscillates in-phase with the field and so the interaction energy is minimized at the intensity maxima. This is the operational principle of optical dipole traps, wherein the \emph{dipole force} $F_{dip} = -\nabla U_{dip} \propto \nabla I(\rvec)$ confines atoms to the focus of a laser beam\todo{assumes some stuff about the mode structure.  Rephrase.}\footnote{the dipole force can also be generated with blue-detuned beams to create repulsive potential barriers}. Ashkin made the first demonstration of the dipole force by trapping micron-sized particles in 1970, and Letokhov suggested 1D confinement of atoms the next year. Ashkin suggested 3D trapping only in 1978, and shortly afterward the dipole force was demonstrated on neutral atoms by Bjorkholm \emph{et al.} 1978. In 1986 Chu \emph{et al.} accomplished the first optical trap, and the first BEC produced exclusively using optical trapping was achieved in 2001 \cite{barrett01}. The dipole force is most pertinent to the works in chapter \ref{chap:lattice} as it is the basic principle underpinning optical lattice traps. \todo{cite the various papers}

	In the case where the detuning $\Delta$ is small in comparison to $\omega_0$ (as will be the case throughout this dissertation), the dipole potential can be written in the form \cite{grimm00}
	\begin{equation}
		U_{dip}(\textbf{r}) = \frac{3\pi c^2}{2\omega_0^3}\frac{\Gamma}{\Delta}I(\textbf{r}),
	\end{equation}
	\todo{This is def atom-light interactions, title it appropriately}
	and absorption of light is captured by the imaginary part of the polarizability\footnote{A careful derivation can be found in \cite{FootAtomic}, and an extremely detailed one in \cite{CohenTannoudji}}, which is related to the scattering rate as
	\begin{equation}
		\Gamma_{sc} = \frac{\textrm{Im}(\alpha)I(\rvec)}{\hbar\varepsilon_0 c}.
	\end{equation}
	Light scattering competes with the dipole force because repeated absorption of photons with momentum $\hbar k$ at a rate $\Gamma_{sc}$ gives rise to an effective force of $F_{sc}=\Gamma_{sc}\hbar k$, not to mention the deleterious effects of heating by repeated absorption events. Fortunately, in all situations relevant to our concerns here, the scattering rate can be written 
	\begin{equation}
		\Gamma_{sc}(\textbf{r}) = \frac{3\pi c^2}{2\hbar\omega_0^3}\left(\frac{\Gamma}{\Delta}\right)^2 I(\textbf{r}),
	\end{equation}
	from which it can be seen that $\Gamma_{sc}/U_{dip} \propto \Gamma/\Delta$ - that is, for large enough detuning, the scattering rate is dominated by the dipole force. 

	\todo{stuff below would be better discussed in spectroscopy chapter}
	Returning to multilevel atoms, a fine way to better approximate the atomic polarizability is to compute the dipole potential of an atom initially in the $\ket{1}$ state by summing over the level shifts associated with transitions to all other states, giving the form
	\begin{equation}
		\Delta U = \sum_i \frac{|\bra{1}H\ket{i}|^2}{\omega_i}.
	\end{equation}
	Because the polarizability arising from an atomic transition is negative (positive) when an electric field is blue (red) detuned with respect to the respective resonance, there exist wavelengths between transitions for which $\alpha(\omega)\propto\Delta U = 0$. These \emph{tune-out} wavelengths have utility for mixed-species traps and as precision test of QED \cite{henson15,mitroy13,TOforthcoming}. During the course of my PhD research, I undertook a measurement of the the 413nm tune-out wavelength in helium under the leadership of B M Henson, and we achieved a 25x improvement over the last measurement. This measurement was the major motivation for acquiring the tunable laser described in the next chapter, and used to perform the experiments described in chapter \ref{chap:spectroscopy}.

	In heavier atoms than Hydrogen\footnote{Or as astronomers are wont to call them, `metals'}, interactions between electrons also play an important role, as we will see in the case of the \emph{most noble of metals (omit?)} and the central element of this thesis: Helium.
		
\section{Helium} 

	The idealized model of the hydrogen atom is fine for illustrating some important features of atomic physics, but the focus of this thesis is, of course, helium. Although the structure of the helium atom is simple enough that theoretical calculations can confidently accrue many significant figures, with precision rivalling similar calculations for Hydrogen, the presence of a second electron does considerably complicate the physics\footnote{As the old joke goes, atomic physicists count `\emph{one, two, many...}'}. The Helium Hamiltonian 
	$$
	\left(\frac{-\hbar^2}{2m}\nabla_{1}^2+\frac{-\hbar^2}{2m}\nabla_{2}^2 + \frac{e^2}{4\pi\epsilon_0}\left(-\frac{Z}{r_1}-\frac{Z}{r_2}+\frac{1}{r_{12}}\right) \right)\ket{\psi} = E \ket{\psi}
	$$

	\noindent includes kinetic terms $\propto\nabla_i^2\ket{\psi}$ and a central potential $\propto 1/r$ for each electron, plus a repulsive interaction inversely proportional to the electron sepration $r_{12}$. The presence of a second electron also introduces another defining feature of quantum mechanics: spin. The electron wavefunctions must be antisymmetric under exchange of particle labels because all fermions obey the Pauli exclusion principle. On the other hand, the Hamiltonian is invariant under exchange of the electrons. If we denote the exchange operator by $\hat{X}$, then we have $[\hat{H},\hat{X}] = 0$, implying $\hat{X}$ and $\hat{E}$ have the same eigenstates. Therefore the energy eigenbasis satisfies $\hat{X}\ket{e} = -\ket{e}$. Because the electron wavefunctions have the quantum numbers $\ket{n,L,m_L,S,m_S}$ which can be separated into the product of spatial ($\ket{n,L,m_L}$) and spin ($\ket{S,m_S}$) parts. It must be, then, that a given eigenstate must be of the form $\ket{\psi} =  \psi^{S}_{space}\psi^{A}_{spin}$ or $\psi=\psi^{A}_{space}\psi^{S}_{spin}$
	\todo{Aren't the numbers for quantum states J, m_J for helium?  Just make sure it's consistent between the intro and the spectroscopy chapter.}
	We can enumerate the possibilities for the symmetric (spin) wavefunctions, 
	% \begin{equation}
	\begin{align}
	\psi^S_\text{spin} =& \ket{\uparrow\uparrow},\\
		&\ket{\downarrow\downarrow},\\
		&(\ket{\uparrow\downarrow}+\ket{\downarrow\uparrow}))/\sqrt{2}
	\end{align}
	% \end{equation}
	 and the antisymmetric term
	\begin{equation}
	\psi^A_\text{spin} = (\ket{\uparrow\downarrow}-\ket{\downarrow\uparrow}))/\sqrt{2}
	\end{equation}

	\noindent which are all degenerate in energy because the atomic Hamiltonian does not couple to the spin sector. For a singly-excited helium atom, as will always be the case in this thesis, the interaction term splits the spatial part of the wavefunction for a given set of quantum numbers into symmetric and antisymmetric forms
	$$
	\psi^{A}_{space} = \frac{1}{\sqrt{2}}\left(u_{1s}(1)u_{nl} - u_{1s}(2)u_{nl}(1)\right)
	$$
	$$
	\psi^{S}_{space} = \frac{1}{\sqrt{2}}\left(u_{1s}(1)u_{nl} + u_{1s}(2)u_{nl}(1)\right)
	$$


	which furnish the spin wavefunctions to form the $n\triplet L_{J}$ and $n\singlet L_{J}$ states, referred to as (triplet) ortho- and (singlet) para-helium, respectively\footnote{In what follows, we use the spectroscopic convention and label states by the $n^{2S+1}L_J$, where $J=S+L$ because the helium nucleus is spinless, and we drop the $(1s)$ term as the second electron will invariably be in the ground state in all cases we consider -  Doubly-excited helium is highly unstable because the first excited state (19.8eV) has about 80\% of the ionization energy of Helium (25.4eV) \todo{is doubly-excited helium even possible, then?}}. It can be shown via degenerate perturbation theory that the exchange antisymmetry produces the \emph{exchange energy} difference between the singlet and triplet states, where the  orthohelium state has a lower energy.  The $\metastable$ state, also denoted \mhe, distinguishes helium amongst the zoo of atomic species available to the cold atom physicist by virtue of its 19.8 eV excitation energy and $\approx 7800$s lifetime \cite{hodgman09}.  This state is forbidden to decay to the ground state by the $\delta L\neq0$ selection rule and also because the dipole operator does not couple states of different spin, such as the \mhe state and $1s^2$ \todo{ensure notation is consistent} ground state. Decay from the metastable state is  therefore called \emph{doubly forbidden}. Forbidden transitions can generally occur by higher-order interactions that are omitted from the dipole approximation, and the \mhe state in particular can decay via a magnetic dipole transition. Nonetheless, the two-hour lifetime is effectively a ground state for the purposes of ultracold helium experiments, which generally last less than a minute. Fortuitously, the metastable state is connected to the $2\triplet P_2$ state by a transition with a wavelength of 1083.331nm, which is far more readily accessible with compact laser systems than the $\leq 63$nm X-ray transitions from the true ground state. The $2\triplet P_2$ state also provides a\todo{n essentially} closed transition cycle for laser cooling as it can only decay to the \mhe state\todo{Not strictly true - there is also a  forbidden decay path to the ground state (see e.g. our paper where we measured it!).  However, with an inverse decay time of ~3s decays to the metastable state dominate.}, obviating the need for an additional \todo{re?}pump laser. 

	The astute reader will notice the absence of presentation of an explicit form for the electron wavefunctions in the helium atom. Indeed	helium is not analytically solvable because its eigenfunctions are not separable into a form $\Psi = \psi_1\otimes\psi_2$. Furthermore, modern spectroscopy of helium has advanced to the level of accuracy where nuclear recoil effects must be included, which can be incorporated in the form of a series expansion in powers of  $m_e/M_He\approx10^{-4}$. Relativistic effects must be represented as another series expansion in powers of the fine structure constant $\alpha=\sqrt{2h c R_\infty/m_e c^2}\approx1/137$ \todo{citations for this}. The combination of both corrections constitutes a double-series expansion of the form $\sum_i\sum_j \alpha^i(m_e/M_{He})^j$, and these terms can be calculated with sufficient accuracy to compete with modern experiments. A variational approach is required for tractable and accurate calculations, which was developed by Hylleraas \cite{Hylleraas1920,Hylleraas1929,Hylleraas1930}. In the intervening century, numerical methods for calculating the energy levels and transition rates in the Helium atom have kept pace with precision experiments, and a survey of recent progress is given in chapter \ref{chap:spectroscopy}.

	 % Jones and Stokes calculus  
	 % Waveplates  
	 % Atomic polarizability
	
	In real atoms, electrons are also bound by the laws of special relativity and interact with external fields and with the nucleus, which means the single-electron wavefunctions do not correspond perfectly to this form. 

\subsection*{Magnetic fields and the Zeeman effect}

	The inclusion of spin introduces another important feature of atomic spectra, the Zeeman effect, whose discovery heralded a `watershed' moment of modern physics. The Zeeman effect refers to the phenomenon of spectral line splitting that occurs when an atom is immersed in a DC magnetic field, the synthesis of the prior concepts of the \emph{normal} and \emph{anomalous} Zeeman effect \todo{A one sentence reminder to the reader of what these are would be good}. The interaction energy of an atom in a magnetic field, 
	% \footnote{Along with  the discoveries of X-rays, radioactivity, and the observation that `cathode rays' had a mass-to-charge ratio equal to particles within atoms, now known as \emph{electrons}\cite{FootAtomic}}. 
	% H spectrum provided observations to challenge class mech, then QM, then RQM, and indeed the current QED
	\begin{equation}
		H = -\mu\cdot \textbf{B},
	\end{equation}
	has contributions from both orbital and spin angular momenta ($\textbf{L}$ and $\textbf{S}$, respectively) through the atom's magnetic moment 
	\begin{equation}
		{\bf \mu} = -\mu_B\textbf{L} - g_s\mu_B \textbf{S}.
	\end{equation}
	Working in the $\ket{LSJm_J}$ basis, where $J$ and $m_J$ are the total angular momentum and its z-projection, yields the eigenenergies $E_Z = g_J \mu_b B M_J$. The atomic g-factor can be written as
	\begin{equation}
		g_J = \frac{3}{2} + \frac{S(S+1)-L(L+1)}{2J(J+1)},
	\end{equation}
	using the approximate value of the electron g-factor $g_s=2$.The eigenstates of the field-free atomic Hamiltonian will be $2J+1$-fold degenerate and specified by the $\ket{Lm_L S m_S}$ quantum numbers. Adding the magnetic field interaction breaks this degeneracy and leads to the Zeeman splitting. The field-free and magnetic-interaction terms can be written in a common basis in terms of the Clebsch-Gordan coefficiencts and then diagonalized, as described in chapter \ref{chap:spectroscopy}. The aforementioned anomalous Zeeman effect arises in triplet states because the inter-level spacing depends on $m$ and $g_J$, which splits spectral lines as well as levels, whereas singlet states have $g_J=1$ and transitions between them do not fan out in the same fashion. This distinction, whose explanation was an early victory for quantum mechanics, is illustrated in Figure \ref{fig:ZeemanLines}. 

	% More than merely splitting lines, the Zeeman effect also induces a polarization-dependence  relative to the quantization axis. The classical dipole envisions only circular polarization is visible along the quantization axis (z oscillation is not visible), but from transverse directions one can see the z oscillation and also the one-dimensional oscillations (eg along x when viewing the y axis) - so in a spatially varying mag field... And we're nearly at the point of being able to project the Stokes vectors into the local field
	


	% RSP [10] ac stark shift -> dipole interaction, can include more levels for better calc [126] 
	% A Askkin acceleration and trapping of particles by radiation pressure, phys rev lett 24, 1970
	% RSP scattering happens [68] 
	% % Grimm, dipole traps
	
\section{Interactions between atoms}
\todo{Might make more sense after BEC}

	For all the intricacy of atomic structure, life would be very dull indeed were it not for the interactions between them. Indeed, the material reality of the world depends, in a sense, less on the structure of its building blocks and more on how they fit together. The varied and central roles of interactions will be revisited in later chapters, but for now it will suffice to note that modern experimental techniques allow access to a large range of interaction strengths and modalities. For example, optical lattices or Feshbach resonances can be used to tune short-range interaction parameters between extreme values, from the ultradilute gases to the so-called strongly correlated regime. The spectrum of possibilities from isolated, to weakly-interacting, to distinctly many-body systems will be traversed over the course of the following chapters. 
	
	Here we briefly review elastic scattering, and then turn to important inelastic scattering processes present in our experiments. A detailed primer in atomic scattering physics can be found in the classic texts \cite{PitaevskiiStringari} and \cite{PethickSmith}, with more detail in the latter. An exhaustive review of low-temperature scattering studies up to the turn of the millenium can be found in \cite{weiner99}. I focus here on two-body collisions, which are by far the dominant interactions in the low-density regime of ultracold gases\todo{given the importance of 3-body losses as the dominant loss process in non-He* systems, I'd maybe tone this down a little to either say that 2-body are only dominant in He* or that 2-body dominates the elastic scattering}. Low densities imply that low temperatures are required to achieve high phase space density and reach the degenerate regime. Two-body collisions are the crucial enabler for thermalization during the evaporative cooling necessary to reach such low temperatures, so long as the relaxation times are shorter than the sample lifetime, which is also determined by kinetic effects like three-body recombination or, in helium, Penning ionization. 

	Neglecting spin-orbit and relativistic effects the two-body scattering problem reduces to the Keplerian Schr\"{o}dinger equation in the centre-of-momentum frame,
	\begin{equation}
	\left(\frac{\hbar^2}{2m^*}\Delta + V(r) - E\right)\psi(r) = 0,
	\end{equation}
	\todo{psi here is the two atom wavefunction?  Might be good to specify, given that you've been dealing exclusively in single atomic wavefunctions up to here}
	in terms of the separation $r=|\rvec_1-\rvec_2|$ between the particles and the reduced mass $m^*=m_1m_2/(m_1+m_2)$. In the asymptotic regime where $r$ is much larger than the scale of $V(r)$, the solution takes the form of a superposition of the initial plane wave and the scattered solution,
	\begin{equation}
	\psi(r) \propto e^{ikz} + f(\theta)\frac{e^{ikr}}{r},
	\end{equation}

	where $k=\sqrt{2m^*E/\hbar}$ is the plane wave-vector of the initial approach and $\theta$ is the angle from the direction of incidence. A general solution can found by expanding $f(\theta)$ into a convenient basis of \emph{partial waves} (spherical harmonics) which are labeled $s,p,d,f,..$ in order of increasing angular momentum. In the low-energy limit, $f(\theta)$ is independent of angle and only the spherically symmetric s-wave term contributes, and the limit $f(\theta)\rightarrow-a$ is accordingly called the s-wave scattering length. At temperatures below 5mK \todo{oddly specific. Unique to He? Relax specificity?} the scattering physics is determined by just a few partial waves \cite{mcnamara07}, and in the ultracold regime only the s-wave scattering channel is significant. The total cross section $|f(\theta)^2|/|\psi|^2$, which is the total probability that a near collision results in particle scattering, approaches $\sigma=8\pi a^2$ for polarized bosons\footnote{For fermion pairs with odd total spin, the cross section tends to zero because of the Pauli exclusion principle, and thus the s-wave scattering length vanishes.} \cite{przybytek05}. 

	The s-wave scattering length is also an important determinant of the energetics of degenerate matter such as BEC. Because BECs are dominated by long-wavelength behaviour, a theoretical treatment can be considerably simplified by considering only the \emph{effective interactions}. By formulating the scattering problem in momentum space, the effective interaction strength for low-energy scattering  $g=4\pi \hbar^2 a/m$, also referred to as the pseudopotential, can be found by integrating out the high-frequency modes (also known as the Born approximation). This necessarily washes out extremely short-range correlations but makes fairly accurate calculations much more tractable, in particular those employing Hartree or Hartree-Fock methods. It is therefore difficult to overstate the importance of accurately knowing the s-wave scattering length for the purposes of understanding ultracold gas experiments. \todo{Some citations could be good}

	In molecular collisions the scattering process will obviously depend on the relative orientation of the molecules. In collisions between single atoms, though, there is a more subtle orientation-dependence which arises from the total spin of the two-particle system. The three possible configurations between pairs of \mhe atoms correspond to the singlet $^1\Sigma_g^+$, triplet $^3\Sigma_u^+$, and quintet $^5\Sigma_g^+$ Born-Oppenheimer molecular potentials, with total spin 0, 1, and 2\footnote{the subscript \emph{g} and \emph{u} are short for \emph{gerade} and \emph{ungerade} (German for even and odd) label the reflection symmetry of the two-body wavefunction, and the superscript + sign...}. When the atoms are spin-polarized, as they are when confined in magnetic traps, then the only scattering that occurs is in the quintet channel. The most accurate determination of the s-wave scattering length in this configuration is 7.512nm \cite{moal06}, in agreement with calculations performed the year before the measurements \cite{przybytek05}. \todo{why is $a_{11}=a_{10}$?}
	
	A powerful tool available in some cold atom experiments are Feshbach resonances\footnote{Feshbach resonances originated in Feshbach's work in nuclear physics (Feshbach 1958, 1962) and Fano worked on in atomic context (fano 1961), but these fano-feshbach resonances are referred to in general as feshbach resonances. }. A detailed description is found in \cite{Chin10}, but from an operational standpoint they allow control of the scattering length as $a = \tilde{a}(1-\Delta/(B-B_0)$, where B is the strength of an ambient magnetic field, $B_0$ is the resonance value of the field, $\tilde{a}$ is the value when the field is far from a resonance and $\Delta$ sets the resonance width. The scattering length can thus be tuned in size and even in sign. Fundamentally, the stability of BEC requires a positive s-wave scattering length\footnote{\todo{although in a lattice negative T states can support a stable BEC with attractive interactions}}, and switching from stable to unstable configurations permits one to examine condensate collapse (as in the spectacular Bosenova experiments \cite{papers}) and also to explore the BEC-BCS crossover \cite{papers}. The spinless nucleus of $^4$He prohibits coupling of bound states within an $m_f$ \todo{as in... there isn't any meaningful $m_f$?} manifold from crossing an open-channel threshold, precluding this pathway to a Feshbach resonance \cite{goosen10}. Nonetheless, Feshbach resonances induced by spin-spin interactions between helium atoms have been predicted \cite{venturi99, goosen10}, but have not observed to date \cite{borbely12}. \todo{Interestingly, I'm working on a paper with Danny and one of his students that explains the experimental non-observation (the previous theory was wrong).  Sadly, it says that there are no experimentally useful FB resonances for He*...}
	
	Inelastic scattering processes are those which exchange energy between the internal and motional states of either atom. They can be represented as a complex scattering potential \cite{leo01} which permits losses from on-shell scattering channels. An important inelastic process characteristic of metastable noble gases \cite{VassenReview} is \emph{Penning ionization}.  This can occur through the decay channels
	\begin{equation}
		\textrm{He}^*+\textrm{He}^*\rightarrow 
		\begin{cases}
			\textrm{He}^* He^+ + e-&\textrm{(PI)}\\
			\textrm{He}_{2}^{+} + e-&\textrm{(AI)}
		\end{cases}
	\end{equation}
	\todo{Neither atom is excited after PI}
	\todo{Why is this spacing so huge?}
	Formally, the first channel is called Penning ionization and the second is called Auto-ionization, but the rate of the latter is insignificant in comparison to the former \cite{Muller91}. \todo{this sentence doesn't add much} Indeed, the energy of the metastable state is sufficient to ionize any neutral atom (except helium or neon) from its ground state, and \cite{bell68} presents a tabulation of the ionization rates involving other species. Aside from attracting intensive study in its own right \cite{partridge10,stas06,mcnamara07}, this explosive potential was a significant hurdle for researchers attempting to achive Bose-Einstein condensation with helium. The density achieved in early magneto-optical traps (MOTs) was limited to some hundredfold less than the alkali-metal MOTs of the day \cite{bardou92,kumukura92,mastwijk98}. Helium MOT densities were limited by losses through two-body collisions involving atoms in the $\metastable$ and those excited to the $2\triplet P_2$ state by the trapping beams\todo{I'm pretty sure the ionisation rate for excited-excited state is something like 5 orders of magnitude greater than for ground states, so this Penning process will dominate.}, as opposed to rescattering pressure as in the case of alkali metals. Such light-assisted collisions limited the density of helium MOTs and thus constrained their trapped population  number until larger beams and detunings were used \cite{tol99}, whereby Penning ionization rates were reduced to the order of 5e-9 cm$^3$/s at large detunings, an improvement of about a factor of twenty. In the dark, the rate constant is a factor of 50 lower again, but the process is  still too fast to permit access to magnetic traps of sufficient lifetime and density to reach degeneracy. Fortunately, the inelastic scattering cross-sections depend on the molecular potentials in such a way that condensation becomes attainable: When all the atoms are polarized in the either of the $m_J=\pm1$ states, the collision takes place in the $^5\Sigma_g^+$ potential\todo{notation probably not necessary; maybe in a footnotes explaining the scat len equivalence}, while the reaction products have a total spin of 1, and so this process is forbidden. In reality, it does occur through a weak virtual spin-dipole transition \cite{shlyapnikov94} , but slowly enough that spin-polarized \mhe exhibits a $10^4$-fold reduction in ionization rate. At field strengths above 50G, however, the suppression weakens \cite{shlyapnikov94,Borbely12}. Other noble gases also exhibit highly energetic metastable states, but the lifetime and suppression of Penning ionization decreases with increasing mass \cite{orzel99, spoden05}. Thus Helium may be the only noble gas ever to be Bose-condensed. 
	% F bardou, O emile, J M courty, C I Westbrook, A Aspect, magneto-optical trapping of metastable ehlium: Collisions in the presence of resonant light, Europhysics letters 20, Dec 1992
	% M Kumakura N Morita, visible observation of metastable helium atoms confined in an invisible/visible resonance trap, japanese journal of applied physics 31, march 1992
	% H C Mastwijk, J W thomsen, P van der straten, A niehaus, optical collisions of cold, metastable helium atoms, physical review letters 80, june 1998

	% \cite{shlyapnikov94} predicts 1e5x reduction in PI in spin-pol samples; relaxation-induced penning; virtual spin-dipole transitions (induced by spin-dipole interaction) to the zero spin state of the quasimolecule can lift the spin-conservation rule and lead to regular penning ionization	Direc dipole-exchange ionization is also a possibility but dominated by spin-relazation (relaxation-induced penning; virtual spin-dipole transitions to the zero spin state of the quasimolecule can life the spin-conservation rule and lead to regular penning ionization)
	
	
	

\section{Bose-Einstein condensation in dilute gases}

		% M H ANderson, J R Ensher, M R Mathews, C E Wieman, and E A cornell, observation of bose-einstein condensation in a dilute atomic vapor, Science 269, 198-201, July 1995
		% K B Davis, M O Mewes, M R Andrews, N J van Druten, D S Durfee, D M Kurn, W Ketterle, Bose-Einstein condensation in a gas of sodium atoms - Physical Review Letters 75, 3969-3973, Nov 1995
		% C C Bradley, C A Sackett, J J Tollett,  G Hulet, evidence of Bose-Einstein condesnation in an atomic gas with attractive interactions, phyiscal review letters 75, 1687-1690, aug 1995
		% W D Phillips, H Metcalf, Laser deceleration of an atomic beam, Phys Rev Lett 48, 1982
		% Chu et al, three-dimensional viscous confinement and cooling of atoms by resonance radiation pressure, phys rev lett 55, 1985
		% Ch et al, experimental observation of optically trapped atoms, phys rev lett 57, 1986
		% Raab et al, trapping of neutral sodium atoms with radiation pressure, phys rev lett 59, 1987
		% P D Lett et al, observation of atoms laser cooled below the Doppler limit, phys rev lett 61, 1988

% \subsection{What is a BEC?}
	% Things I might like to put in later: Partition function Z = Tr(\exp(\beta H)), expectation values Tr(O rho)/Z, 

	The `fifth state of matter'\footnote{The familiar first phases, solid, liquid, and gas, are vanishingly rare in cosmological terms. The fourth, plasma, is the state of at leats 99\% of the ordinary matter in the universe \cite{plasmastuff}. Helium comprises about 23\%, most of which being primordial baryons formed during the recombination epoch.}  has a long and storied history\cite{mukundanote}. Following the oft-cited seven decades between the initial theoretical descriptions and the experimental realization of atomic Bose-Einstein condensates (BEC) \cite{anderson95,davis95,bradley95}, the topic has become an industry \todo{Can a field really become and industry?} in its own right at the eve of its centenary. As pithily put by a review only five years after the Nobel-winning experiments, `Any attempt to review recent progress is out of date as soon as it is published' \cite{courteille01}. This is no less true today, as the number of ultracold gas experiments worldwide now number nearly 200\footnote{See \url{everycoldatom.com}} and numerous companies have been founded on the promise of selling better sensors and computers \todo{based on BEC technology}. Back in the middle of the 20th century, BEC garnered further attention when Fritz London proposed that Bose-Einstein condensation was connected to the superfluid phenomenon in liquid helium. Nikolay Nikolayevich Bogolyubov\footnote{Nikolay was a darling of Russian theoretical physics, receiving his PhD-equivalent qualification at 19 and made important contributions to quantum field theory. In his famous paper on the problem of interacting bosons, his name is transliterated as \emph{Bogolubov}. Bogolyubov and \emph{Bogoliubov} are also common transliterations.} formalized this connection and so, historically speaking, helium was the element which hosted the earliest experimental realization of Bose-Einstein condensation, albeit with a very small condensed fraction. While liquid helium is a rare thing in cosmological terms, BEC may have existed already for millions of years in the superdense quark matter of neutron stars \cite{haskell18, martin16,baym69,page11}, wresting the claim of cosmic novelty from human hands \todo{The arguments that superfluids and neutron stars are BECs is why the field is normally referred to as BEC in dilute atomic gases}. Nonetheless, the essentially pure atomic condensates and the emerging study of molecular condensates in laboratory settings are surely among the most extreme conditions in the universe. There are numerous treatments of the theory of Bose-Einstein condensation, for example the classic textbooks \cite{PitaevskiiStringari,PethickSmith} and review articles \cite{DalfovoReview, yukalov11_basics,courteille01}. The essential background for discussion here draws on these standard sources unless otherwise cited.
	% anquez16 also cites Bogoliubov theory as being used in analysis of quantum phase transitions in spinor condensates


	% There's something to be said here; there is a distinction between the ground state, which is global, and the single-particle states...
	The canonical description of Bose-Einstein condensation is the condition where the de Broglie wavelength associated with thermal kinetic energy
	\begin{equation}
		\lambda_T = \frac{h}{\sqrt{2\pi m k_B T}}
	\end{equation}
	is comparable to the interparticle spacing, coinciding with a macroscopic occupation of the single-particle ground state as the de Broglie waves of many bosons constructively interfere. A precise criterion for this condition is when the phase space density
	\begin{equation}
		\aleph = n \lambda_T^3
	\end{equation}
	\todo{Maybe you could be a little clearer with this explanation - talk about how below this critical value the de Broglie waves overlap}
	exceeds the critical value of $\zeta(3)\approx2.612$. However, in the presence of interactions, this criterion is unsatisfactory: Just as interactions between electrons preclude an expression of the stationary states of helium in terms of products of single-particle energy eigenstates, the stationary states of an interacting gas of $N$ atoms cannot be written as a prduct $\ket{\Psi} = \ket{\psi_i}^{\otimes N}$ of single-particle eigenstates $\ket{\psi_i}$. The Penrose-Onsager criterion \cite{penrose56} provides an alternative in terms of the density matrix $\rho$ for the isolated composite system comprised of all the gas particles. The single-particle density matrix is then the expected value of the one-body field operator
	\begin{equation}
		\rho^{(1)} = \Tr\left(\rho\hat{\Psi}^\dagger\hat{\Psi}\right),
	\end{equation}
	whose eigenvalues $p_{l}^{1}$ give the occupation probability of the $l^{\rm th}$ eigenvector of $\rho^{(1)}$. The eigenvectors themselves are the single-particle modes. If any eigenvalue $p_{l}^{i}$ is proportional to $N$ in the limit $N\rightarrow\infty$, then the system is said to have undergone Bose-Einstein condensation (or, simply, \emph{condensed}) into the $l^\textrm{th}$ mode. 

	Of course, real systems are subject to atom losses and heating, violating the assumptions of equilibrium underpinning both the approaches above. Nonetheless, the Penrose-Onsager criterion was found to be valid in a non-Hermitian polariton condensate \cite{manni12}. Indeed, there are compelling reasons to call the experimentally realized ultracold gases `Bose-Einstein condensates' without any qualms. As the saying goes, if it interferes like a condensate \cite{andrews97} , undergoes number fluctuations like a condensate \cite{kristensen19},  has HBT correlations like a condensate \cite{schellekens05,jeltes07}, Kibble-Zureks like a condensate \cite{anquez16}, and quacks like a condensate \cite{duck01}, then it probably \emph{is} a condensate. \todo{I don't quite follow your arguments here... are you saying dilute gases may not be BECs, or that Penrose-Onsager theory may not be valid, or both?}

	While most atomic condensates, and all of those in this thesis, are trapped in non-uniform potentials, many important features of condensates are easier to state for homogeneous systems. One can usually extend calculations to harmonically trapped systems by a local density approximation, wherein one performs a density-weighted average across a condensate, considering each volume element as a homogenous condensate in its own right. Thus, for the most part the following discussion will focus on homogeneous systems for simplicity's sake. I present some particular results in the case of a harmonically trapped gas at the end of this section.

	% distinction worth making; the ground state is a global property of the configuration. Interactions mean it is not the same as the product of single-particle ground states: I mean, it *is* actually in this case, I think, but the ground states are not solutions of a free hamiltonian (I guess they are in the bogo pictures though??)


	% Potential for historical note about liquid helium
	
	%  Potential resolution; Mastsubara (1955) A new approach to quantum-statistical mechanics, Progress of Theoretical Physics vol 14, no 4
	% Many characteristic features predicted for Bose-Einstein condensates have been observed in ultracold trapped gases. 
	% Something something grand-canonical ensemble; role of chemical potential; tracing out reservoir
	% We have the other formlation, \rho = \exp(\beta \hat{H});
\subsection*{Bogoliubov theory}
	The fundamental theoretical object of interest is the Hamiltonian of a bosonic quantum field with two-body interactions,
	\begin{equation}
		\hat{H} \int\left(\frac{\hbar^2}{2m}\nabla\hat{\Psi}^\dagger(\textbf{r})\nabla\hat{\Psi}(\textbf{r})\right)d\textbf{r} + \frac{1}{2}\int\left(\hat{\Psi}^\dagger(\textbf{r}')\hat{\Psi}^\dagger(\textbf{r})V(\textbf{r}'-\textbf{r}) \hat{\Psi}(\textbf{r}')\hat{\Psi}(\textbf{r})\right)d\textbf{r}'d\textbf{r}
		\label{eqn:ham}
	\end{equation}
	where $\Psi(r)$ are the field operators subject to the bosonic commutation relations
	\begin{align}
		[\Psi(r),\Psi^\dagger(r')] &= \delta(r-r')\\
		 [\Psi^\dagger(r),\Psi^\dagger(r')]&=[\Psi(r),\Psi(r)]=0.
	\end{align}	
	We can then write the field operator in the suggestive form	
	\begin{align}
		\hat{\Psi} &= \psi_0 \hat{a}_0 + \sum_{i\neq0}\psi_i \hat{a}_i
	\end{align}
	in terms of an orthonormal basis of single-particle modes $\phi_i$ and corresponding field operators $\hat{a}_i$. In doing so we distinguish $\pvec=0$ as the condensed mode, and say that condensation occurs when $N_0=\langle\hat{a}^\dagger_0\hat{a}_0\rangle \propto N$ \todo{should this be an approx sign rather than a proportional sign?  Doesn't really make sense to me what you are saying as written}. The observation that the condensed mode has a population of order $N$ means that in the thermodynamic limit ($N\rightarrow\infty,~V\rightarrow\infty$), one particle here or there will not really make a measurable difference. This heuristic can be expressed quantitatively as the Bogoliubov approximation wherein the annihilation and creation operators for the condensed mode are replaced with complex numbers,
	\begin{equation}
		\hat{a}_0 = \sqrt{N_0}e^{i\alpha}, \hat{a}_0^\dagger= \sqrt{N_0}e^{-i\alpha}, 
	\end{equation}
	\todo{I don't quite understand why this follows from replacing the operators with complex numbers}
	which permits the condensate wavefunction to take the form
	\begin{align}
		\hat{\Psi} &= \sqrt{N_0}e^{i\alpha} \psi_0 + \delta\hat{\Psi}\\
					&= \Psi_0 + \delta\hat{\Psi}
	\end{align}


	% Where  $\sqrt{N_0}e^{i\alpha}$ is the \emph{order parameter} which gives the amplitude of the condensate wavefunction and $\psi_0$ is some self-consitent defn of the condensed mode, however one gels the atom-density-matrix defn with the bosonic field... 

	% usuall The eigenfunctions of the non-interacting(?) case can be used to express the field operator $\Psi(r) = \sum_i \phi(i) \hat{a}_i$ in terms of the creation operator, which obeys similar commutation relations. This then leads to a similar separation of the field operator into the condensed and noncondensed part,
	% $$
	% \Psi(r) = \phi_0\hat{a}_0 + \sum_{i\neq0}\phi_r(r)\hat{a}_i,
	% $$
	% Does bogo assume $[\hat{a}_o,\hat{a}^{\dagger}_0] = 0$? 
	% form is then $\Psi(r) = \Psi_0(r) + \delta\Psi(r)$ where the first term on RHS is the complex fn $\sqrt{N_0}\phi_0$ and the last term is a sum over other modes. 
	%%See also; assuming \Psi = \phi + \hat{\psi} with \int|phi|^2 >> \langle\psi^\dagger\psi\rangle permits discarding third-and-higher order terms in the full bosonic field hamiltonian, leading to GPE... how relates to Born approx?
	The first term is the condensate wavefunction, and the second corresponds to the population of non-condensed modes thanks to the effect of interactions, which are captured by the quasiparticle picture sketched in the next section.	

	The emergence of a condensate has many of the hallmarks of a classical phase transition: kinetic effects are necessary to redistribute energy and reach steady-state\footnote{\todo{Bose enhancement re: scattering into the	ground state. }}; a unique critical temperature $T_c$ exists; below $T_c$ an \emph{order parameter} takes on a nonzero value; and condensation is equivalent to the spontaneous breaking of a U(1) gauge symmetry \cite{yukalov11_symmetry}. Above the critical temperature, $|\Psi_0|=0$, and in general it exhibits a discontinuous derivative at the critical temperature. Hence, in the Landau-Ginzburg framework, condensation is a second-order phase transition\footnote{In the Ehrenfest picture one is instead concerned with the number of times one must differentiate some state function (e.g. specific heat, compressibility, pressure, free energy) before finding a discontinuity at the critical point. In this picture, the transition is first-order as one has continuous state functions with discontinuous derivatives.}. The other hallmark of Landau-Ginzburg phase transitions is the spontaneous breaking of symmetry as one crosses from the disordered to the ordered phase (as when a solid breaks the translational symmetry of the fluid phase). Condensates do exhibit such symmetry breaking: The Hamiltonian has a $U(1)$ gauge symmetry, but the ground state of a condensate spontaneously chooses a fixed but unpredictable phase $\alpha$. By interfering two independently prepared condensates, one observes interference fringes \cite{andrews97}, and indeed, the fringe locations will change with each realization and measurement. More directly, one can interfere light leakage from a reservoir-coupled photon condensate against a reference beam, and observe that phase jumps occur in the output when the condensate field drops to zero. That is, the re-emergence of the condensate is heralded by the selection of a new, specific, phase, apparently uncorrelated with the phase that existed before it \cite{schmitt16}. 
	
	Symmetry breaking is a subtle point discussed infrequently in standard textbooks. It happens that one can substitute complex numbers for the field operators,  even when $\langle N_0\rangle \rightarrow 0$ and still obtain correct results \cite{ginibre67}: Condensation is not necessary for the Bogoliubov approximation to be valid. However, it \emph{is} the case that the onset of condensation coincides with the ground state breaking the $U(1)$ symmetry \cite{suto05} \footnote{This is sometimes called `breaking the gauge symmetry'. This is a misnomer, as a gauge symmetry is a property of the theory, not a state, and all consistent theories must be gauge symmetric throughout. These subtleties are discussed at length in lucid terms in \cite{poniatowski19}}. That this coincides with the validity of the `coherent state' approximation is quite profound: Like the transition from the quantized electromagnetic field to the Maxwell equations, BEC has the flavour of a classical field but one which is consitued by matter which has dispersed in some delocalized state. The interactions between atoms present a marked departure from the physics of the electric field, however, and yet the system remains in the grasp of human theorists thanks to the Bogoliubov approximation, which I introduce here and examine experimentally in chapter \ref{chap:QD}.\todo{This paragraph (and several others in your thesis) has a couple of extremely long sentences involving colons.  I'd break them into two independent sentences to aid readability. Probably just delete the stuff after the colon.}
	%  First-order phase transition as the chemical potential has a discontinuous derivative \cite{ykalovreview}, cf ehrenfest critioerion of discontinuous derivative of a state function; the ginzburg-landau classification is in terms of the order parameters; BEC is a second-order transition as the order parameter is continuous with a discontinuous derivative.
	% Penrose and Onsager generalized the definition of condensation to include interacting particles and non-uniform systems.  In fact, the Bogoliubov picture of superfluidity actually refers to a condensation of non-interacting collective bosonic modes, rather than the condensation of the constituent atoms. The definitions coincide fairly well.../
	% NB most BEC theory most simply stated for homogeneous systems, so will consider these initially, and return to a few remarks about harmonically trapped gases at the end of the section.


	% Actually in the following, one starts from the full second quantized hamiltonian and assumes smooth density in the born approx to get to the  GPE. The variational approach is just an alternative and need not be included. 
	
	The Hamiltonian \ref{eqn:ham} has an alternative presentation in second-quantized form,
	\begin{equation}
		\hat{H} = \sum_\pvec \frac{p^2}{2m} a^\dagger_\pvec a_\pvec +\sum_{\pvec,\textbf{p},\textbf{q}} \frac{\nu(\pvec)}{V}a^\dagger_{\pvec+\textbf{p}} a^\dagger_{\textbf{q}-\textbf{p}} a_\pvec a_\textbf{q}, 
	\end{equation}
	\todo{double p in summation term}
	where the $\hat{a}^\dagger_\pvec$  ($\hat{a}_\pvec$) operators create (annihilate) a bosonic atom in the state with momentum $\pvec$. The first sum is the kinetic term, and the second captures momentum-conserving collisions, where $\nu(\pvec)$ is the Fourier transform of the scattering potential $V(\textbf{r}'-\textbf{r})$ and the sum omits terms which do not conserve momentum. 

	The Fourier conjugate of a contact interaction of the form $V(\textbf{r}'-\textbf{r})=g\delta_{\textbf{r},\textbf{r}'}$ is $\nu(\kvec)=g/2$, which appears 	following a lengthy calculation (see, for example, \cite{PitaevskiiStringari}) in the symmetric quadratic form
	% Also, there's the issue of identifying the reservoir... is it identified with the thermal fraction, or some other system?
	\begin{equation}
		H = \frac{gN^2}{2V} + \sum_{\pvec\neq0}\frac{p^2}{2m}a^\dagger_\pvec a_\pvec + \frac{gn}{2}\sum_{\pvec\neq0}\left(2a^\dagger_\pvec a_\pvec + a^\dagger_\pvec a^\dagger_{-\pvec} + a^\dagger_{-\pvec} a^\dagger_{\pvec} + \frac{mgn}{p^2}\right),
		\label{eqn:bog_H}
	\end{equation}
	which is specified by $g=4\pi\hbar^2a/m$, and is visibly diagonalized in the non-interacting case, where $g=0$. The off-diagonal scattering terms can be eliminated by following the Bogoliubov approach (which was actually first employed by Holstein and Primakoff in the context of spin waves \cite{schwabl} and then independently applied to the problem of interacting bosons by Bogoliubov \cite{bologiubov47}). The Bogoliubov transform amounts to the substitution of operators
	% bogoliubov transform to diagonalize this H, bogo dispersion, population graphs inc depletion from LDA
	% From Schwabl:
	% Bogo transform first used in T. Holstein and H. Primakoff (Phys.
	% Rev. 58, 1098 (1940) for spin-wave hamiltonians. Bogo rediscovered and got the name
	\begin{align}
		a_\pvec &= u_\pvec \hat{b}_\pvec + v_\pvec \hat{b}^\dagger_{-\pvec}\\
		a_\pvec^\dagger &= u_\pvec\hat{b}_\pvec^\dagger + v_\pvec \hat{b}_{-\pvec}
		\label{bogotrans}
	\end{align}
	whose inverse
	\begin{align}
		\hat{b}_\pvec &= u_\pvec a_\pvec - v_\pvec a^\dagger_{-\pvec}\\
		\hat{b}_\pvec^\dagger &= u_\pvec a_\pvec^\dagger - v_\pvec a_{-\pvec}
		\label{bogoinverse}
	\end{align}
	make clear their physical interpretation: the $\hat{b_\pvec}$ operators act on the subspaces corresponding to \emph{collective excitations} which are made up of superpositions of oppositely-moving single particles. This correspondence has been directly observed in experiments \cite{vogels02} using absorption imaging techniques; an unrealized goal in cold-atom science is the observation of this single-particle decomposition at the level of single atoms \todo{How would you observe this?  What would it look like?}. More will be said about the challenges of this goal in chapter Y. 
	
	The quasiparticles are pairs of bosons, whose total spin will also be an integer, and thus the $\hat{b}$ should obey the bosonic commutatation relations. This constrains the coefficients such that $|u_\pvec|^2-|v_{-\pvec}|^2=1$, permitting the parametrization $u_\pvec = \cosh \alpha_\pvec$, $v_{-\pvec} = \sinh \alpha_\pvec$. One picks $\alpha_\pvec$ such that the off-diagonal terms in the transformed Hamiltonian vanish, which provides the form 
	\begin{equation}
		u_\pvec,v_{-\pvec} = \pm \sqrt{\frac{p^2/2m+gn}{2\epsilon(p)} \pm \frac{1}{2}},
	\end{equation}
	in terms of the Bogoliubov dispersion relation
	\begin{equation}
		\epsilon(p) = \sqrt{\frac{gn}{m}p^2+\left(\frac{p^2}{2m}\right)^2}.
	\end{equation}
	One could rewrite $\epsilon(p)$ in terms of the speed of sound $c=\sqrt{gn/m}$ \todo{use another symbol to avoid overloading with speed of lightf} to obtain the alternative form $\epsilon(p) = \sqrt{p^2c^2+\left(\frac{p^2}{2m}\right)^2}$, making clear the extreme behaviours of the excitations: At low momentum, they are wavelike phonons with $\epsilon\approx pc$ and at high momentum they are effectively free particles with $\epsilon\approx p^2/2m$, with the smooth intermediate region determined by the effective coupling strength $g$.

	The population of the single-particle states can be calculated by substituting the quasiparticle field operators into the standard expectation value, giving
	\begin{align}
		n(\pvec) &= \langle\hat{a}^\dagger_\pvec \hat{a}_\pvec\rangle\\
		&= \bra{\Psi}(u_\pvec\hat{b}_\pvec^\dagger + v_\pvec \hat{b}_{-\pvec})(u_\pvec \hat{b}_\pvec + v_\pvec \hat{b}^\dagger_{-\pvec})\ket{\Psi}\\
		&=\bra{\Psi}(u_\pvec^2\hat{b}_\pvec^\dagger \hat{b}_\pvec  + v_\pvec^2 \hat{b}_{-\pvec} \hat{b}^\dagger_{-\pvec} + u_\pvec v_\pvec \left(\hat{b}_{-\pvec} \hat{b}_\pvec + \hat{b}_\pvec^\dagger  \hat{b}^\dagger_{-\pvec}\right))\ket{\Psi}.
	\end{align}
	In the steady-state, the expected value of the mode populations is constant and hence the third term is zero because it does not conserve particle number. It follows from the bosonic commutation relations that
	\begin{align}
		\bra{\Psi} \hat{b}_{-\pvec} \hat{b}^\dagger_{-\pvec}\ket{\Psi} &= \bra{\Psi} 1 + \hat{b}_\pvec^\dagger \hat{b}_\pvec\ket{\Psi}\\
								&= \bra{\Psi}\Psi\rangle  + \bra{\Psi}\hat{b}^\dagger \hat{b}\ket{\Psi}
	\end{align}
	hence the second term survives and allows the population to be written as
	\begin{equation}
		n(\pvec) = (u_\pvec^2+v_\pvec^2)\langle\hat{b}_\pvec^\dagger \hat{b}_\pvec\rangle + v_\pvec^2,
	\end{equation}
	Where the quasiparticle population can be computed from thire Bose-Einstein statistics as
	\begin{equation}
		\langle\hat{b}^\dagger_\pvec \hat{b}_\pvec\rangle = \frac{1}{e^{\beta \epsilon(p)}-1},
	\end{equation}
	and the corresponding population of $\pvec\neq0$ single-particle modes is the thermal population. The $v_\pvec^2$ term persists even at $T=0$ and, owing to its nature as the zero-point energy of the quasiparticle vacuum, gives rise to a population outside the condensed mode known as the \emph{quantum depletion}, which is the focus of chapter Y.

	% The experimental measurement of this feature of the condensate wavefunction is the subject of chapter Y, and more details are presented there. 

	% We can find the single-particle population directly from the bogo transformation. We wind up with
	% $<n_p> = <a+_p a_p> = |v_-p|^2 + |u_p|^2<b+_p b_p> + |v_-p|^2<b+_-p b_-p>$
	
	% Then one gets $n_p = (p^2/2m + mc^2)/(2 eps(p)) - 1/2$ 
	% which goes like $m^2c^2/p^4$ at large p (???)
	% This can be used to calculate the number of particles outside the condensate (schwabl 3.2.22-23) and also the momentum distribution
	% The energy of large-k modes is just the free-particle energy plus the mean-field potential
	
	% See Pitaevskii chap 4 for more details, including means to include short-wavelength corrections that mitigate an ultraviolet divergence.

	% For discussions of how the Bogoliubov picture connects condensation to superfluidity, see Bogolubov47 or Schwabl

	% One can extend these to non-uniform gases by a local density approximation, as is done later in this thesis  (either in QD chapter or as latter part of TF section.

	\todo{given the main need for the Bogoliubov theory is for QD, you may be better off moving some of the previous few sections to the QD chapter for clarity? }

\subsection*{Harmonically trapped condensates}

	Turning back toward harmonic gases, we can begin from the full Hamiltonian (Eqn. (\ref{eqn:ham}) and produce an effective Schr\"{o}dinger equation, by assuming slow variations in the density, integrating out short-wavelength modes as in the Born approximation and arriving at the Gross-Pitaevskii equation (GPE)\todo{more specificity re: mean field assumption},
	\begin{equation}
		-\hbar\partial_t\Psi(r,t) = \left(-\frac{\hbar^2\nabla^2}{2m} + V(r,t) + g|\Psi(r,t)|^2\right)\Psi(r,t).
		\label{eqn:GPE}
	\end{equation}
	which is valid for arbitary interactions dominated by the s-wave scattering length. If one further assumes that the condensate density varies on scales larger than the healing length $\xi = \hbar/\sqrt{2mgn}$, one can make the Thomas-Fermi approximation and ignore the kinetic term in the GPE, whereby the condensate density profile can be written as

	% in the Thomas-Fermi approximation, whereby one assumes  the density varies 	he GPE is a common starting point for numerical simulations of BEC physics, but the salient feature for our purposes is the introduction of the effective interaction energy $g=4\pi\hbar^2a/m$. The importance can be seen by noting that 
	\begin{equation}
		n(\textbf{r}) = \frac{\mu-V(\textbf{r})}{g},
	\end{equation}
	where $\mu$ is the chemical potential. 
	For a Bose gas confined in a harmonic potential of the form
	\begin{equation}
		V = \frac{1}{2} m \omega_x^2 x^2 + \frac{1}{2} m \omega_y^2 y^2 + \frac{1}{2} m \omega_z^2 z^2,
	\end{equation}
	this produces the famous inverted-parabola density profile,	for which the peak density of the condensate is
	\begin{equation}
		n_0 = \frac{1}{8 \pi}\left( (15N_0)^2 \left(\frac{m \bar{\omega}}{\sqrt{a_{1,1}} \hbar}\right)	 ^{6}\right)^{1/5}.
		\label{eqn:n0}
	\end{equation}
	The Thomas-Fermi approximation is valid when $N a/a_{ho}\gg1$, where $a_{ho} = \sqrt{\hbar/m\omega}$ is the length scale of the ground state of the harmonic oscillator. The phase space density for condensation is achieved at the (ideal) critical temperature 
	\begin{align}
		T_c^{0} &= \frac{\hbar \bar{\omega}}{k_B}\left(\frac{N}{\zeta(3)}\right)^{1/3}\\
				&\approx0.94\frac{\hbar \bar{\omega} N^{1/3}}{k_B}.
	\end{align}
	where $\bar{\omega}=(\omega_x\omega_y\omega_z)^{1/3}$ is the geometric mean of the trapping frequencies. The condensed fraction below the critical temperature is 
	\begin{equation}
		\frac{N_0}{N} = 1 - \left(\frac{T}{T_c^{0}}\right)^3,
	\end{equation}
	Interactions between particles reduce the critical temperature in a harmonic trap, and the resulting shift in the critical temperature can be written as 
	\begin{equation}
		\frac{\delta T_c}{T_c^{0}} = -1.3 \frac{a}{a_{ho}} N^{1/6} -0.73\frac{ \langle\omega\rangle}{\omega_{ho} N^{1/3}}
	\end{equation}
	where the latter term, the correction for finite atom number, includes the arithmetic mean trapping frequency $\langle\omega\rangle$ and vanishes in the thermodynamic limit $(N\rightarrow\infty,~V\rightarrow\infty,~N/V$ finite). This effect can be understood in terms of the reduction in density due to repulsive interactions, thus driving down the phase space density $n\lambda^3$ at a given temperature, lowering the temperature required for condensation. These deviations from ideal behaviour have been observed in experiments \cite{tammuz11,smith11}.
	\todo{what is the critical density?}

	% For an ideal gas, this is the expected fraction of the population in the ground state of the trap. The rest of the population, occupying excited single-particle states, is called the thermal fraction. The population $N_T$ of the thermal fraction saturates as $\frac{N_T} = \left(\frac{k_B T}{\hbar \bar{\omega}}\right)^3$, wherein any atoms added to the gas fall into the condensate mode, regardless of the density of the gas. Real gases generally don't show such a feature, and this bifurcation is corrected by including the effect of interactions on the thermodynamics of the Bose gas.	 

	The condition $\mu-V(\textbf{r})=0$ determines the boundary of the condensate, giving the Thomas-Fermi radii $R_i = \sqrt{2\mu/m \omega_i^2}$, where $i\in\{x,y,z\}$. The chemical potential for the harmonically trapped condensate also fixes the average energy per particle:
	\begin{equation}
		\mu = \frac{\hbar\bar{\omega}}{2}\left(\frac{15 N a}{a_{ho}}\right)^{2/5} = \frac{7}{5}\frac{E}{N}
	\end{equation}

	As for the practical production of condensates, much quality literature has been written on the theory and techniques of atomic cooling technqiques employed to reach degeneracy. These days, such techniques are standard across hundreds of laboratories and so space will not be spared for their general consideration here. The curious reader is directed to \cite{MakingProbingUnderstanding,courtielle01,MetVdS, tychkovThesis} for detailed discussions.  Rather, in the next chapter I discuss the specifications of the apparatus used in the course of this research to implement the general principles of atomic cooling and trapping to achieve condensation.
	
	\todo{The last couple of pages I feel you could add a little more insight in terms of simple physical pictures of what is going on behind some of the maths.  What you have isn't wrong and is perfectly fine as is, but a couple of simple physical insights can add a lot. }

	% \subsubsection{Producing BEC}

	% Ketterle96 evaporative cooling of trapped atoms

	% The classic result in QM gives a velcity field
	% $v(r,t) = \hbar/m \nabla S(r,t)$ where S is the phase of the order parameter; 

	% The momentum distribution can be obtained from the Fourier transform of the TF density but the expansion dynamics affect the profile...
	% during our experiments they expand in freefall to an	ellipsoidal volume of XXX - I wonder what the largest coherent volume is	otherwise? 

	% time-dependent stuff and expansion profiles;
	% from Castin \& Dum
	% After release from the trap, the mean-field interaction disperses into kinetic energy of the particles, contravening the assumptions of the Thomas-Fermi approximation. The subsequent expansion from a harmonic trap can be characterized by a scaling $R_i(t) = \lamba_i(t)R_i(0)$, where $R_i(t)$ is the position of an infinitestimal volume of the cloud at time $t$ in the $i$th coordinate. The scaling parameters $\lambda$ evolve as per the coupled second-order ODEs
	% \begin{equation}
	% 	\ddot{\lambda_j} = \frac{\omega_j(0)^2}{\lambda_j\lambda_1\lambda_2\lambda_3}-\omega_j(t)\lambda_j
	% \end{equation}

	% In this generalization of TF to the time-dependent setting, the condensate density evolves like
	% \begin{equation}
	% 	N|\Phi_{TF}(r,t)|^2 = \frac{\mu-\sum_j \frac{m\omega_j^2(0)r_j^2}{\lambda_j^2(t)}}{g\lambda_1\lambda_2\lambda_3},
	% \end{equation}

	% In the case of an instantaneous switching-off of a cigar-shaped trap ($\omega_3=\omega_x\gg\omega_1=\omega_2=\omega_\perp), the scaling parameters evolve like 
	% \begin{align}
	% 	\frac{d^2}{d\tau^2} \lamba_\perp &= \frac{1}{\lamba_\perp^3\lambda_z}\\
	% 	\frac{d^2}{d\tau^2} \lamba_x &= \frac{\varepsilon^2}{\lamba_\perp^2\lambda_x^2},
	% \end{align}
	% where $\varepsilon = \omega_x(0)/\omega_\perp(0)\ll1$. These equations are amenable to numerical integration, but the aspect ratio of an expanded condensate can be written in a relatively straightforward manner as
	% \begin{align}
	% 	\frac{W_x(t)}{W_\perp(t)} = \frac{\lambda_z(t)}{\lambda_\perp(t)}\frac{1}{\varepsilon}.
	% \end{align}
	% hence, if one has prior knowledge of the trapping frequencies, one can use the aspect ratio as measured at a single point in time $t$ to determine the scaling parameters from a low-order series expansion in the form
	% \begin{align}
	% 	\lambda_\perp(t) = \sqrt{1+(\omega_\perp(0)t)^2}\\
	% 	\lambda_x(t) = 1+\varepsilon^2(\omega_\perp(0)t) \arctan(\omega_\perp(0)t)) - \ln\sqrt{1+(\omega_\perp(0)t)^2}).
	% \end{align}


	% from Tychkov
	



% \begin{equation}
	% Z(\beta,\mu) = \sum_{N'=0}^{\infty} e(\beta\mu N') Q_{N'}(\beta)
	% \end{equation}
	% where $Q_{N'}$ is the canonical partitionfunction for $N'$ particles. This is assuming that P(System|Reservoir) = P(system)P(reservoir) ie states of sys and res are independent. One gets the thermodynamic potentials
	% \begin{equation}
	% \Omega = E - TS - \mu N = - k_B T \ln Z
	% \end{equation}

	% Consider now a system described by single-particle hamiltonians $\hat{H} - \sum_i\hat{H}^{(1)}_i$, then the schrodinger equation $H^{(1)}\phi_i(r) = \varepsilon_i\phi_i(r)$ specifies the many-body eigenstate. 
	% In the idal bose gas one can find $\Omega = k_B T\sum_i \ln(1-e^{\beta(\mu-\varepsilon_i)})$ which then gives the number of particles via 

	% \begin{equation}
	% N = -\frac{\partial \Omega}{\partial\mu},
	% \end{equation}

	% or in terms of the occupation numbers (what's the general principle here? something something conjugates)
	% \begin{equation}
	% \bar{n}_i = -\frac{\partial Z}{\partial \beta \varepsilon_i} = \frac{1}{\exp(\beta(\varepsilon_i-\mu))-1}
	% \end{equation}
	% which is the famous Bose-Einstein statistics. One must have $\mu<\varepsilon_0$ which would otherwise entail a negative number of particles in the ground state. In the limit $\mu\rightarrow\varepsilon_0$ the condensed number $N_0=\bar{n}_0$ grows rapidly. 


	% Which contains the one-body density $n(r) = n^{(1)}(r,r)$ on the diagonal, where $\int dr n(r) = N$. One can also work in the Fourier basis via $\Psi(p) = (2\pi\hbar)^{-3/2}\int dr exp(-ip\cdot r/\hbar)\Psi(r)$. 
	
	% Below the critcal temperature the density has the form $n(p) = N_0\delta(p) + \tilde{n}(p)$, where $N_0$ is the condensed number. This means that one has $n^{1}(s)\rightarrow n_0=N_0/V$ which involves offdiagonal parts of the one body density.
	% If we have the single-particle eigenfunctions $\phi_i$ then we can also write $n^{1}(r,r') = \sum_i n_i \phi_{i}^*(r)\phi_{i}(r')$ where $n_i$ are the occupation numbers of the single particle states. One can conveniently separate the condensed and noncondensed parts 
	% $n^{1}(r,r') = N_0 \psi_0^*(r)\phi_0(r') + \sum_{i\neq0} n_i \phi_{i}^*(r)\phi_{i}(r')$
	% wherein the latter term is replaced by an integral in the thermodynamic limit which vanishes at infinity (incoherence??)

