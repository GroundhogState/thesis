\chapter*{Abstract}
% An abstract of 250-500 words is on the page or pages following the acknowledgement. The electronic version of the abstract must use standard text only.
Modern physics is rooted in the reductionist pursuit of resolving ever finer atomic objects and understanding how their interrelations give rise to all other phenomena. The word `atomic' is now overloaded; the supposedly irreducible elements have since been found to be composed of finer grains, the quarks and electrons. Nonetheless, composite atoms continue to provide testing grounds for fundamental physics and means for engineering exotic and, perhaps, useful phenomena. This dissertation presents work in each of these categories, made possible by the versatility of quantum gas experiments. Dilute gases at ultracold temperatures in precisely controllable environments provide ground for state-of-the-art measurements of atomic structure which are competitive with the cutting edge of theoretical capabilities, and fine control over collective effects useful for understanding the emergence and characteristics of collective behaviour. This dissertation first describes laser spectroscopic measurements of helium structure, including the first direct measurement of two ultraweak transitions in the helium atom and a tune-out wavelength with sufficient accuracy to compare with the latest predictions calculated in theory of quantum electrodynamics. The second part of this thesis concerns methods of exploring many-body effects in ultracold helium. The first finding in this part concerns the applicability of time-of-flight detection to the investigation of collective effects in weakly interacting gases. This dissertation closes with a discussion of the opportunities in the emerging field of quantum simulation which could be explored using the unique detection methods available for ultracold helium, and a documentation of progress towards reaching this goal.


\newpage

\section*{Works discussed in this thesis}


% Title
% Author
% Journal
% Year
% Link
% citekey
% Comment
\newcommand{\fullpaper}[5]{\normalsize {{\bf#1}\newline #2, \IfNoValueTF{#5}{\emph{#3}}{\href{#5}{\emph{#3}}} (#4) }}
\newcommand{\citem}[1]{\item[\cite{#1}]}

\begin{itemize}

	\citem{Abbas21} \fullpaper{Rapid generation of metastable helium Bose-Einstein condensates}{A. H. Abbas, X. Meng, R. S. Patil, \underline{J. A. Ross}, A. G. Truscott, S. S. Hodgman}{Physical Review A}{2021}{https://journals.aps.org/pra/abstract/10.1103/PhysRevA.103.053317}

	\citem{Ross20} \fullpaper{Frequency measurements of transitions from the $2\triplet P_2$ state to the $5\singlet D_2$, $5\triplet S_1$, and $5\triplet D$ states in ultracold helium}{\underline{J. A. Ross}, K. F. Thomas, B. M. Henson, D. Cocks, K. G. H. Baldwin, S. S. Hodgman, A. Truscott}{Physical Review A}{2020}{https://journals.aps.org/pra/abstract/10.1103/PhysRevA.102.042804}

	\citem{Thomas20} \fullpaper{Direct measurement of the forbidden $2\triplet S_1 - 3\triplet S_1$ atomic transition in helium}{K. F. Thomas, \underline{J. A. Ross}, B. M. Henson, D. K. Shin, K. G. H. Baldwin, S. S. Hodgman, A. G. Truscott}{Physical Review Letters}{2020}{https://journals.aps.org/prl/abstract/10.1103/PhysRevLett.125.013002}

	
	\citem{Ross21} \fullpaper{Survival of the quantum depletion of a condensate after release from a harmonic trap in theory and experiment}{\underline{J. A. Ross}, P. Deuar, D. K. Shin, K. F. Thomas, B. M. Henson, S. S. Hodgman, A. G. Truscott}{ArXiv}{2020}{https://arxiv.org/abs/2103.15283}

	\item \fullpaper{Precision measurement of the helium $2\triplet S_1 - 2\triplet P/3\triplet P$ tune-out frequency as a test of QED}{B. M. Henson, \underline{J. A. Ross}, K. F. Thomas, C. N. Kuhn, D. K. Shin, S. S. Hodgman, Y. H. Zhang, L. Y. Tang, G. W. F. Drake, A. T. Bondy, A. G. Truscott, K. G. H. Baldwin}{Undergoing peer review}{2021}{}
	
	\item \fullpaper{Trap frequency metrology with Bose-Einstein condensates}{B. M. Henson, \underline{J. A. Ross}, D. K. Shin, K. F. Thomas, S. S Hodgman, A. G. Truscott}{In progress}{2021}{}

\end{itemize}	

\section*{Other publications during the course of study}
\begin{itemize}
	\citem{Shin16} \fullpaper{Widely tunable, narrow linewidth external-cavity gain chip laser for spectroscopy between 1.0-1.1 $\mu$m}{D. K. Shin, B. M. Henson, R. I. Khakimov, \underline{J. A. Ross}, C. J. Dedman, S. S. Hodgman, K. G. H. Baldwin, A. G. Truscott}{Optics Express}{2016}{https://www.osapublishing.org/abstract.cfm?uri=oe-24-24-27403}

	\citem{Henson18_ML} \fullpaper{Approaching the adiabatic timescale with machine learning}{B. M. Henson, D. K. Shin, K. F. Thomas, \underline{J. A. Ross}, M. R. Hush, S. S. Hodgman, A. G. Truscott}{Proceedings of the National Academy of Science}{2018}{https://www.pnas.org/content/115/52/13216.short}

	\citem{Shin20} \fullpaper{Entanglement-based 3D magnetic gradiometry with an ultracold atomic scattering halo}{D. K. Shin, \underline{J. A. Ross}, B. M. Henson, S. S. Hodgman, A. G. Truscott}{New Journal of Physics}{2020}{https://iopscience.iop.org/article/10.1088/1367-2630/ab66de/meta}

\end{itemize}

\newpage