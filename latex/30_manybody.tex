% \hypertarget{many-body-physics}{%
% \section{Many-body physics}\label{many-body-physics}}
% \chapter*{Interplay, emergence, complexity}
% \addcontentsline{toc}{chapter}{Interplay, emergence, complexity}
% \section*{There's always a bigger Hilbert space}%stat phys, Thermalization, coherence, quantum matter, state engineering
% \addcontentsline{toc}{section}{There's always a bigger Hilbert space}
\subsection*{It to bit}% QMI/condmat, QML, church-turing, Physics as a branch of computer science, process theories?
\subsection*{Over the horizon}% Exotic matter, quantum simulation, is this more like string theory or thermodynamics? 

\begin{adjustwidth}{3cm}{0cm}
\begin{flushright}
\emph{``From the Tao comes one,\\
from one comes two,\\
from two comes three, and\\
from three comes all things."}\\
 - Lao Tzu\footnote{Tao Te Ching, circa 4th-6th century BCE}
\end{flushright}
\end{adjustwidth}
    


	% [Many body phys] I bloch, J Dalibard, W Zwerger many-body physics with ultracold gases, rev mod phys 80, 2008
	% [Many body phys/scattering] C Chin et al, feschbach resoannces in ultracold gases, rev mod phys 82, 2010

	What actually happens when you add more particles? What do we mean things get more complex - what resources are there more of in the state, and what does it take more of to simulate etc
		% https://journals.aps.org/rmp/abstract/10.1103/RevModPhys.82.1225 feshbach resonance review
Oh look, this might be handy https://arxiv.org/pdf/1711.04105.pdf
	Ingo Peschel shows that one can determine reduced density matrices from correlation functions of lattice systems \cite{Peschel03} %https://iopscience.iop.org/article/10.1088/0305-4470/36/14/101/fulltext/ 	(reduced) density matrices of solvable models have the form $\exp(-H)$, where $H$ is a solvable operator confined to the chosen subsystem (this is in the context of dmrg).
	You just integrate/trace out the degrees of freedom outside the subsystem.


	% Kibble-Zurek mechanism and criticality?
	% Pichler et al 2010 nonequlibrium dynamcics of bosonic atoms in optical lattices: Decoherence of many-body states dues to spontaneous emissions, phys rev a, atomic molecular and optical phscs 82
	% Sachdev QPT book
	% RSP [97,96] classical phase transitions are accompanied by non-analytic free energy (deriv.
	of partition fn?) and micro fluctuations (thermal) actually cause the transition 
	% RSP in contrast to Quantum phase transitions occur when tuning g in H = H0 + g H1, where [H0,H1]!=0 - and can occur at T=0, usuall refer to ground state transitions, dominant character of state changes due to eg avoided crossing? 
Section introduction\\
Statistical physics Phase transitions\\
Thermal, quantum, dynamical, computable, semantic Emergent complexity

The third section of this thesis transitions from the study of atomic
structure to the emergent dynamics of interacting systems.
	In teh old
essay `More is different', there is an argument that when you put a lot
of thigns together they start acting in genuinely novel ways.
	A single
water molecule is not wet - nor does it mean anything to say it is any
given phase, or that is has a temperature.
	(this is of course
problematic because it ascribes a specific state but the idea's there I
guess).
	So ya.
	One of the watershed(?) moments in the history of physics
was the \emph{derivation} of thermodynamics from statistical mechancics
- the assumption of a set of postulates about the microscopic nature of
the world that led through the law of large numbers to a new
understanding of empirical laws of the past; this gave a framework to
understand and extend thermo, and it was a triumph to provide systematic
ways to determine macro physics from teh nature of interactions.
	What
was more profouond was the discovery of universality classes - that at
phase boundaries there were unifying properties across disparate
systems, things that tied together quite general phenomena in terms of
these scaling relationships.
	This is kind of garbled.
	But yeah, look, we
have agaaghagha this is just insane rambling - I wonder if I can push to
10k words in the process of smooshing out all this.
	Not an entirely
honest drive if I'm honest but w/e the thing is I'm here and I'm writing
even if it's completely useless and will get trashed.
	This is a start,
even if it's bogus and hard.
	Anyhow back to progress.
	So ya -
statistical mechnics also provided an actual understanding of
temperature as a sublime phenomenon in all the emergent phenomenon -
that really, at the end of the day, thermal equlibrium and the
`spontaneous' processes we see in nature are just outcomes of,
basically, the law of large numbers.
	Phase transitions have been noted
in all kinds of places now; the BEC transitiion is a classical phase
transition.
	The idea is that an \emph{order parameter} changes value
from zero - representing a kind of disorganization - to something
nonzero, representing an emergent order.
	These come in different
flavours and in different systems but tied together by their scaling
laws - universality classes.
	Something really beautifully profound
there.
	And of course, pepole have tried to take the numinous idea of the
phase transition, of the \emph{qualitative} change in the emergent
properties of otherwise unchanged constituents, just by fiddling how
they relate to each other - and staple it to all kinds of things.
	People
talk of phase transitions in general network theory, presumably in
social dynamics, and more recently even in the study of grammars,
arguing that meaningful language corresponds to a marked phase
transition in lexical trees.
	But turning back from flights of fancy -
thermodynamics unified these ideas (in the earlier days) with the idea
of \emph{Free energy} which has since run rampant.
	But the idea is that
- well, if you have tunable parameters, then are there different
solutions to some equation? How do you wind up with these multiple
free-energy landscapes from a microscopic basis? Or do you need to
staple models toether? Idk, this would have been a great topics for a
PhD, hey?

Regardless.
	The idea is that we should probably add more signposts to
this meandering mess.
	I want to lead from classical phase transitions to
order parameters - using BEC as an example, identify its universality
class, no need to list off a ton or ramble about them too much.
	Then say
hey, once you're in the ground state, you have different kinds of phase
transitions.
	Quantum phase transitions, which aren't driven thermally or
by statistical laws but by the structure of the gruond state, and they
were first observed in an optical lattice.
	Free energy comes up as
should correlations, I think, as they both come hand in hand with order
etc.


Information In which the prospects of a quantum simulator are outlined.
	Beginning
with the discussion of a qubit and then the purity of states, the quantum analog
algorithms will be described.
	The necessity of state purity i.e.
	low entropy should
be discussed, and the question of realizing a high-purity quantum register should be
suggestively posed.
 Turing machines
 qubits
 Bloch sphere and evolution
 Purity
 Density matrix and the trace operator
 entanglement
 decomposition (partial traces, Schmidt product)
 second quantization
 coherence
 entropy
 renormalization
 thermodynamics
 thermalization
Matter In which the realization of the quantum analog algorithm is fixed in the form of
an optical lattice.
	The theory of atom-light interactions should be presented, and
the content of any field theories describing the composite system.
	Where possible,
connections should be drawn to the preceding chapter, and also to experimental
considerations which will be treated in the next chapter.
 Atomic structure
 metastable helium
 light and two-level atoms (qubits)
 Scattering and the radiative force
 Dipole force
 Sublevels
 Zeeman effect
vii
viii
 Angular momentum, polarization, and spin
 Composite systems
 Matter fields
 Correlation functions
 Strongly correlated systems
 Effective field theories: Bogoliubov and Bose-Hubbard
 Information revisited
 Comments on thermodynamics of cold atom ensembles
{ Breakdown of the equipartition theorem and kinetic temperature
{ Efficiency of atomic refrigeration
{ Quantum propagators, thermodynamics, and imaginary time
 Comments on classical, semiclassical, and quantum models
{ Breakdown of assumptions of continuity
{ Classical and quantum dipoles
{
Neural Networks In which any work done on representation of the quantum system is
decribed, along with precursors to any calculations done with these methods.
	In
this section, connections should be drawn between deep learning/state representation/
tensor networks/renormalization/state production/physical computation.
	It's
the dream.
	It's where it all comes together, but also the chapter that will be the
hardest or least likely to produce.
 Matrix product states
 MERA
 quantum circuit model
 RBM, DBM, and representations
 Deep learning and renormalization
 algorithms: tradeoffs and advantages
 categorical formulation
 Equivalence of circuits, ANN states and quantum states
 Church-Turing in the age of QNN and ANNQS
 Curry-Howard and state preparation

 Introduction
Perspective
 Ultra low temperature atomic physics
 Scenic Tour: Computing, quantum challenges
 Quantum simulators
 Optical lattice
Process of physical science
 Game of regularity: New understanding for free
 Triumph of reason: expressivity from compression
 Engineering requires correspondence: unpacking
 Quantifying computational effort
Computing and physics
Machine-assisted Phenomenology
 Convenient to outsource calculations
 How do you accurately predict the future?
 How much does it cost to predict the future?
Cost of computation
 Resolution/memory/time tradeoff
 RSA relies on problems being hard
Going Quantum
Simplifying matters
 Renormalization retains information while reducing dimensions
 Strongly correlated systems: modest many body systems impossible on
classical computers
1
Going Quantum
 One resolution to exponential memory demands
 universal quantum computer models ANY QUANTUM PROCESS
Quantum supremacy
Where's my QC?
 decoherence: arbitrary phase noise
 quantum errors are continuous: harder than bit 
ips
 Error correction threshold unreached, adds overhead
 Universal quantum computing and effciency
 Google's 50 qubits
Comments on complexity
 "Best possible classical algorithm"
 Sorry there's no clear cut definitions
 What is different about quantum computers? Open question.
Feynman's Big Idea
 System replication
 Run experiments on the model system and translate them back
 Move the goal posts: Solve (simulate) specific problems
 hope the simulation is "more effcient" than anything else: difference be-
tween impossble and achievable
Optical Lattices
Lattice overview
 Atom-light interactions (lasers) provide control
 Versatile: HEP, condensed matter, quantum information
 Approaching strongly-correlated regime
2
Quantum Phase Transitions
 Potential depth determines importance of interactions
 Eventually tunnelling suppressed by interactions: MI
 partial localization leads to interference
 MI phase incoherent
Simulations
 Also: Artificial graphene, non-thermal states, negative temperatures
 Role of entanglement in thermalization
 In the pursuit of effective representations of physical systems we have built
devices that create small-scale controllable simulations of the quantum
fabric of the universe.
	Cool, huh?
 Fundamental quantum mechanics necessary for quantum engineering
Quantum supremacy?
Research at ANU
ANU lattice
 Dipole force creates lattice
 Dipole barely overcomes gravity: Atoms must be ultracold to remain
trapped
 laser cooling: 3 nobel.
	BEC: 3 nobel.
 condensation: Macroscopic wavefunction
Conclusion
 High-resolution quantum dynamics in artifical crystal
 Such systems hard to simulate with existing computers
 No universal quantum computer, so specialized simulator
 Other groups: Have data out of reach of modern computing!
 Not known whether this is 'supreme'
 Analog simulators probably bridge the gap between classical sims and
universal quantum



    https://journals.aps.org/prl/abstract/10.1103/PhysRevLett.124.073401 




The Hamiltonian \ref{eqn:ham} has an alternative presentation in second-quantized form,
	\begin{equation}
		\hat{H} = \sum_\pvec \frac{p^2}{2m} a^\dagger_\pvec a_\pvec +\sum_{\pvec,\textbf{q}} \frac{\nu(\pvec)}{V}a^\dagger_{\pvec+\textbf{p}} a^\dagger_{\textbf{q}-\textbf{p}} a_\pvec a_\textbf{q}, 
	\end{equation}
	where the $\hat{a}^\dagger_\pvec$  ($\hat{a}_\pvec$) operators create (annihilate) a bosonic atom in the state with momentum $\pvec$.
	The first sum is the kinetic term, and the second captures momentum-conserving collisions, where $\nu(\pvec)$ is the Fourier transform of the scattering potential $V(\textbf{r}'-\textbf{r})$ and the sum omits terms which do not conserve momentum.
	

	The Fourier conjugate of a contact interaction of the form $V(\textbf{r}'-\textbf{r})=g\delta_{\textbf{r},\textbf{r}'}$ is $\nu(\kvec)=g/2$, which appears 	following a lengthy calculation (see, for example, \cite{PitaevskiiStringari}) in the symmetric quadratic form
	% Also, there's the issue of identifying the reservoir...
	is it identified with the thermal fraction, or some other system?
	\begin{equation}
		H = \frac{gN^2}{2V} + \sum_{\pvec\neq0}\frac{p^2}{2m}a^\dagger_\pvec a_\pvec + \frac{gn}{2}\sum_{\pvec\neq0}\left(2a^\dagger_\pvec a_\pvec + a^\dagger_\pvec a^\dagger_{-\pvec} + a^\dagger_{-\pvec} a^\dagger_{\pvec} + \frac{mgn}{p^2}\right),
		\label{eqn:bog_H}
	\end{equation}
	which is specified by $g=4\pi\hbar^2a/m$, and is visibly diagonalized in the non-interacting case, where $g=0$.
	The off-diagonal scattering terms can be eliminated by following the Bogoliubov approach (which was actually first employed by Holstein and Primakoff in the context of spin waves \cite{schwabl} and then independently applied to the problem of interacting bosons by Bogoliubov \cite{bologiubov47}).
	The Bogoliubov transform amounts to the substitution of operators
	% bogoliubov transform to diagonalize this H, bogo dispersion, population graphs inc depletion from LDA
	% From Schwabl:
	% Bogo transform first used in T.
	Holstein and H.
	Primakoff (Phys.
	% Rev.
	58, 1098 (1940) for spin-wave hamiltonians.
	Bogo rediscovered and got the name
	\begin{align}
		a_\pvec &= u_\pvec \hat{b}_\pvec + v_\pvec \hat{b}^\dagger_{-\pvec}\\
		a_\pvec^\dagger &= u_\pvec\hat{b}_\pvec^\dagger + v_\pvec \hat{b}_{-\pvec}
		\label{bogotrans}
	\end{align}
	whose inverse
	\begin{align}
		\hat{b}_\pvec &= u_\pvec a_\pvec - v_\pvec a^\dagger_{-\pvec}\\
		\hat{b}_\pvec^\dagger &= u_\pvec a_\pvec^\dagger - v_\pvec a_{-\pvec}
		\label{bogoinverse}
	\end{align}
	make clear their physical interpretation: the $\hat{b_\pvec}$ operators act on the subspaces corresponding to \emph{collective excitations} which are made up of superpositions of oppositely-moving single particles.
	This correspondence has been directly observed in experiments \cite{vogels02} using absorption imaging techniques; an unrealized goal in cold-atom science is the observation of this single-particle decomposition at the level of single atoms \todo{How would you observe this?  What would it look like?}.
	More will be said about the challenges of this goal in chapter Y.
	
	
	The quasiparticles are pairs of bosons, whose total spin will also be an integer, and thus the $\hat{b}$ should obey the bosonic commutation relations.
	This constrains the coefficients such that $|u_\pvec|^2-|v_{-\pvec}|^2=1$, permitting the parametrization $u_\pvec = \cosh \alpha_\pvec$, $v_{-\pvec} = \sinh \alpha_\pvec$.
	One picks $\alpha_\pvec$ such that the off-diagonal terms in the transformed Hamiltonian vanish, which provides the form 
	\begin{equation}
		u_\pvec,v_{-\pvec} = \pm \sqrt{\frac{p^2/2m+gn}{2\epsilon(p)} \pm \frac{1}{2}},
	\end{equation}
	in terms of the Bogoliubov dispersion relation
	\begin{equation}
		\epsilon(p) = \sqrt{\frac{gn}{m}p^2+\left(\frac{p^2}{2m}\right)^2}.
	\end{equation}
	One could rewrite $\epsilon(p)$ in terms of the speed of sound $\chi=\sqrt{gn/m}$ to obtain the alternative form $\epsilon(p) = \sqrt{p^2\chi^2+\left(\frac{p^2}{2m}\right)^2}$, making clear the extreme behaviours of the excitations: At low momentum, they are wavelike phonons with $\epsilon\approx p\chi$ and at high momentum they are effectively free particles with $\epsilon\approx p^2/2m$, with the smooth intermediate region determined by the effective coupling strength $g$.

	The population of the single-particle states can be calculated by substituting the quasiparticle field operators into the standard expectation value, giving
	\begin{align}
		n(\pvec) &= \langle\hat{a}^\dagger_\pvec \hat{a}_\pvec\rangle\\
		&= \bra{\Psi}(u_\pvec\hat{b}_\pvec^\dagger + v_\pvec \hat{b}_{-\pvec})(u_\pvec \hat{b}_\pvec + v_\pvec \hat{b}^\dagger_{-\pvec})\ket{\Psi}\\
		&=\bra{\Psi}(u_\pvec^2\hat{b}_\pvec^\dagger \hat{b}_\pvec  + v_\pvec^2 \hat{b}_{-\pvec} \hat{b}^\dagger_{-\pvec} + u_\pvec v_\pvec \left(\hat{b}_{-\pvec} \hat{b}_\pvec + \hat{b}_\pvec^\dagger  \hat{b}^\dagger_{-\pvec}\right))\ket{\Psi}.
	\end{align}
	In the steady-state, the expected value of the mode populations is constant and hence the third term is zero because it does not conserve particle number.
	It follows from the bosonic commutation relations that
	\begin{align}
		\bra{\Psi} \hat{b}_{-\pvec} \hat{b}^\dagger_{-\pvec}\ket{\Psi} &= \bra{\Psi} 1 + \hat{b}_\pvec^\dagger \hat{b}_\pvec\ket{\Psi}\\
								&= \bra{\Psi}\Psi\rangle  + \bra{\Psi}\hat{b}^\dagger \hat{b}\ket{\Psi}
	\end{align}
	hence the second term survives and allows the population to be written as
	\begin{equation}
		n(\pvec) = (u_\pvec^2+v_\pvec^2)\langle\hat{b}_\pvec^\dagger \hat{b}_\pvec\rangle + v_\pvec^2,
	\end{equation}
	Where the quasiparticle population can be computed from thire Bose-Einstein statistics as
	\begin{equation}
		\langle\hat{b}^\dagger_\pvec \hat{b}_\pvec\rangle = \frac{1}{e^{\beta \epsilon(p)}-1},
	\end{equation}
	and the corresponding population of $\pvec\neq0$ single-particle modes is the thermal population.
	The $v_\pvec^2$ term persists even at $T=0$ and, owing to its nature as the zero-point energy of the quasiparticle vacuum, gives rise to a population outside the condensed mode known as the \emph{quantum depletion}, which is the focus of chapter Y.

	% The experimental measurement of this feature of the condensate wavefunction is the subject of chapter Y, and more details are presented there.
	

	% We can find the single-particle population directly from the bogo transformation.
	We wind up with
	% $<n_p> = <a+_p a_p> = |v_-p|^2 + |u_p|^2<b+_p b_p> + |v_-p|^2<b+_-p b_-p>$
	
	% Then one gets $n_p = (p^2/2m + mc^2)/(2 eps(p)) - 1/2$ 
	% which goes like $m^2c^2/p^4$ at large p (???)
	% This can be used to calculate the number of particles outside the condensate (schwabl 3.2.22-23) and also the momentum distribution
	% The energy of large-k modes is just the free-particle energy plus the mean-field potential
	
	% See Pitaevskii chap 4 for more details, including means to include short-wavelength corrections that mitigate an ultraviolet divergence.

	% For discussions of how the Bogoliubov picture connects condensation to superfluidity, see Bogolubov47 or Schwabl

	% One can extend these to non-uniform gases by a local density approximation, as is done later in this thesis  (either in QD chapter or as latter part of TF section.