\documentclass{article}
\usepackage[utf8]{inputenc}
\usepackage{fullpage}
\usepackage[utf8]{inputenc}
\usepackage[margin=1in]{geometry}
\usepackage{graphicx}
\usepackage{amsmath}
\usepackage{amssymb}
\usepackage{verbatim}
\usepackage{mathtools}
\usepackage{rotating}
\usepackage{braket}
\usepackage{url}
\usepackage{tikz}
\usetikzlibrary{angles,quotes}
\usepackage{hyperref}
\newcommand{\bt}{\textbf{tr}}
\newcommand{\difn}[3]{\frac{d^#3 #1}{d #2^#3}}
\title{Derivation of the linear Tune-Out equation (linear form of the ac
Stark shift of a Zeeman sublevel)}
\author{Kieran Thomas}
\date{March 2019}

\newcommand{\pderivn}[3][{}]{
    \frac{\partial^{#1} #2}{\partial #3^{#1}}
}

\newcommand{\derivn}[3][{}]{
    \frac{d^{#1} #2}{d #3^{#1}}
}

\begin{document}

\maketitle
\section*{Dynamic Polarizability of Atoms in an Arbitrary light field}
Consider the interaction between an atom (in an external magnetic field \textbf{B} and Zeeman sublevel \(\ket{JM}\)) and a monochromatic classical light field given 
\begin{align}
    \textbf{E} &= \frac{1}{2} \mathcal{E} \textbf{u} e^{-i \omega t} \, + \, \text{c.c.},
\end{align}
where \(\omega\) is the angular frequency, \(\mathcal{E}\) is the electric field amplitude, and \textbf{u} the complex polarization vector. Note that only the real component of \textbf{E} has a physical meaning. For a strong enough magentic field (specifically when the off-diagonal terms are much smaller than the   Zeeman splitting of the sublevel, see \cite{} for more detail) the dynamic polarizability of the atom is:
\begin{align}
    \alpha(\omega) &= \alpha^s(\omega) + C \alpha^v(\omega) \frac{M}{2J} + D \alpha^T(\omega) \frac{3M^2-J(J+1)}{2J(2J-1)}
\end{align}
where \(\alpha^s\), \(\alpha^v\), and \(\alpha^T\) are the conventional scalar, vector, and tensor polarizabilities respectively. If we assume that the B-field is pointing along the \textit{z}-axis then the Coefficents \(C\) and \(D\) are given by:
\begin{align}
    C &= 2 \text{Im}(u_x^* u_y),\\
    D &= 3|u_z|^2 -1
\end{align}
We can furthermore define these constants in terms of experimentally measurable variables (see ... for derivations)
\begin{align}
     C &= - \mathcal{V} \cos \left( \theta_k \right), \label{eqn:C} \\
     D &= 3 \sin^2\left( \theta_k \right) \left(\frac{1}{2} +  \frac{\mathcal{Q}}{2}\right) -1 = 3 \sin^2\left( \theta_k \right) \left( \frac{1}{2} + \frac{1}{2} \frac{p_{max}-p_{min}}{p_{max}+p_{min}} \cos(2\theta_\varepsilon)\right) - 1 = 3 \sin^2(\theta_k) \mathcal{P}-1 \label{eqn:D}
\end{align}
where \(\mathcal{V}\) is the fourth stokes parameter, whose magnitude is given by \(|\mathcal{V}| = \frac{2\sqrt{p_{min}p_{max}}}{p_{min}+p_{max}}\) and sign depends on the handedness of the light, \(\mathcal{Q}=\frac{p_{max}-p_{min}}{p_{max}+p_{min}} \cos(2\theta_\varepsilon)\) is the second stokes parameter, and $\cos(\theta_\epsilon)=\text{Re}\left\{\textbf{u}\right\}\cdot(\textbf{B}\times \hat{k})$.

Near the tune-out the the dynamic \(\alpha(\omega)\) and scalar \(\alpha^s(\omega)\) terms can be approximated by their truncate linear Taylor expansions about their respective zero points (\(\omega_{to}\), \(\omega_{s,0}\)):
\begin{align}
    \alpha(\omega) &= \alpha(\omega_{to}) + \difn{\alpha(\omega)}{\omega}{{}}(\omega-\omega_{to})=\difn{\alpha(\omega)}{\omega}{{}}(\omega-\omega_{to}), \\
    \alpha^s(\omega) &= \alpha^s(\omega_{s,0}) + \difn{\alpha^s(\omega)}{\omega}{{}}(\omega-\omega_{s,0})= \difn{\alpha^s(\omega)}{\omega}{{}}(\omega-\omega_{s,0})
\end{align}
where \(\omega_{to}\) is the desired term. Note that for these approximations to be valid we need to only consider a small range about the respective zero values (\(\omega_{to}\), \(\omega_{s,0}\)), and as a consequence of this we must assume that the difference between these values is small. Substituting we obtain:
\begin{align}
    \difn{\alpha(\omega)}{\omega}{{}}(\omega-\omega_{to}) &= \difn{\alpha^s(\omega)}{\omega}{{}}(\omega-\omega_{s,0}) + C \alpha^v(\omega) \frac{M}{2J} + D \alpha^T(\omega) \frac{3M^2-J(J+1)}{2J(2J-1)}
\end{align}
The major contribution to the rate of change of the dynamic polarizability (\(\difn{\alpha(\omega)}{\omega}{{}}\)) is from the scalar polarizability (or equivalently that the vector and tensor terms are approximately constant over the range being considered), hence \(\difn{\alpha(\omega)}{\omega}{{}} \simeq \difn{\alpha^s(\omega)}{\omega}{{}}\). Rearranging and making this approximation into equation \ref{} we obtain:
\begin{align}
    \omega_{to} &= \omega_{s,0} - C \omega^v \frac{M}{2J} - D \omega^T \frac{3M^2-J(J+1)}{2J(2J-1)},
\end{align}
where \(\omega^{v,T} = \alpha^{s,T}/\difn{\alpha(\omega)}{\omega}{{}}\) are constants from the vector and tensor contributions. We are interested in the \(2^3S_1\) state \(\ket{J=1,M=1}\)
\begin{align}
    \omega_{to} &= \omega_{s,0} - C \omega^v \frac{1}{2} - D \omega^T \frac{1}{2},\\
    &= \omega_{s,0} + \frac{1}{2} \omega^T + \frac{1}{2} \mathcal{V} \cos \left( \theta_k \right) \omega^v - \frac{3}{2} \mathcal{P} \sin^2(\theta_k) \omega^T.
\end{align}
If we set \(\mathcal{V}=0\) and \(\mathcal{P}=0\) we obtain \(\omega_{to} = \omega_{s,0} + \frac{1}{2} \omega^T\) which is the tune-out frequency for the dynamic polarizability \(\alpha(\omega) = \alpha^s(\omega) - \frac{1}{2}  \alpha^T(\omega)\).
\section*{Calculation of \(C\) and \(D\)}
Consider light propagating along the \(z-\)axis (\textit{i.e.} light vector \(\vec{k} = \left(0,0,1\right)\)), the Jone's vector of this light is given by:
\begin{align}
    \vec{u} &= \begin{pmatrix}
    u_{0,x} \\
    u_{0,y} e^{i \phi_y} \\
    0
\end{pmatrix}.
\end{align}
By definition the stokes parameters are defined as follows (in order from first to fourth):
\begin{align}
    \mathcal{I} &= |u_x|^2 + |u_y|^2 \\ 
    \mathcal{Q} &= |u_x|^2 - |u_y|^2 \\
    \mathcal{U} &= 2 \text{Re} \left(u_x u_y^*\right) \\
    \mathcal{V} &= -2 \text{Im}\left(u_x u_y^*\right)
\end{align}
where \(\vec{u} = (u_x,u_y,u_z)\). Note we only wish to consider that normalised Jones vector/ Stokes parameters, hence \(\mathcal{I}=1=u_{0,x}^2+u_{0,y}^2\). Let the plane formed by the light vector \(\vec{k}\) and the magnetic field vector at the atoms (the quantization axis) \(\vec{b}\) be the \(y-z\) plane, hence if the angle between the light vector and the magnetic field is \(\theta_k\) then:
\begin{align}
        \vec{b} &= \begin{pmatrix}
    0 \\
    \sin(\theta_k) \\
    \cos(\theta_k)
\end{pmatrix}.
\end{align}
Now if the polarisation is rotated by an angle \(\theta_\varepsilon\) the Jone's vector becomes 
\begin{align}
    \vec{u} &= \begin{pmatrix}
    \cos(\theta_\varepsilon) u_{0,x} -\sin(\theta_\varepsilon) u_{0,y} e^{i \phi_y}\\
    \sin(\theta_\varepsilon) u_{0,x} +\cos(\theta_\varepsilon) u_{0,y} e^{i \phi_y} \\
    0
\end{pmatrix}.
\end{align}
Now to align with equations \ref{eqn:C} and \ref{eqn:D} we want to rotate the axes such that the B-field is pointing along the \(z\)-axis, this corresponds to a rotation matrix of:
\begin{align}
    R_k &= \begin{pmatrix}
 1& 0 & 0 \\
 0 & \cos(\theta_k) & -\sin(\theta_k) \\
 0 & \sin(\theta_k) & \cos(\theta_k)
\end{pmatrix}.
\end{align}
applying this we get the Jone's vector in the B-field's frame of reference (denoted \(\vec{u}'= R_k \vec{u}\) as
\begin{align}
    \vec{u}' &= \begin{pmatrix}
    \cos(\theta_\varepsilon) u_{0,x} -\sin(\theta_\varepsilon) u_{0,y} e^{i \phi_y}\\
    \cos(\theta_k) \left[\sin(\theta_\varepsilon) u_{0,x} +\cos(\theta_\varepsilon) u_{0,y} e^{i \phi_y}\right] \\
    \sin(\theta_k) \left[\sin(\theta_\varepsilon) u_{0,x} +\cos(\theta_\varepsilon) u_{0,y} e^{i \phi_y}\right] 
\end{pmatrix}.\label{eqn:Jones_mag}
\end{align}
Note that in this notation equations \ref{eqn:C} and \ref{eqn:D} are
\begin{align}
    C &= 2 \text{Im}({u'_x}^* u'_y),\\
    D &= 3|u'_z|^2 -1
\end{align}
as they have the B-field along the \(z\)-axis. Combining this with equation \ref{eqn:Jones_mag} we obtain
\begin{align}
    C&=2\cos(\theta_k)\text{Im}\left(\cos(\theta_\varepsilon)^2 u_{0,x} u_{0,y} e^{i \phi_y} -\sin(\theta_\varepsilon)^2 u_{0,x} u_{0,y} e^{-i \phi_y}\right)\\
    &= 2\cos(\theta_k)\text{Im}\left(u_{0,x} u_{0,y} e^{i \phi_y}\right)\\
    &= \cos(\theta_k)2\text{Im}\left(u_x^* u_y\right)\\
    &= -cos(\theta_k) \mathcal{V}\\
    D&=3\sin(\theta_k)^2 \left[\sin(\theta_\varepsilon)^2 u_{0,x}^2 +\cos(\theta_\varepsilon)^2 u_{0,y}^2 + 2\sin(\theta_\varepsilon)\cos(\theta_\varepsilon)u_{0,x}u_{0,y} \cos(\phi_y)\right] -1\\
    \begin{split}&=\frac{3}{2}\sin(\theta_k)^2 \large[\sin(\theta_\varepsilon)^2 u_{0,x}^2 + (1-\cos(\theta_\varepsilon)^2) u_{0,x}^2 +\cos(\theta_\varepsilon)^2 u_{0,y}^2+\\&\quad\quad\quad(1-\sin(\theta_\varepsilon)^2) u_{0,y}^2 + 4\sin(\theta_\varepsilon)\cos(\theta_\varepsilon)u_{0,x}u_{0,y} \cos(\phi_y)\large]-1\end{split}\\
    \begin{split}&= \frac{3}{2}\sin(\theta_k)^2 \large[-(\cos(\theta_\varepsilon)^2 u_{0,x}^2-2\sin(\theta_\varepsilon)\cos(\theta_\varepsilon)u_{0,x}u_{0,y} \cos(\phi_y)+ \sin(\theta_\varepsilon)^2 u_{0,y}^2) \\&\quad \quad + (\sin(\theta_\varepsilon)^2 u_{0,x}^2 + 2\sin(\theta_\varepsilon)\cos(\theta_\varepsilon)u_{0,x}u_{0,y} \cos(\phi_y) + \cos(\theta_\varepsilon)^2 u_{0,y}^2) + u_{0,x}^2 + u_{0,y}^2 \large]-1\end{split}\\
    &=\frac{3}{2}\sin(\theta_k)^2 \left[-|u_x|^2 + |u_y|^2 + 1 \right] - 1\\
    &=3 \sin^2\left( \theta_k \right) \left(\frac{1}{2} -  \frac{\mathcal{Q}}{2}\right) -1 
\end{align}
where we have used the fact that we are using the normalised Jones vector, hence \(u_{0,x}^2 + u_{0,y}^2 = 1\). \\ \\

Consider the polarisation ellipse (the ellipse that the electric field traces out) see figure \ref{fig:ellipse}. Using the ellipse notation we can write
\begin{align}
    \mathcal{Q} &= \cos(2\psi) \cos(2\chi)\\
    \mathcal{V} &= \sin(2\chi)
\end{align}
Next note that the power of a light field is proportional to its electric field amplitude squared so \(p_{min} \propto b^2\) and \(p_{max} \propto a^2\), where \(p_{min}\) and \(p_{max}\) are the minimum and maximum power transmitted through a linear polariser respectively. By trig laws we have 
\begin{align}
\cos(2\chi)&=\cos(\chi)^2-2\sin(\chi)^2\\
&= \left(\frac{a}{\sqrt{a^2+b^2}}\right)^2 - \left(\frac{b}{\sqrt{a^2+b^2}}\right)^2 \\
&= \frac{a^2-b^2}{a^2+b^2}\\
\Rightarrow \mathcal{Q}&=\frac{p_{max}-p_{min}}{p_{max}+p_{min}} \cos(2\theta_\varepsilon)\\
\sin(2\chi) &= 2\sin(\chi)\cos(\chi)\\
&=2 \frac{a}{\sqrt{a^2+b^2}} \frac{b}{\sqrt{a^2+b^2}}\\
&=\frac{2ab}{a^2+b^2}\\
\Rightarrow |\mathcal{V}| &= \frac{2\sqrt{p_{min}p_{max}}}{p_{min}+p_{max}}
\end{align}


\centering

\tikzset{
   pics/.cd,
   vector out/.style={
      code={
         \draw[#1] (0,0)  circle (1) (45:1) -- (225:1) (135:1) -- (315:1);
      }%end code   
   }%end style
}%end tikzset
\tikzset{
   pics/.cd,
   vector in/.style={
      code={
        \draw[#1] (0,0)  circle (0.25);
        \fill[#1] (0,0)  circle (.05);
      }%end code   
   }%end style
}%end tikzset
\begin{figure}
\centering
\begin{tikzpicture}[declare function={c=1.4;a=4/c;b=2/c;alpha=30;}]
\begin{scope} 
 \draw[<-,line width=0.4mm] (0,5*b/3) node[right]{$\mathbf{\hat{k}_\perp}$} -- (0,-5*b/3);
 \draw[<-,line width=0.4mm] (4*a/3,0) node[above]{$\mathbf{\hat{x}}$} -- (-4*a/3,0);
\end{scope}

\begin{scope}[dashed,rotate=50] 
 \draw[<-,line width=0.4mm] (0,5*b/3) node[right]{$\, \mathbf{\hat{y}_L}$} -- (0,-5*b/3);
 \draw[<-,line width=0.4mm] (4*a/3,0) node[above]{$\mathbf{\hat{x}_L}$} -- (-3*a/3,0);
\end{scope}

\begin{scope}[rotate=alpha] 
 \draw[line width=0.4mm,blue] (0,0) circle (a and b); 
 \draw (0,5*b/3) node[left]{minor axis} --  (0,-5*b/3) ;
 \draw (4*a/3,0) node[above]{major axis} -- (-4*a/3,0);
  \draw (-a,0) -- (0,-b); %node[pos=0.4,above]; %node[above] 
    \coordinate (a) at (0,0);
    \coordinate (b) at (-a,0);
    \coordinate (c) at (0,-b);
 \pic ["$\mathbf{\chi}$",draw, <->,radius=0.8cm, angle eccentricity=1.9]{angle = c--b--a};
 \path 
 ({a*cos(atan(-(b/a)*tan(alpha)))},{b*sin(atan(-(b/a)*tan(alpha)))}) coordinate (aux1)
 ({a*cos(atan((b/a)*cot(alpha)))},{b*sin(atan((b/a)*cot(alpha)))}) coordinate (aux2) 
 ({-a*cos(atan(-(b/a)*tan(alpha)))},{-b*sin(atan(-(b/a)*tan(alpha)))}) coordinate (aux3)
 ({-a*cos(atan((b/a)*cot(alpha)))},{-b*sin(atan((b/a)*cot(alpha)))}) coordinate (aux4); 
\end{scope}
%\draw (aux3|-aux4) rectangle (aux1|-aux2);
\path (0:1) coordinate (A) (0,0) coordinate[label={[xshift=0.3em]below left:{$ $}}] (O)
 (alpha:1) coordinate (C) 
pic ["$\mathbf{\psi}$",draw,latex-latex,angle radius=1.8cm,angle eccentricity=1.2] {angle = A--O--C};
\path (0:1) coordinate (B) (0,0) coordinate (D)
 (50:1) coordinate (E) 
pic ["$\mathbf{\theta_L}$",draw,latex-latex,angle radius=2.6cm,angle eccentricity=1.25] {angle = B--D--E};
\path (5.5/c,4/c) pic {vector in={line
width=1.5pt}} node[left]{$\hat{k}\, \, \, \,$};
\end{tikzpicture}
\begin{tikzpicture}[declare function={c=1.4;a=4/c;b=2/c;alpha=30;}]
\begin{scope} 
 \draw[<-,line width=0.4mm] (0,5*b/3) node[right]{$\mathbf{\hat{z}}$} -- (0,-5*b/3);
 \draw[<-,line width=0.4mm] (4*a/3,0) node[above]{$\mathbf{\hat{y}}$} -- (-4*a/3,0);
\end{scope}
\begin{scope}[rotate=alpha] 
 \draw[<-,line width=0.4mm,red] (0,5*b/3) node[left]{$\hat{k}$} --  (0,0);
 \draw[<-,line width=0.4mm,blue] (0.5*2.1,2.1*0.866) node[left]{$\hat{B}$} -- (0,0);
\end{scope}
\path (90:1) coordinate (A) (0,0) coordinate[label={[xshift=0.3em]below left:{$ $}}] (O)
 (120:1) coordinate (C) 
pic ["$\mathbf{\theta_k}$",draw,latex-latex,angle radius=1.6cm,angle eccentricity=1.2] {angle = A--O--C};
%\draw (aux3|-aux4) rectangle (aux1|-aux2);
\end{tikzpicture}
\caption{Diagram showing the various parameters used in Eq.~\ref{}. (left) shows the Stokes ellipse (blue) where \(\mathcal{Q} = \cos(2\psi) \cos(2\chi)\) and \(\mathcal{V} = \sin(2\chi)\) and \((\hat{x}_L,\hat{y}_L)\) represents the lab reference frame. Note that \(\hat{k}_\perp=\hat{y}\cos(\theta_k)+\hat{z}\sin(\theta_k)\) and in this case the probe beam wavevector \(\hat{k}\) is out of the page. (right) Shows how the magnetic quantization axis is assumed to be along the \(z\)-axis, with the plane spanned by it and the probe beam wavevector forms the \(z\)-\(y\) plane.}
\label{fig:ellipse}
\end{figure}

\section{Tune-Out Component Linearization}
To obtain the dependence of the net tune-out on the Stoke parameters, we perform a Taylor expansion on the dynamic polarizability components (\(\alpha^S(f)\),\(\alpha^V(f)\),\(\alpha^T(f)\)) in terms of frequency about the zero point of \(\alpha^S(f)\), which we denote \(f^{S}_{TO}\),
\begin{align}
    \alpha^J(f) &= \alpha^J(f^{S}_{TO}) + \left. \derivn{\alpha^J}{f}{{}}(f-f^{S}_{TO})\right|_{f=f^{S}_{TO}} + \alpha^J(f^{S}_{TO}) +\left. \difn{\alpha^J}{f}{2}\right|_{f=f^{S}_{TO}}(f-f^{S}_{TO})^2 + ...,
\end{align}
where \(J=\{S,V,T\}\). To make the analysis tractable we truncate Eq. ... at ... order. We wish to truncate the various polarizability components to the lowest order


 Substituting this expansion into Eq.~(\ref{eq:polarizability_full}) we obtain,
\begin{align}
    \alpha(f) &\approx \derivn{\alpha^S(f)}{f}{{}}(f-f^{S}_{TO}) 
    -\frac{1}{2} \alpha^V(f) \cos \left( \theta_k \right) \mathcal{V}  + 
    \frac{1}{2} \alpha^T(f) \left[3 \sin^2\left( \theta_k \right) \left(\frac{1}{2} +  \frac{\mathcal{Q_{A}}}{2}\right) -1 \right]  . \label{eq:main_to}
\end{align}

We wish to determine the quantity \(f_{TO}\), at which the net polarizability vanishes \(\alpha(f_{TO})=0\).

As changes in the total polarization \(\alpha(f)\) near the tune-out come predominantly from the scalar polarizability, we can make the approximation that the vector and tensor polarizability terms are constant over the range of interest. Setting \(\alpha^V(f)\approx\alpha^V, \alpha^T(f)\approx\alpha^T,\) and \(\derivn{\alpha^S(f)}{f}{{}} \approx \derivn{\alpha(f)}{f}{{}} \),
we can make a further simplification in the form, 
% \begin{align}
%     f-f_{TO} &= 
%     \left(
%         \derivn{\alpha^S(f)}{f}{{}} \bigg/ \derivn{\alpha(f)}{f}{{}}
%     \right)
%     (f-f^{S}_{TO}) 
%     %
%     -\frac{1}{2} 
%     \left(\alpha^V(f) \bigg/\derivn{\alpha(f)}{f}{{}}\right) 
%      \cos \left( \theta_k \right) \mathcal{V}
%     + \\  \nonumber
%     %
%     & \qquad \frac{1}{2} 
%     \left(
%         \alpha^T(f)\bigg/\derivn{\alpha(f)}{f}{{}} 
%     \right) 
%     \left(3 \sin^2\left( \theta_k \right) \left(\frac{1}{2} +  \frac{\mathcal{Q_{A}}}{2}\right) -1 \right) 
%  \end{align}   
 \begin{align}
    f_{TO} &= 
    f^{S}_{TO}
    %
    +\frac{1}{2}\beta^V  \cos \left( \theta_k \right) \mathcal{V_{A}}
    %
    - \frac{1}{2}\beta^T  \left[3 \sin^2\left( \theta_k \right) \left(\frac{1}{2} +  \frac{\mathcal{Q_{A}}}{2}\right) -1 \right], \label{eqn:tune_out_eq} 
\end{align}
where
 \begin{align}
  \beta^V &= \alpha^V(f^{S}_{TO}) \bigg/ \left. \pderivn{\alpha^S(f)}{f} \right|_{f=f^{S}_{TO}} \mathrm{, and}\\
  \beta^T &= \alpha^T(f^{S}_{TO}) \bigg/ \left. \pderivn{\alpha^S(f)}{f} \right|_{f=f^{S}_{TO}} .
 \end{align}
If we set \(\mathcal{V_{A}}=0\) and \(\mathcal{Q_{A}}=-1\) we obtain \(f_{TO}(-1,0) = f^{S}_{TO} + \frac{1}{2} \beta^T\) which is the tune-out frequency for the dynamic polarizability \(\alpha(f) = \alpha^S(f) - \frac{1}{2}  \alpha^T(f)\), which is independent of the magnetic field pointing.

\end{document}
