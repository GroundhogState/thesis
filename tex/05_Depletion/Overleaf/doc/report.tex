\documentclass[%
 reprint,
 amsmath,amssymb,
 aps,
]{revtex4-1}

\usepackage{graphicx}% Include figure files
\usepackage{dcolumn}% Align table columns on decimal point
\usepackage{bm}% bold math

% \newcommand{\bf}{\textbf}

\begin{document}


\title{Determining the Tan contact coefficient}% Force line breaks with \\

\author{J Ross and friends}



\begin{abstract}

1)
What did you do and what happened?
What do the results mean?
What problem or issue is being addressed?
Why should someone read this paper?
2)
LOCATE, FOCUS, REPORT, ARGUE
\end{abstract}

\maketitle

\section{Scraps}
So this depletion thing. Who cares and so what? Well, that sort of depends on what comes out of the literature review. So for the most part it seems that we have a sort of good handle on the depletion things BUT there are some arguments that it should be invisible in far-field (assuming adiabatic expansion) but the french saw heaps. We can explain the excess in terms of a systematic effect - the presnece of extra magnetic states. Uncorrected backgorund noise/systematic effect. However, we do still observe something consistent with depletion being present, and so this verifies bogoliubov but means that something in the argument about the trap expansion should be wrong. I will need to double check the assumptions but I suppose it is that of adiabatic expansion - the trap switch off is likely quick enough (c.f. coil currents vs trap frequencies) to be a projective action, I suppose, non-adiabatic, so the projection onto free-particle eignstates is such that is preserves the in-trap momentum distribution quite well, apparently in accordance with Bogoliubov - it may be an adiabatic expansion after that, and we can't rule out the possibility of smethign happening in the cloud separation process, but it does look like bogo is very good, but need to re-examing whta piotr's simulations say. So I'll email him today and let him know there are some exciting results - but to hang on because they're still pending. Who cares? Cold atom scientists? Like, who cited the works on quantum depletion so far? That would be good to investigate. Also people keen on Contact stuff - we should look into that also, see if/how relevant that stuff is. So a task for soon: look at the future citation tree!


\section{Outline}

\subsection{Background}

\subsection{Results}

\subsection{Methods}

\subsection{Discussion}


\section{\label{introduction}Detector flux model}
\subsection{Definition}
In the image obtained from the MCP-DLD detector of the shots in which the quantum depletion is observed, the pointwise density of atom detection events in momentum space can be described by

\begin{equation}
    n(\textbf{k}) = \sum_{m=1}^3 n_m(\textbf{k},N,\omega_{x,y,z},T) + \delta(\textbf{k}) + \lambda(N_0,\bar{\omega}\textbf{k})\Theta(-k_z)
\end{equation}

The first terms $n_m$ refer to the detected density of atoms from the $m_J=m$ condensates, which depends on the momentum vector $\textbf{k}$ on the total atom number $N$, the temperature $T$, and harmonic trapping frequencies $\omega_{x,y,z}$. 
The dark count rate $\delta$ is momentum-dependent due to its non-uniformity on the detector. 
The measured trap leak rate $\lambda$ stops when the trap is released, hence the Heaviside $\Theta$ function ensures this term 
only contributes on the lower side of the falling BEC. 
Below we adopt the notation $(\delta+\lambda) = \delta + \lambda\Theta(\theta)$ where $\theta$ is the elevation angle from the plane of the detector subtending from the centre of the condensate.
Details of the measurement and calibration of each of these terms is discussed in section \href{calibration}.


\subsection{Contributions to the model}

The count density from the falling BECs depend explicitly on the trapped population and on the trapping frequencies. 
This is most obviously true for the condensate of atoms in the $m_J=0$ state, and also for the $m_J=\pm 1$ condensates when all three condensates land undisturbed on the detector during calibration of the transfer efficiency.

In the depletion detection shots, we have observed a remaining presence of $m_J=1$ atoms, the cause of which is unclear but is suspected to be due to a collision process with a feature inside the chamber. 
Regardless of the cause, we calibrate for this contamination (also called \textit{spin mixing} elsewhere). 
The count density of the $m_J=1$ states depends on the trapping frequencies also - this could be because the trap centre shifts with changing trap frequencies, changing the details of the spin mixing process during freefall. 
We cannot rule out the presence of $m_J=-1$ on the detector also. Calibrating for this contribution is less straightforward, as discussed below. 
The temperature dependence of the condensate density is expected to manifest only as a change in the thermal fraction - calculation of the thermal depletion contributions are pending. 

The contributions from the dark count rate are assumed to be time-invariant and independent of any properties of the condensate itself. 
There may be a change in the dark count rate while the detector is subject to high atom fluxes, but these are not expected to be a problem for the following reasons: 
One, we observe very similar thermal tails both above and below the condensate, suggesting that, at least, the quantum efficiency and dark count rates are not significantly different. 
Two, although there are some temporary hotspots on the detector during the peak BEC flux, these are only observed co-temporally with the falling BEC. 
The quantum depletion is detected far beyond the regions where this effect is noticable, and so they are not expected to contaminate the signal. 

The leak rate should be expected to be dependent on the BEC number. 
The most simple model would be an exponential decay of BEC number with some collection efficiency by the detector. 
There may also be density-dependent collision effects feeding the trap leak rate driven by changing the trapping frequencies, which could also alter any spatial structure in the leak rate due to a shift in trap position. 

\subsection{Obtaining the Tan constant}

The far-field density of a harmonically trapped condensate is described by contributions from the condensed, thermal, and depleted populations,

\begin{equation}
        n(\textbf{k}) = n_{BEC}(\textbf{k}) + n_{T}(\textbf{k}).
\end{equation}

The momentum density of the thermal fraction is [Pitaevskii \& Stringari]
$$
n_{T}(\textbf{k}) = \frac{1}{\left(\lambda_T m \bar{\omega}\right)^3} g_{3/2}\left(e^{-\beta k^2/2m\hbar^2}\right)
$$
where $T$ is the temperature, $\lambda_T = h^2/\sqrt{2\pi m k_B T}$ is the thermal de Broglie wavelength, $\bar{\omega}$ is the geometric mean of the trapping frequencies, $\beta=1/ k_B T$ is the thermodynamic Beta, and $g_{3/2}(z)$ is the Bose integral 
$$
g_{3/2}(z) = \frac{2}{\sqrt{\pi}}\int_0^\infty \frac{\sqrt{x}}{z^{-1}e^x-1}=\sum_{l=1}^\infty \frac{z^l}{l^{3/2}}.
$$

The short-wavelength density of the BEC momentum density is described by the Thomas-Fermi approximation. The asymptotic momentum distribution is that of the quantum depletion,
$$
n(\textbf{k}) = \frac{C_\infty}{(2\pi)^3 k^4},
$$

where $C_\infty$ is the universal Tan constant, defined by $\mathcal{C}_{\infty}=\textrm{lim}_{|\textbf{k}|\to\infty}(2\pi)^3 |\textbf{k}|^4 n(\textbf{k})$, and for a harmonically trapped Bose gas is $\frac{64\pi^2}{7}a_{s}^{2}N_0 n_0$ in the local density approximation [Chang et al].

The asymptotic behaviour of the single-particle probability density function for a particle in the ground state (not in the thermal fraction) is therefore $\frac{64\pi^2}{7}a_{s}^{2} n_0$, hence the contact constant can be seen as a parameter defining a probability distribution, which we estimate using the procedure described in the next section.

\section{\label{calibration}Calibration}


We assume that the RF pulse and magnetic separation of the condensates does not affect their far-field distribution. 
This may be false if the inter-condensate scattering rate is high (calculation/reference pending).
The RF pulse transfers atoms into the different $m_J$ states with efficiency $\eta_J$, therefore we estimate the momentum density of the initial condensate via

\begin{equation}
    n(k) = \frac{1}{\eta_0}\left(\bar{n}_0 - \eta_1\bar{n}_1 - (1+\eta_1)(\delta+\lambda)\right)
\end{equation}

where $\bar{n}_0$ and $\bar{n}_1$ refer respectively to the count densities obtained from the measurement and spin mixing calibration shots. We produce a background calibration for $(\delta+\lambda)$ by defining background density in the upper and lower hemisphere of $\bar{\delta}_\pm$ to be the backgrounds obtained from the dark count and leak calibrations, respectively.
We determined the transfer efficiencies $\eta_J$ from the initially released condensate to the other $m_J$ states by separating the clouds with a magnetic field gradient generated by auxiliary field coils. (Refinement pending).
We calibrate the dark count rate by building a spherical histogram centred at the same position on the detector as the $m_J=0$ condensate, but 2 seconds after the condensate impact and before the trap loading sequence begins, and calibrate the trap leak similarly but centering the histogram about a point earlier in time than the BEC impact. 
We calibrate the contamination by stray $m_J=1$ counts by running the depletion measurement sequence without the RF transfer, obtaining the distribution of stray counts, plus the background count rate. 
By subtracting our empirical measurement of the background, we obtain the estimated density of the stray counts, $n_1 = \bar{n}_1 - (\delta+\lambda)$.

The total atom number $N$ and trapping frequencies $\omega_i$ are determined by a pulsed atom-laser measurement, which we describe in detail in [another publication]. 
A fit of the empirical thermal fraction produces a measurement of the temperature, through which we may determine the condensed number $N_0$.

Average dark count rate is determined to be $5.6228E3 Hz/m^2$, or $9.3713E-16 (m^{-1})^{-3}$. The latter is calculated by converting 1 sec interval into a distance c.f. the centre of mass velocity of the BEC - about 4m or 4E6 micron. 

\subsection{SNR and uncertainty}
This section concerns the calculation of two related quantities: The signal and noise (power, as wavefunction squared? Or amplitude, as number of atoms? Well, probably the latter until we reformulate.), and the errors associated with each of these. 

Using the terms defined above, the SNR is
\begin{equation}
    \textrm{SNR} = \frac{n_0(k)}{\eta_0\left(\eta_1\bar{n}_1 - (1+\eta_1)(\delta+\lambda)\right)}
\end{equation}

Notice that this assumes there are no additional contributions - we'll see how to deal with this as we move forward.

A first estimate of the fractional uncertainty of this signal is obtained via propagation of errors:

\begin{equation}
\begin{split}
    \left(\frac{\sigma}{\textrm{SNR}}\right)^2 =& (\left(\frac{\sigma_{\eta_1}}{\eta_1}\right)^2+\left(\frac{\sigma_{\eta_0}}{\eta_0}\right)^2+\cdots\\
                                &\left(\frac{\sigma_n_0}{n_0}\right)^2+\left(\frac{\sigma_n_1}{n_1}\right)^2+\cdots\\ 
                                &\left(\frac{\sigma_A}{A}\right)^2+\left(\frac{\sigma_A}{A}\right)^2
\end{split}
\end{equation}

So, yeah, we need to find the uncertainty in each of these terms. Plot 'em.



\section{\label{results}Results}

\section{\label{tasks}Future tasks}

\section{\label{conclusion}Conclusion}



George Whitesides addressed some writing styles.\cite{Whitesides2004}

\medskip

\begin{thebibliography}{9}
\bibitem{Whitesides2004}
Whitesides, G.M.
\textit{Whitesides' Group: Writing a Paper}.
Advanced Materials, 16(8):1375--1377, 2004.
\end{thebibliography}




\end{document}
