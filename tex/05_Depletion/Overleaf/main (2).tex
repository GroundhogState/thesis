\documentclass{article}
\usepackage{natbib}
\usepackage{fullpage}
\usepackage[utf8]{inputenc}
\title{Quantum depletion}
\author{JR}
\begin{document}
\maketitle

\section{Introduction}
Interactions between particles are a fact of life that theory must confront in attempts to phenomena. Modern experimental techniques allow access to a large range of interaction strengths, sometimes within otherwise identical systems. For example, optical lattices or Feshbach resonances can be used to tune interaction parameters between extreme values, from the so-called strongly correlated regime (of great interest in itself) to extremely weak or even interaction-free (?) regimes. A common element in many of these experiments is the use of Bose-Einstein condensates (BECs) or degenerate Fermi gases as test systems for theoretical descriptions of interacting gases. 

The criterion for condensation is, roughly speaking, when most of the particles in a dilute bosonic gas occupy the same quantum state. More formally this is captured by the Penrose-Onsager criterion, wherein a homogeneous gas is said to be condensed if there is a macroscopic population of the ground-state eigenvalue of the ensemble density matrix. Any particles not in the ground state are said to be part of the depleted fraction. The depletion of the condensate is comprised of two parts, the thermal depletion and the quantum depletion. The thermal depletion is an artefact of the finite temperature of the condensate and has a number distribution described by Bose-Einstein statistics. The quantum depletion is a consequence of particle interactions/quantum fluctuations?



What is quantum depletion?
Why is quantum depletion interesting?
What is the state of current knowledge?
What does this paper contribute?
How did we make our measurements?
How do we interpret the results in context of current research?
What else should we measure to further understand?

Equilibration time scales 
https://www.sciencedirect.com/science/article/pii/S037843710201350X homogeneous trap
https://arxiv.org/pdf/1607.06939.pdf !! 
https://journals.aps.org/prl/pdf/10.1103/PhysRevLett.77.5315
https://journals.aps.org/pra/pdf/10.1103/PhysRevA.54.R1753

estimating correlations

https://arxiv.org/pdf/astro-ph/0310831.pdf
https://arxiv.org/abs/1305.4613

Interacting BEC in harmonic trap => Penrose criterion not valid
Instead, depleted (/excited) fraction and condensate fraction coexist
Two different momentum density profiles, depletion driven by collision?

Applicability of Bogoliubov theory in optical lattices
https://www.nature.com/articles/nphys1476




\section{From \cite{Makotyn}:}
Strongly-interacting systems like neutron stars and High-T superconductors are cool. Liquid Helium was the first discovered but hard to probe, so ultracold gases have stepped up instead.

This paper did not observe the expected $k^{-4}$ tail, which would be expected to arise from two-body short-range interactions. The amplitude of this tail is the parameter known as the contact (ref 33); they do concede the tail may exist below their sensitivity as their SNR is bad at high momentum. The unitary experimental conditions might also lead to three-body interactions affecting the tails. 

Check: Did they observe in-trap or after release?
\section{From \cite{PSText}:}
Quantum depletion in a harmonic trap given by 
$$\frac{\delta N}{N_0} = \frac{5\sqrt{\pi}}{8} \sqrt{a^3 n(0)}$$
Which is fixed by the gas parameter (number density) in the centre of the trap. This conflicts with the result quoted below!

\section{From \cite{Muller}:}
The authors explicitly compute the quantum depletion contribution to the momentum distribution in external potentials. The so-called `potential depletion' is a small correction to the homogeneous depletion. The Penrose-Onsager criterion is only relevant for homogeneous condensates, hence this paper's use of Bogoliubov theory to determine the one-body density matrix. This means writing the atom field as a mean-field plus perturbation, which gives another criterion for condensation in non-homogeneous condensates. The momentum distribution similarly splits into $n_\textbf{k} = |\Phi_{\textbf{k}}|^2 + \langle \delta \Psi_\textbf{k}^\dagger \delta \Psi_\textbf{k}\rangle = n_{c\textbf{k}} + \delta n_\textbf{k}$, contributions from the condensed and non-condensed fraction respectively. In section 4 of the paper, the quantum depletion is computed. The authors note that, as shown here, the population of the zero-momentum state is not a good indicator of BEC in homogeneous systems. That is, the number of particles in a nonzero momentum state is NOT a measure of the depletion; the integral of the momentum fluctuations about the mean-field, however, is. 

\subsection{Quantum depletion of the condensate}
The quantum depletion is a finite non-condensed fraction at zero temperature, arising from quantum fluctuations around the mean-field approximation to the condensate. The deformation of the condensate by the mean-field term in the GP equation does not lead to depletion. The authors comment that the depleted fraction may be \emph{less}, and not \emph{more} quantum than the condensate. Nonetheless: the depletion that follows is the quantum depletion as opposed to the thermal depletion: It is given by integrating the momentum fluctuations
$$\delta n = L^{-d} \sum_{\textbf{k}} \delta n_{\textbf{k}}$$

The Bogoliubov approximation is justified whenever $n a_{s}^3$ is small, i.e. low density or weak scattering, when the depleted fraction is small.  In this case, in three dimensions, the depleted fraction is $8(n a_s ^3)^{1/2} / 3 \pi^{1/2}$ in terms of the s-wave scattering length, or $n\xi^d >>1$, with the healing length$\xi = \hbar/\sqrt{2mgn_c}$ and $g=4\pi \hbar^2 a_s/m$.

The potential depletion turns out to be (small and) proportional to the homegenous depletion. The most relevant result here is the depleted fraction expression given above - there are no results given for potential depletion in a harmonic trap, although one could use their methods to compute this. 

N.B. This paper also computes the momentum distribution and depletion of a condensate in a lattice, of interest to the upstairs experiment. 

\section{From \cite{Sirjean}:}
The authors find that the ionization rate is almost solely responsible for the decay of the confined condensate. They characterize the two- and three-body contributions to the ionization loss rate, and find that quantum depletion makes a significant contribution to the three-body rate. The two- and three-body rates should be proportional to $a^2$ and $a^3$ respectively. The two-body rate they found was well-fitted for the by a quadratic, but the three-body rate was found to be $L\approx L_{20} (\frac{a}{20})^3 (1-0.21\frac{a-20}{20})$, where they assumed $a=20$nm (wrong), and it's not clear what the subscript 20 on the L means. This paragraph is quite vague and has no citations, but if correct suggests quantum depletion matters to collision rates. Their fitted rates for two- and three-body collisions are of order 10E-14 cm$^3$/sec and 10E-26 cm$^6$/sec respectively. They state that their assumed detector efficiency means a=10nm would require additional loss mechanisms for half the losses, or imply an overestimate of the detector efficiency. They quote an estimated 40\%! So given the recent measurements of scattering length, that seems very likely. (then again they were detecting ions, not neutral atoms)

The atom number was measured by MCP. To avoid saturation, they lower the MCP gain and record the signal in analog mode with a 400$\mu s$ time constant. They also leave their RF knife on at the end of a sequence and maintain a very low thermal fraction. 

\section{From \cite{Lopes}:}
The authors measure quantum depletion in a homogeneous condensate. By using a Feshbach resonance to tune the interactions, they are able to control the depleted fraction. They measure the condensed fraction by Bragg momentum spectroscopy. The results are qualitatively similar to theory, however the theory at T=0 predicts lower quantum depletion than was observed at finite temperatures. The thermal depletion also populates the high-momentum tails and is not scattered out by the Bragg process Their measurements are consistent with a small nonzero temperature, consistent with their trap depth but they do not quote doing actual thermometry. 15\% and 20\% statistical and systematic error. 

In theory referenced, depletion of a homogeneous condensate is found to be proportional to $\sqrt{na^3}$ and valid for $na^3<\approx 10^-3$. Refs 15-17 measure depletion by using lattices to increase interactions, and through expansion of dilute gas in 18 (i.e. the paper we are concerned with)

Authors use K-39 BEC and separate the BEC from high-momentum components by Bragg two-photon process. Demonstrated dependence on scattering length by Feshbach resonance and showed that depleted fraction depends (reversibly) on scattering length.

See refs 8, 9, for a connection to superconducitivity, superfluidity and condensation.
Side note: Connection of superfluidity to BEC was proposed by Londin and Tisza. 


\section{From \cite{Qu}:}
Theoretical study of time dependence of density and momentum distribution of expanding gases, includding Bose case. They show that the $k^{-4}$ tal vanishes for large expansion times, in conflict with the work of Chang et al.

Suggestive links between short-range wavefunction behaviour and thermodynamics via Tan contact (ref 1-7). Ref 1 for Tan defn. Ref 11-12, measure momentum tail after expansion by tuning out interactions. 

Claim: Momentum distribution affected by interactions during expansion, which is expected from the ballistic relation given the momentum distribution. Expansion dynamics of momenta smaller than the inverse healing length are well studied (by scaling relations or hydrodynamic approx) but higher momentum not so. Section II and the conclusion are the only interesting parts.

\subsection{Many-body interacting systems b) weakly repulsive Bose gas}

The contact evolves as $C(t) = \int d\textbf{r} (16\pi ^2 \rho^2(\textbf{r},t)a^2)$ where $a$ is the s-wave scattering length and the integrand is the local contact density at equilibrium (from Bogoliubov theory). They use the scaling transformation from hydrodynamic theory to examine the evolution of the contact, which shows it decays with $C_0 /(\omega{ho} t)^3$, where $C_0$ is the initial contact. They open their argument with ``The adiabatic ansatz is expected to hold'' - seems dodgy. 

The fourth-power tail is observed in the section on Fermi gases, but is ruled out by arguing that it implies infrared divergence. (Recall Stringari comment - you have a natural cutoff which prevents this? But this is in thermodynamic limit maybe?) - the fourth-power tail also vanishes explicitly in the two-body case, but there's no strong result here! 


\bibliographystyle{plainnat}
\bibliography{bibliography}

\end{document}

%Setting up expt

