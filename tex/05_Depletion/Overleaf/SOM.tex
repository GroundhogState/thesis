%\section*{Supplementary Materials}

%\comment{sections just dumped from main text for now - clean up later}



%Our bespoke software interfaces with the LabView control environment and loops over a sequence of experiments, consisting in this case of: Atom number calibration, Trap frequency calibration, and RF transfer efficiency, followed by 10 data collection runs.

%To calibrate the atom number, we used an RF pulse sequence to create a pulsed atom laser with intensity well below the saturation threshold of our delay-line detector. We corrected the number of detected atoms to account for the quantum efficiency ($\sim$8\%) of our detector. 
%\comment{Should use QD measurements to compute condensate/thermal fraction relative sizes and further improve $N_0$ measurement}.
%\comment{how does this compare to our trap binning? What is the uncertainty of the QE?}

%Together with the atom number, a measurement of the centre-of-mass motional frequencies allowed us to compute the peak density of the condensate. We measured the trap frequency by transient magnetic field to the trapped condensate for $\sim 150\mu s$, forcing centre-of-mass motion in the lab frame. We used the same pulsed outcoupling scheme as in the number measurement, and measured the  centre-of-mass momentum of the cloud at the time of each pulse. Our outcoupling frequency is well below the natural frequency of the confining potential so we used a subsampled reconstruction method to recover the trap frequency. Prior to running experiments we varied the outcoupling pulse frequency to determine in which Nyquist zone the actual trap frequency lies relative to our sampling frequency. We can then obtain the trap frequency to within precision X in a single shot of the experiment by taking a Fourier transform of the centre of mass motion.

%Our calibration sequence provided accurate number and peak density measurements along with the condensate profile. We verified the spherical symmetry of the thermal and depleted fractions, and so compute the momentum density by an average over a large section of the momentum distribution. 

%We verified the fitting method by analysing a test data set of known parameters, which was generated by the Metropolis-Hastings algorithm, and find the method recovers the test set parameters within a factor less than our experimental uncertainties.  


%There may still be errors in the analysis. The fitting procedure apparently produces a good fit, but it may be possible to get indistinguishable or better fits by fixing the exponent of the power law to be some $\alpha\neq 4$. This would challenge the assertion that the cause of the observed profile is in fact quantum depletion.  Assuming the observed data does follow a power law, statistical fluctuations in the high-momentum region become amplified by the $k^4$ scaling procedure and could introduce further error in the estimated Tan constant. \comment{Y-error bars are computed in a fairly ad-hoc way, so ours and Chang may be drastically under-reported...)}Reliably estimating the parameters of a power-law distribution is nontrivial\cite{Clauset2009}, and the robustness of the fitting method above has not been exhaustively tested. The LDA prediction is essentially that the single-particle wavefunctions, hence the single-particle probability distribution over momentum, are affected by many-body effects. One computes the single-particle probability density by dividing the condensate density by the number of particles (hence the $N_0$ term in the LDA), and then the problem of extracting the contact parameter from the data is an exercise in statistical parameter estimation. As shown by Clauset et al \cite{Clauset2009}, fitting these functions tends to produce incorrect estimates of parameters. The method outlined in Clauset was proven to asymptotically converge with probability 1 to the correct fit parameters of a power law distribution, but would need to be extended to derive a maximum likelihood estimator for a power law overlaid on a constant background.